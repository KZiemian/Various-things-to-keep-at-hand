% ---------------------------------------------------------------------
% Podstawowe ustawienia i pakiety
% ---------------------------------------------------------------------
\RequirePackage[l2tabu, orthodox]{nag} % Wykrywa przestarzałe i niewłaściwe
% sposoby używania LaTeXa. Więcej jest w l2tabu English version.
\documentclass[a4paper,11pt]{article}
% {rozmiar papieru, rozmiar fontu}[klasa dokumentu]
\usepackage[MeX]{polski} % Polonizacja LaTeXa, bez niej będzie pracował
% w języku angielskim.
\usepackage[utf8]{inputenc} % Włączenie kodowania UTF-8, co daje dostęp
% do polskich znaków.
\usepackage{lmodern} % Wprowadza fonty Latin Modern.
\usepackage[T1]{fontenc} % Potrzebne do używania fontów Latin Modern.



% ---------------------------------------
% Podstawowe pakiety (niezwiązane z ustawieniami języka)
% ---------------------------------------
\usepackage{microtype} % Twierdzi, że poprawi rozmiar odstępów w tekście.
% \usepackage{graphicx} % Wprowadza bardzo potrzebne komendy do wstawiania
% grafiki.
% \usepackage{verbatim} % Poprawia otoczenie VERBATIME.
% \usepackage{textcomp} % Dodaje takie symbole jak stopnie Celsiusa,
% wprowadzane bezpośrednio w tekście.
\usepackage{vmargin} % Pozwala na prostą kontrolę rozmiaru marginesów,
% za pomocą komend poniżej. Rozmiar odstępów jest mierzony w calach.
% ---------------------------------------
% MARGINS
% ---------------------------------------
\setmarginsrb
{ 0.7in}  % left margin
{ 0.6in}  % top margin
{ 0.7in}  % right margin
{ 0.8in}  % bottom margin
{  20pt}  % head height
{0.25in}  % head sep
{   9pt}  % foot height
{ 0.3in}  % foot sep



% ---------------------------------------
% Często używane pakiety
% ---------------------------------------
% \usepackage{csquotes} % Pozwala w prosty sposób wstawiać cytaty do tekstu.
\usepackage{xcolor} % Pozwala używać kolorowych czcionek (zapewne dużo
% więcej, ale ja nie potrafię nic o tym powiedzieć).



% ---------------------------------------
% Pakiety do tekstów z nauk przyrodniczych
% ---------------------------------------
% \let\lll\undefined % Amsmath gryzie się z językiem pakietami do języka
% % polskiego, bo oba definiują komendę \lll. Aby rozwiązać ten problem
% % oddefiniowuję tę komendę, ale może tym samym pozbywam się dużego Ł.
% \usepackage[intlimits]{amsmath} % Podstawowe wsparcie od American
% % Mathematical Society (w skrócie AMS)
% \usepackage{amsfonts, amssymb, amscd, amsthm} % Dalsze wsparcie od AMS
% % \usepackage{siunitx} % Dla prostszego pisania jednostek fizycznych
% \usepackage{upgreek} % Ładniejsze greckie litery
% % Przykładowa składnia: pi = \uppi
% \usepackage{slashed} % Pozwala w prosty sposób pisać slash Feynmana.
% \usepackage{calrsfs} % Zmienia czcionkę kaligraficzną w \mathcal
% % na ładniejszą. Może w innych miejscach robi to samo, ale o tym nic
% % nie wiem.





% ---------------------------------------
% Dodatkowe ustawienia dla języka polskiego
% ---------------------------------------
\renewcommand{\thesection}{\arabic{section}.}
% Kropki po numerach rozdziału (polski zwyczaj topograficzny)
\renewcommand{\thesubsection}{\thesection\arabic{subsection}}
% Brak kropki po numerach podrozdziału



% ---------------------------------------
% Pakiety napisane przez użytkownika.
% Mają być w tym samym katalogu to ten plik .tex
% ---------------------------------------
\usepackage{latexgeneralcommands}
% \usepackage{mathshortcuts}



% ---------------------------------------
% Ustawienia różnych parametrów tekstu
% ---------------------------------------
\renewcommand{\arraystretch}{1.2} % Ustawienie szerokości odstępów między
% wierszami w tabelach.





% ---------------------------------------
% Pakiet „hyperref”
% Polecano by umieszczać go na końcu preambuły.
% ---------------------------------------
\usepackage{hyperref} % Pozwala tworzyć hiperlinki i zamienia odwołania
% do bibliografii na hiperlinki.










% ---------------------------------------------------------------------
% Tytuł, autor, data
\title{Pozycje do przeczytania}

% \author{}


% \date{}
% ---------------------------------------------------------------------










% ####################################################################
% Początek dokumentu
\begin{document}
% ####################################################################





% ######################################
\maketitle  % Tytuł całego tekstu
% ######################################





% ######################################
\section{Przeczytaj}

\vspace{\spaceTwo}
% ######################################



% ##################
\begin{enumerate}

\item Biblia;



\item Gene Wolf, \textit{Cień kata};



\item Roger Scruton, \textit{Zielona filozofia};



\item Andrew Klavan, \textit{Prawdziwa zbrodnia};



\item \textit{Teoria pomiarów};



\item Wojciech Roszkowski, \textit{Świat Chrystusa. Tom I};



\item Red. L. A. Steen, \textit{Matematyka współczesna. Dwanaście
    esejów};



\item S. J. Gould, \textit{Niewczesny pogrzeb Darwina};



\item Olga Tokarczuk, \textit{Bieguni};



\item Hanya Yanagihara, \textit{Małe życie};



\item Homer, \textit{Iliada}, \textit{Odyseja};



\item Ks. Jelonek, \textit{Wprowadzenie do Biblii};



\item Gustaw Meyrink, \textit{Golem};



\item Tomasz Mann, \textit{Dokotor Faustus};



\item Roger Scruton, \textit{Przewodnik po~kulturze współczesnej dla
    inteligentnych};



\item Radek Rak, \textit{Baśń o wężowym sercu albo wtóre słowo o Jakubie
    Szeli};



\item Alain Besancon, \textit{Anatomia widma};



\item Red. E. Tarkowska, \textit{Zrozumieć biednego. O~dawnej i~obecnej
    biedzie w~Polsce};



\item Mario Vatgas Lloysa, \textit{Rozmowa w katedrze};



\item Juan Gabriel Vasquez, \textit{Kształt ruin};



\item Marquez, \textit{Miłość w czasach zarazy};



\item Wiesław Myśliwski, \textit{Traktat o łuskaniu fasoli},
  \textit{Widnokrąg}, \textit{Kamień na kamieniu};



\item Bourbaki, \textit{Elementy historii matematyki};



\item Platon, \textit{Państwo};



\item Arystoteles, \textit{Etyka Nikomochejska};



\item \textit{Recepcja w~Polsce nowych kierunków i~teorii naukowych};



\item Herodot;



\item H. Steinhaus;



\item \textit{Liryka polska. Interpretacje};



\item \textit{Zarys dziejów filozofii w~Polsce};



\item \textit{Państwo Boże};



\item \textit{Drabina Raju};



\item Tomas Halik, \textit{Co nie jest chwiejne jest nietrwałe};



\item Bernard Korzeniewski, \textit{Trzy ewolucje};



\item R. Krasowski;



\item Benedict, \textit{Wzory kultury};



\item I. Bokwa, \textit{Wprowadzenie do teologii Karla Rahnera};



\item R. Brandstaetter, \textit{Patriarchowie};



\item G. K. Chesterton, \textit{Ortodoksja};



\item M. Davies, \textit{Liturgiczne bomby zegarowe Vaticanum II.
    Zniszczenie katolickiej wiary przez zmiany w katolickim kulcie};



\item Michał Heller, \textit{Filozofia przyrody};



\item A. MacIntyre, \textit{Dziedzictwo cnoty};



\item M. Takesaki, \textit{Theory of operator algebras};



\item R. Penrose, \textit{Droga do rzeczywistości};



\item P. Johnson, \textit{Historia świata XX wieku};



\item M. Dzielski;



\item A. Zamoyski, \textit{Własną drogą};



\item P. Johnson, \textit{Narodziny nowoczesności};



\item R. Terlecki, \textit{Dzieje sowieckiej kolonii};



\item B. Cywiński, \textit{Rodowody niepokornych};



\item A. Nowak, \textit{Od Polski do postpolityki};



\item P. Gay;



\item A. Leder, \textit{Prześniona rewolucja};



\item \textit{Oświecenie dzisiaj};



\item E. J. Hobsbawn, \textit{Tradycje wynalezione};



\item \textit{Słowacki. Szat-Anioł};



\item T. Enderson, \textit{Kultura popularna, tożsamość narodowa i~życie
    codzienne};



\item A. Friszke, \textit{Rok 1989~r.};



\item R. Legutko, \textit{Esej o~duszy polskiej};



\item K. Brodacki, \textit{Trzy twarze Juliana Haraschina};



\item J. Kurtyka, \textit{Z dziejów agonii i~podboju. Prace zebrane
    z~zakresu najnowszej historii Polski};



\item G. Kucharczyk, \textit{Polska myśl polityczna po 1939~r.};



\item A. L. Sowa, \textit{Historia polityczna Polski 1944-1991};



\item K. Janicki, \textit{Epoka hipokryzji. Seks i~erotyka
    w~przedwojennej Polsce};



\item A. Wielomski, \textit{Prawica w~XX wieku};



\item N. Ferguson, \textit{Niebezpieczne związki};



\item W. Roszkowski, \textit{Najnowsza historia Polski};



\item Frank E. Manuel, (książka o~religioznawstwie);



\item Urs Alterman;



\item \textit{Chrześcijaństwo, demokracja, kapitalizm};



\item J. Aumont, M. Marie, \textit{Analiza filmu};



\item J. Osterhammel, \textit{Historia XIX wieku. Przebudowa świata};



\item P. Holmes, \textit{Wiek cudów};



\item E. Black, \textit{Wojna przeciw słabym};



\item \textit{Polska poezja baroku};



\item J. Gray, \textit{Liberalizm};



\item E. Gellner;



\item A. Golicyn, \textit{Nowe kłamstwa w~miejsce starych};



\item \textit{Monografie historii nauki \textsc{pau}};



\item J. Browne, \textit{Darwin, o~powstawaniu gatunków. Biografia};



\item R. Butterwick, \textit{Polska rewolucja a~Kościół Katolicki};



\item B. Gogol, \textit{Czerwony Sztandar. Rzecz o~sowietyzacji ziem
    Małopolski Wschodniej};



\item Timothy Gray Ash, \textit{Polska rewolucja. Solidarność};



\item A. Brzezicki, \textit{Tadeusz Mazowiecki. Biografia naszego
    premiera};



\item I. Berlin, \textit{Korzenie romantyzmu};



\item E. Lucas, \textit{Nowa zimna wojna};



\item A. R. Hall, \textit{Rewolucja naukowa 1500-1800};



\item P. Jenkis, \textit{Historia Stanów Zjednoczonych};



\item Lew Gumilow;



\item M. Janowski, \textit{Polska myśl liberalna do~1918 roku};



\item M. Blondel;



\item K. Wyka, \textit{Pan Tadeusz. Studia o~poemacie};



\item M. Zaremba, \textit{Wielka trwoga};



\item \textit{Eklektyzm, synkretyzm, uniwersa};



\item P. Fiegut, \textit{Poezja w fazie krytycznej};



\item T. Snyder, \textit{Nacjonalizm, marksizm i współczesna Europa
    Środkowa};



\item Ziemkiewicz, \textit{Polactwo};



\item M. Voelle, et al., \textit{Człowiek Oświecenia};



\item Ks. J. Tischner, \textit{Polski kształt dialogu};



\item F. Collins, \textit{Język Boga};



\item F. Braudel;



\item Ks. J. Tischner, \textit{Nieszczęsny dar wolności};



\item Ks. J. Tischner, \textit{Ksiądz na manowcach};



\item \textit{Darwin, żywot uczonego};



\item Roger Kimball, \textit{The rape of the masters: how political
    correctness sabotages art};



\item Arnold Janssen;



\item Ludwik von Mises, \textit{Socjalizm};



\item Robert Conquest, \textit{Wielki terror};



\item S. Runciman, \textit{Dzieje wypraw krzyżowych};



\item R. Graves, \textit{Mity greckie};



\item R. Graves, \textit{Mity hebrajskie};



\item L. Strauss, \textit{Prawo naturalne w świetle historii};



\item Berlinski, \textit{Szatańskie urojnie};



\item R. Wiltgen, \textit{Ren wpada do Tybru};



\item Sokal, Briemont;



\item Paul Davies;



\item Wolfgang Schivelbusch, \textit{Culture of Defeat: On National
    Trauma, Mourning, and Recovery};



\item Bockenheim K., \textit{Dworek, kontusz, karabela};



\item Paweł Śpiewak, \textit{Gramsci};



\item Polska 1989-2009: ilustrowany komentarz historyczny;



\item Spór o Polskę 1989-99: wybór tekstów prasowych;



\item J. Tazbir, \textit{Świat panów Pasków};



\item A. Sosnowska, \textit{Zrozumieć zacofanie: spory historyków
    o~Europę Wschodnią, 1947-1994};



\item \textit{Patron i dwór. Magnateria Rzeczypospolitej w XVI-XVIII
    wieku.};



\item Jon Dover, Helen W.~Kennedy, \textit{Kultura gier komputerowych};



\item Diarmaid MacCulloch, \textit{The Reformation: A History},
  alternatywny tytuł to \textit{Reformation: Europe's House Divided};



\item \textit{Skonsumowani: jak rynek psuje dzieci, infantylizuje
    dorosłych i~połyka obywateli};



\item Bartłomiej Dobroczyński, \textit{New Age};



\item Leszek Kołakowski, \textit{Główne nurty marksizmu};



\item \textit{Państwo Boże Osiemnastowiecznych Filozofów};



\item \textit{Liberty. The god that Failed};



\item Ron Jeffery, \textit{Wisła jak krew czerwona};



\item R. Browning, \textit{Cesarstwo Bizantyńskie}, \textit{Justynian
    i~Teodora};



\item H. Chadwick, \textit{Historia rozłamu Kościoła Wschodniego
    i~Zachodniego. Od~czasów apostolskich do~soboru florenckiego};



\item P. K. Hitti, \textit{Dzieje Arabów};



\item H. Kennedy, \textit{Wielkie arabskie podboje};



\item \textit{Historia Persji. Tom~I. Od~czasów najważniejszych
    do~najazdu arabów};



\item Michał Lubina, \textit{Niedźwiedź w~cieniu smoka. Rosja-Chiny
    1991--2014};



\item Bronisław Wildstein, \textit{Śmieszna dwuznaczność świata, który
    oszalał};



\item Bronisław Wildstein, \textit{Długi cień PRL-u, czyli dekomunizacja
    której nie było};



\item D. Góra-Szopiński, \textit{Zakorzenienie wolności. Myśl polityczna
    Michaela Novaka};



\item J. Grzybowski, \textit{Jacques Maritain i nowa cywilizacja
    chrześcijańska};



\item Roger Kimball, \textit{Długi marsz: jak rewolucja kulturalna z lat
    60. zmieniła Amerykę};



\item Bellantoni Patti, \textit{Jeśli to fiolet, ktoś umrze. Teoria
    koloru w~filmie};



\item A.~Wolff-Powęzka, \textit{Pamięć~-- brzemię i~uwolnienie. Niemcy
    wobec nazistowskiej przeszłości (1945--2010)};



\item Valentin L. Popov, \textit{Contact Mechanics and Friction: Physics
    Principles and Applications};



\item L. Ambrosio, N. Dancer, \textit{Calculus of Variations and Partial
    Differential Equations: Topics on Geometrical Evolution Problems
    and Degree Theory};



\item Stephen Wiggins, \textit{Global Bifurcations and Chaos: Analytical
    Methods};



\item Serbio Albeverio, \textit{Operator Methods in Ordinary and Partial
    Differential Equations};



\item D. Boccaletti, G. Pucacco, \textit{Theory of Orbits. 1: Integrable
    Systems and Non-perturbative Methods};



\item Vasil E. Tarasov, \textit{Fractional Dynamics: Applications of
    Fractional Calculus to Dynamics of Particles, Fields and Media};



\item E. C. Curtius;



\item Ch. West, \textit{Teologia ciała dla początkujących};



\item R. Hilbert, \textit{Zagłada Żydów Europejskich};



\item J. Delumeau, \textit{Cywilizacja odrodzenia};



\item P. Manent, \textit{Intelektualna historia liberalizmu};



\item B. Baczko, \textit{Filozofia francuskiego oświecenia};



\item J. Juszczak, \textit{Ordoliberalizm};



\item Albaro Vargas Llosa, \textit{Mit Che a przyszłość wolności};



\item T. Snyder, \textit{Rekonstrukcja narodów};



\item Red. M. Rechowicz, \textit{Dzieje teologii katolickiej w Polsce};



\item Ricceure, \textit{Symbolika zła};



\item J. Śniadecki;



\item S. McMeekin, \textit{Największa grabież w historii. Jak bolszewicy
    złupili Rosję};



\item R. Syme, \textit{Rewolucja rzymska};



\item E. von Kuchnelt-Leddhin, \textit{Ślepy tor};



\item A. McGrath, \textit{Jan Kalwin. Studium kształtowania się kultury
    Zachodu};



\item P. Kuncewicz, \textit{Samotni wobec historii};



\item C. Ginzburg, \textit{Ser i~robak};



\item James Conrayd Martin, \textit{Nie ponaglaj rzeki};



\item W. Zajewski, \textit{Czy historycy piszą prawdę};



\item S. Węgrzynowicz, \textit{Patrioci i zdrajcy};



\item F. Musiał, \textit{Raj grabarzy narodu};



\item Red. Marek Kornat, \textit{Pius XII~-- papież w~epoce
    totalitaryzmów};



\item H. Głębocki, \textit{„Diabeł Asmodeusz” w~niebieskich binoklach
    i~kraj przyszłości. Henryk Gurowski i~Rosja};



\item R. Fiegut, \textit{Zaproszenie do „Quidama”};



\item M. Goliczak, \textit{Związek Radziecki w~myśli politycznej
    polskiej opozycji 1976-1989};



\item M. Urbankowski, \textit{Romans z~Polską};



\item E. J\"{u}nger, \textit{Węzeł gordyjski. Eseistyka lat
    pięćdziesiątych};



\item J. Besal, \textit{Stanisław Żółkiewski};



\item J. Skowronek, \textit{Adam Jerzy Czartoryski, 1770-1861};



\item C. Shindler, \textit{Historia współczesnego Izraela};



\item W. Bernacki, \textit{Myśl polityczna I Rzeczpospolitej};



\item \textit{Polsko, uwierz w~swoją siłę};



\item Encyclopedia of Mathematical Sciences;



\item \textit{Open GL. Księga eksperta};



\item A. Nowak, \textit{Putin. Źródła imperialnej agresji};



\item Red. P. Musiewicz, \textit{Ronald Reagan. Nowa odsłona w 100-lecie
    urodzin};



\item H. Pilus, \textit{Własność i zasady w katolickiej myśli
    społecznej};



\item N. von Below, \textit{Byłem adiunktem Hitlera};



\item G. Kucharczyk, \textit{Czerwone karty Kościoła};



\item G. Kucharczyk, \textit{Kielnią i cyrklem. Laicyzacja Francji w
    latach 1870-1914};



\item F. Koneczny, \textit{Dzieje Polski opowiedziane dla młodzieży};



\item M. Ekstein, \textit{Święto Wiosny. Wielka wojna i narodziny nowego
    wieku};



\item B. Kiereś, \textit{Tylko rodzina!};



\item M. Soska, \textit{Za Świętą Ruś. Współczesny nacjonalizm Rosyjski
    -- zarys ideologi};



\item H. Pająk, \textit{Rytualna zemsta na~„kolebce” Solidarności
    1981-2011};



\item M. Skousen, \textit{Narodziny współczesnej ekonomii};



\item P. Gontarczyk, \textit{Najnowsze kłopoty z~historią};



\item G. Bardy, \textit{Charles de~Gaulle. Biografia katolika i~męża
    stanu};



\item J. Garrison, \textit{Ameryka jako imperium. Przywódcy świata czy
    bandycka potęga};



\item E. Lucas, \textit{Operacja Snowden};



\item C. S. Lewis, \textit{Ostatnia noc świata};



\item N. Janner SJ, \textit{Krótka historia Kościoła Katolickiego. Nowe
    spojrzenie};



\item \textit{Literahistorica};



\item \textit{Bóg Zła};



\item J. Wieliczka-Szarkowa, \textit{III Rzesza. Zbrodnia bez kary};



\item P. Gontarczyk, \textit{Polska Partia Robotnicza. Droga do władzy
    1941-1944};



\item P. Moa, \textit{Mity wojny domowej w Hiszpania 1936--1939};



\item Władymir Arnold, \textit{Lectures on Partial Differential
    Equations};



\item Herman H. Goldstein, \textit{A History of Numerical Analysis. From
    the 16th through the 19th century};



\item G. Edward Griffin, \textit{Finansowy potwór z Jekyll Island};



\item L. Ulicka, \textit{Daniel Stein, tłumacz};



\item R. Scruton, \textit{Kultura jest ważna};



\item P. Gottfried, \textit{Wojna i demokracja};



\item S. Didler, \textit{Rola neofitów w dziejach Polskich};



\item A. Wielomski, \textit{Konserwatyzm. Główne idee i postaci};



\item C. S. Lewis, \textit{Bóg na ławie oskarżonych};



\item R. Spałek, \textit{Komuniści przeciw komunistom};



\item Doug Stanton, \textit{Dwunastu odważnych. Odtajniona historia
    konnych żołnierzy};



\item Wicek Warszawiak, \textit{Humor w~czasie okupacji};



\item Lawrence Wright, \textit{Wyniosłe wieże. Al-Kaida i~atak
    na~Amerykę};



\item Robert Mason, \textit{Powiedz, że się boisz};



\item Dla taty: Chufo Llorens;



\item Jearl Walker, \textit{Latający cyrk fizyki};



\item David J.~Griffits, \textit{Introduction to~Quantum Mechanics};



\item B. Dembowski, \textit{O filozofii chrześcijańskiej w Ameryce
    Północnej};



\item Wiesław Caban, \textit{Powstanie styczniowe. Polacy i Rosjanie w
    XIX wieku};



\item Jacek Wegner, \textit{Biesy sarmackie};



\item Andrzej Józef Kamiński, \textit{Koszmar niewolnictwa. Obozy
    koncentracyjne od 1896 do dziś. Analiza};



\item Jochen B\"{o}hler, \textit{Wojna domowa. Nowe spojrzenie
    na~odrodzenie Polski};



\item F. Wesołowski, \textit{Zasady muzyki};



\item Red. A. Czarniecka-Stefańska, \textit{Szukając prawdy. Edyta Stein
    w~kulturze polskiej};



\item E. Stein, \textit{Kobieta. Jej zadanie według natury i~łaski};



\item Red. Umberto Eco, \textit{Historia piękna};



\item Zdzisław Krasnodębski, \textit{Rozumienie ludzkiego zachowania.
    Rozważania o~filozoficznych podstawach nauk humanistycznych
    i~społecznych};



\item Adam Przechrzta, \textit{Chorągiew Michała Archanioła};



\item Anna Sobolewska, \textit{Mapy duchowe współczesności: co~nam
    zostało z~Nowej Ery?};



\item Blake J. Harris, \textit{Wojny konsolowe};



\item Nikołaj Zieńkowicz, \textit{Tajemnice mijającego wieku. Władza
    zakulisowe działania zatargi};



\item Nikołaj Zieńkowicz, \textit{Od~Lenina do~Jelcyna. Kremlowska
    księga zamachów};



\item Marek Jan Chodakiewicz, \textit{Transformacja czy~niepodległość?};



\item Helmuth Plessner, \textit{Śmiech i~płacz. Badania nad~granicami
    ludzkiego zachowania};



\item Samuel M.~Katz, \textit{Aman. Wywiad wojskowy Izraela};



\item Praca zbiorowa, \textit{Rosja -- Chiny. Dwa modele transformacji};



\item Mariola Marczak, \textit{Poetyka filmu religijnego};



\item Lech Bukowski, \textit{Sade, Kafka, Kierkegaard. Między rozkoszą a
    opresją};



\item Philip Earl Steele, \textit{Nawrócenie i chrzest Mieszka I};



\item Henryk Samsonowicz, \textit{My o sobie. Portret własny mieszkańców
    ziem polskich u schyłku średniowiecza};



\item \textit{Piłsudski (nie)znany. Historia i popkultura};



\item Jakub Z.~Lichański, \textit{Niepopularnie o popularnej. O
    narzędziach badań literatury};



\item Tadeusz Manteuffel, \textit{Historia Powszechna. Średniowiecze};



\item Robert Jung, \textit{Jaśniej niż tysiąc słońc. Losy badaczy
    atomu};



\item Wiesław Bator, \textit{Religia starożytnego Egiptu. Perspektywa
    religioznawcza};



\item Gabriela Matuszek, \textit{Maski i demony wczesnego modernizmu};



\item Artur Szarecki, \textit{Kapitalizm somatyczny. Ciało i władza w
    kulturze korporacyjnej};



\item \textit{Francuskie pisma o dramacie (1537-1631)};



\item John A. McClure, \textit{Półwiary};



\item Jerzy Axer, Tadeusz Bujnicki, \textit{Wokół „W pustyni i w
    puszczy”. W stulecie pierwodruku powieści};



\item Gerd-Klaus Kaltenbrunner;



\item Steve Brusatte, \textit{Era dinozaurów - od narodzin do upadku.
    Nowe odkrycia i fakty o zaginionym świecie};



\item Robert Fabbri, \textit{Wespazjan, trybun Rzymu};



\item Michael Billing, \textit{Banalny nacjonalizm};



\item Ernest Gellner, \textit{Narody i~nacjonalizm};



\item \textit{Oświecenie, nieoświecone. Człowiek, natura, magia};



\item Eric Hobsbawm, Terence Ranger, \textit{Tradycja wynaleziona};



\item Aleksander Śpiewakowski, \textit{Samuraje};



\item N. Davies, \textit{Serce Europy};



\item Thomas Hylland Eriksen, \textit{Etniczność i~nacjonalizm};



\item Benedict Andreson, \textit{Wspólnoty wyobrażone};



\item Karol Tarnowski, \textit{W~mroku uczonej niewiedzy};



\item Barbary Tuchman, \textit{Odległe zwierciadło, czyli rozlicznymi
    plagami nękane XIV stulecie};



\item Homi K. Bhabha, \textit{Miejsca kultury};



\item Tim Edensor, \textit{Tożsamość narodowa, kultura popularna i~życie
    codzienne};



\item Justyna Balisz-Schmelz, \textit{Przeszłość niepokonana. Sztuka
    niemiecka po 1945 roku jako przestrzeń i medium pamięci};



\item Anthony D.~Smith, \textit{Etniczne źródła narodów};



\item Anthony D.~Smith, \textit{Kulturowe podstawy narodów};



\item Piotr Eberhardt, \textit{Rozwój światowej myśli geopolitycznej};



\item P. Bąk, \textit{Gramatyka języka polskiego. Zarys popularny};



\item Jacek Wegner, \textit{Rzeczpospolita. Duma i~wstyd};



\item Wassily Kandinsky, \textit{Punkt i~linia a~płaszczyzna. Przyczynek
    do~analizy elementów malarskich};



\item Red. Aneta Pawłowska, Julia Sowińska-Heim, \textit{Afryka
    i~(post)kolonializm};



\item Robert J. C. Young, \textit{Postkolonializm. Wprowadzenie};



\item G.~Michaelson, \textit{An Introduction to~Functional Programming
    through Lambda Calculus};



\item Gilberto Freyre, \textit{Panowie i niewolnicy};



\item H. P. Barendregt, \textit{The Lambda Calculus: Its Syntax and
    Semantics};



\item Leon Degrelle, \textit{Wiek Hitlera};



\item N. D. Jones, \textit{Computability and Complexity: From
    a~Programming Perspective};



\item Ks. Marcin Worbs, \textit{Człowiek w~misterium liturgii};



\item Anna Grześkowiak-Krwawicz, \textit{Dyskurs polityczny
    Rzeczypospolitej Obojga Narodów};



\item Nowak \textit{Dzieje Polski};



\item Margaret Atwood, \textit{Dług. Rozrachunek z~ciemną stroną
    bogactwa};



\item Jared Diamond, \textit{Strzelby, zarazki, maszyny. Losy ludzkich
    społeczeństw};



\item Edward E. Evans-Pritchard, \textit{Czary, wyrocznie i~magia
    u~Azande};



\item W. Szumowski, \textit{Historia medycyny filozoficznie ujęta};



\item Michał Dondzik, Krzysztof Jajko, Emil Swoiński, \textit{Elementarz
    Wytwórni Filmów Oświatowych};



\item James M. Murray, \textit{Brugia: Kolebka kapitalizmu};



\item Maciej Janowski, \textit{Narodziny inteligencji: 1750--1831};



\item Jerzy Jedlicki, \textit{Jakiej cywilizacji Polacy potrzebują:
    studia z dziejów idei i~wyobraźni XIX wieku};



\item Jerzy Jedlicki, \textit{Droga do narodowej klęski};



\item Jerzy Jedlicki, \textit{Błędne koło: 1832-1864};



\item Jerzy Jedlicki, \textit{Nieudana próba kapitalistycznej
    industrializacji: analiza państwowego gospodarstwa przemysłowego
    w~Królestwie Polskim XIX w.};



\item Jerzy Jedlicki, \textit{Klejnot i~bariery społeczne: przeobrażenia
    szlachectwa polskiego w~schyłkowym okresie feudalizmu};



\item Jerzy Jedlicki, \textit{Świat zwyrodniały: lęki i~wyroki krytyków
    nowoczesności};



\item Rob Riemen, \textit{Wieczny powrót faszyzmu};



\item Łukasz A. Plesnar, \textit{Twarze Westernu};



\item S. Prat, \textit{Język C++. Szkoła programowania};



\item Oskar Halecki, \textit{Tysiąc lat polski katolickiej};



\item Krzysztof Mazur, \textit{Przekroczyć nowoczesność. Projekt
    polityczny ruchu społecznego Solidarność};



\item Marian Henryk Serejski, \textit{Europa a rozbiory Polski: studium
    historiograficzne};



\item Jerzy Łanowski, \textit{Antologia anegdoty antycznej: teraz trzeci
    raz wydane historyjki budujące i niebudujące z autorów greckich i
    rzymskich};



\item Artur Domosłowski, \textit{Kapuściński non-fiction};



\item Michael Moran, \textit{Kraj z Księżyca: podróże do serca Polski};



\item Pierre Hadot, \textit{Filozofia jako ćwiczenie duchowe};



\item Robert D. Richtmayer, \textit{Principles of advanced mathematical
    physics};



\item Walter Burkert, \textit{Stwarzanie świętości. Ślady biologii
    we~wczesnych wierzeniach religijnych};



\item A. I. Anselm, \textit{Podstawy fizyki statystycznej
    i~termodynamiki};



\item Z. Krasnodębski, \textit{Demokracja peryferii};



\item Maria Dzielska, \textit{Hypatia z~Aleksandrii};



\item Paweł Śpiewak, \textit{Spór o~Polskę, 1989--99};



\item Tadeusz Zieliński, \textit{Religia starożytnej Grecji};



\item Ernest Gellner, \textit{Postmodernizm, rozum i~religia};



\item Daniel Beauvois, \textit{Polacy na Ukrainie 1831-1863. Szlachta
    polska na Wołyniu, Podolu i Kijowszczyźnie};



\item \textit{Polskie mity polityczne XIX i XX wieku};



\item \textit{O nas bez nas. Historia Polski w historiografiach
    obcojęzycznych};



\item Leibniz, \textit{Wyznanie wiary filozofa, Rozprawa metafizyczna;
    Monadologia; Zasady natury i łaski oraz inne pisma filozoficzne};



\item Hanna Świda-Ziemba, \textit{Człowiek wewnętrznie zniewolony.
    Mechanizmy i konsekwencje minionej formacji --~analiza
    psychologiczna};



\item Braudel Fernand, \textit{Dynamika kapitalizmu};



\item Bod Rens, \textit{Historia humanistyki};



\item Lord Acton, \textit{Historia wolności: wybór esejów};



\item Andrzej Żbikowski, \textit{Żydzi};



\item Stefan Bartkowski, \textit{Pod wspólnym niebem: krótka historia
    Żydów w~Polsce i~stosunków polsko-żydowskich};



\item Arystoteles, \textit{Retoryka};



\item Agnieszka Urbańczyk, Diamentowy Grant, \textit{Polityczność
    science fiction w recepcji fanowskie};



\item \textit{Lech Wzbudzony};



\item \textit{Unintended Reformation};



\item J. Polit, \textit{Chiny};



\item Paweł Śpiewak, \textit{Teologia i~filozofia żydowska wobec
    Holocaustu};



\item J.K. Fairbank, \textit{Historia Chin. Nowe spojrzenie};



\item \textit{Nowożytna historia Chin}, red. R. Sławiński;



\item R. Sławiński, \textit{Geneza Chińskiej Republiki Ludowej};



\item K. Seitz, \textit{Chiny. Powrót Olbrzyma};



\item A. Bolesta, \textit{Chiny w okresie transformacji};



\item \textit{Chiny. Przemiany państwa i społeczeństwa w okresie reform
    1978--2000}, red. K. Tomala;



\item Andrzej Napiórkowski OSPPE, \textit{Teologie XX i~XXI wieku};



\item Alfred V.~Aho, Jeffrey D.~Ullman, \textit{Wykłady z~informatyki
    z~przykładami w~języku~C};



\item Jerzy Eisler, \textit{Co nam zostało z tamtych lat. Dziedzictwo
    PRL};



\item Peter Burke;



\item B. Kozera, \textit{Literatura a~religia. Polska współczesna
    powieść katolicka};



\item \textit{Physics of living systems};



\item Jadwiga Staniszkis, \textit{Samoograniczająca~się rewolucja};



\item Jadwiga Staniszkis, \textit{Postkomunizm. Próba opisu};



\item Z. Wójcik, \textit{Dzikie Pola w~ogniu. O~Kozaczyźnie w~dawnej
    Rzeczypospolitej};



\item Jan Kofman, Wojciech Roszkowski, \textit{Transformacja
    i~postkomunizm};



\item M. Gołaszewska, \textit{Estetyka współczesna};



\item E. Badinter, \textit{XY~-- tożsamość mężczyzny};



\item R. Bly, \textit{Żelazny Jan. Rzecz o~mężczyznach};



\item Z. Wójcik, \textit{Wojny kozackie w dawnej Polsce};



\item Z. Wójcik, \textit{Dzieje Rosji: 1533-1801};



\item Vigarello Georges, \textit{Historia gwałtu};



\item Tannahill Reay, \textit{Historia kuchni};



\item Éliphas Lévi, \textit{Historia magii};



\item Meyer Michel (red.), \textit{Historia retoryki od Greków do dziś};



\item Simmel Georg, \textit{Filozofia pieniądza};



\item Dahl Robert, \textit{Demokracja i~jej krytycy};



\item Z. Wojcik, \textit{Jan Sobieski: 1629-1696};



\item Z. Wójcik, \textit{Jan III Sobieski};



\item Ariès Philippe, \textit{Historia dzieciństwa. Dziecko i rodzina w
    czasach ancien régime’u};



\item Bataille Georges, \textit{Historia erotyzmu};



\item Vigarello Georges, \textit{Historia otyłości};



\item Flandrin Jean-Louis, \textit{Historia rodziny};



\item Z. Wójcik, \textit{Jan Kazimierz Waza};



\item Z. Wójcik, \textit{Józef Piłsudski 1867-1935};



\item \textit{Legendy uświęcone. Twórczość J. R. R. Tolkiena a
    chrześcijaństwo};



\item Stefan Bratkowski, \textit{Nieco inna historia cywilizacji: dzieje
    banków, bankierów i obrotu pieniężnego};



\item Izrael Szahad, \textit{Żydowskie dzieje i religia; Żydzi i goje –
    XXX wieków historii};



\item Izrael Szahad, \textit{Tel Awiw za zamkniętymi drzwiami};



\item Andrzej Żbikowski, \textit{Ideologia antysemicka w~Polsce
    1848-1918};



\item Władysław Bruliński, \textit{Antykościół};



\item Władysław Bruliński, \textit{Co to jest marksizm?};



\item Władysław Bruliński, \textit{Czerwone palmy historii};



\item Władysław Bruliński, \textit{Dokąd idziesz Polsko?};



\item \textit{Żydzi i judaizm we współczesnych badaniach polskich};



\item Artur Eisenbach, \textit{Emancypacja Żydów na ziemiach polskich
    1785-1870 na tle europejskim};



\item Artur Eisenbach, \textit{Z dziejów ludności żydowskiej w Polsce w
    XVIII i XIX w.};



\item Artur Eisenbach, \textit{Kwestia równouprawnienia Żydów w
    Królestwie Polskim};



\item Marian Fuks, \textit{Humor Żydów polskich (do 1939 r.)};



\item Marian Fuks, \textit{Żydzi w Polsce – Dawniej i dziś};



\item Marian Fuks, \textit{Prasa żydowska w Warszawie 1823-1939};



\item Marian Fuks, \textit{Z dziejów wielkiej katastrofy narodu
    żydowskiego};



\item August Grabski, \textit{Studia z dziejów i kultury Żydów w Polsce
    po 1945 r.};



\item Michael Hesemann, \textit{Kłamstwa Hitlera};



\item David Hockney, \textit{Wiedza tajemna. Sekrety technik malarskich
    Dawnych Mistrzów};



\item Anna Magdalena Mandrela, \textit{Tomizm Garrigou-Lagrange’a wobec
    wizji filozoficznej Teilharda de Chardin};



\item Leonie Swann, \textit{Powiększ Sprawiedliwość owiec. Filozoficzna
    powieść kryminalna};



\item Antoine de Saint-Exupéry, \textit{Twierdza};



\item Umberto Eco, \textit{Historia brzydoty};



\item Brian Reynolds Myers, \textit{Najczystsza rasa: Propaganda Korei
    Północnej};



\item Tomasz Strzyżewski, \textit{Wielka księga cenzury PRL w
    dokumentach};



\item Orlando Figes, \textit{Tragedia narodu. Rewolucja rosyjska
    1891-1924};



\item Jon Savage, \textit{Teenage: The Creation of Youth Culture};



\item Lawrence Weschler, \textit{Mr. Wilson's Cabinet of Wonder: Pronged
    Ants, Horned Humans, Mice on Toast, and Other Marvels of Jurassic
    Technology};



\item Christophe Galfard, \textit{Wszechświat w twojej dłoni};



\item Swietłana Aleksijewicz, \textit{Cynkowi chłopcy};



\item Aleksander Hertz, \textit{Amerykańskie stronnictwa polityczne};



\item Aleksander Hertz, \textit{Żydzi w kulturze polskiej};



\item Aleksander Hertz, \textit{Wyznania starego człowieka};



\item Maurycy Horn, \textit{Żydowskie bractwa rzemieślnicze na ziemiach
    polskich, litewskich, białoruskich i ukraińskich w latach
    1613-1850};



\item Maurycy Horn, \textit{Walka chłopów czerwonoruskich z wyzyskiem
    feudalnym w latach 1600-1643};



\item Adam Kaźmierczyk, \textit{Sejmy i sejmiki szlacheckie wobec Żydów
    w II połowie XVII wieku};



\item Adam Kaźmierczyk, \textit{Żydzi w dobrach prywatnych. W świetle
    sądowniczej i administracyjnej praktyki dóbr magnackich w wiekach
    XVI-XVIII};



\item Krystyna Kersten, \textit{Polacy-Żydzi-Komunizm. Anatomia półprawd
    1939-1968};



\item Krystyna Kersten, \textit{Pogrom Żydów w Kielcach 4 lipca 1946
    r.};



\item Andrzej Sulima Kamiński, \textit{Historia Rzeczypospolitej Wielu
    Narodów 1505-1795. Obywatele, ich państwa, społeczeństwo,
    kultura};



\item Cynarski S., \textit{Zygmunt August};



\item Cyra A., \textit{Rotmistrz Pilecki. Ochotnik do Auschwitz};



\item \textit{Czy ktoś przebije ten mur? Sprawa Pyjasa};



\item Dudek A., Zblewski Z., \textit{Utopia nad Wisłą. Historia
    Peerelu};



\item Czapliński W., \textit{Władysław IV i jego czasy};



\item Dybiec J., \textit{Nie tylko szablą. Nauka i kultura polska w
    walce o~utrzymanie tożsamości narodowej 1795--1918};



\item Eisler J., \textit{Zarys dziejów politycznych Polski 1944--1989};



\item Grzybowski S., \textit{Henryk Walezy};



\item Ignatowicz I., \textit{Społeczeństwo polskie 1864--1914};



\item \textit{Inteligencja polska XIX i XX wieku. Studia}, red. R.
  Czapulis-Rastenis, t.1-6;



\item Jedynak B., \textit{Obyczaje domu polskiego w~czasach niewoli
    1795--1918};



\item Kaczmarczyk J., \textit{Bohdan Chmielnicki};



\item Kawalec K., \textit{Roman Dmowski};



\item \textit{Kobieta i kultura życia codziennego. Wiek XIX i XX. Zbiór
    studiów}, red. A. Żarnowska, A. Szwarc;



\item \textit{Kobieta i społeczeństwo na ziemiach polskich w XIX wieku,
    zbiór studiów}, red. A. Żarnowska, A. Szwarc;



\item Konopczyński W., \textit{Dzieje Polski nowożytnej};



\item Kowecka E., \textit{W salonie i w kuchni. Opowieść o kulturze
    materialnej pałaców i dworów polskich w XIX wieku};



\item Krawczak T., \textit{W szlacheckim zaścianku};



\item Kuchowicz Z., \textit{Miłość staropolska};



\item Kuchowicz Z., \textit{Obyczaje staropolskie XVII-XVIII w.};



\item Litak S., \textit{Od reformacji do Oświecenia. Kościół katolicki
    w~Polsce nowożytnej};



\item Mączak A., \textit{Klientela. Nieformalne systemy władzy w Polsce
    Europie XVI-XVIII w.};



\item Łuczak C., \textit{Polska i Polacy w drugiej wojnie światowej};



\item Molenda J., \textit{Chłopi. Naród. Niepodległość. Kształtowanie
    się postaw narodowych i~obywatelskich chłopów w~Galicji
    i~Królestwie polskim w~przededniu odrodzenia Polski};



\item Możdżyńska-Nawotka M., \textit{O~modach i~strojach};



\item \textit{Obyczaje w Polsce. Od średniowiecza do czasów
    współczesnych};



\item Olszewski D., \textit{Polska kultura religijna na przełomie XIX
    i~XX wieku};



\item Olszewski H., \textit{O skutecznym rad sposobie};



\item \textit{Polska XVII wieku. Państwo, społeczeństwo, kultura}, red.
  J. Tazbir;



\item Paczkowski A., \textit{Pół wieku dziejów Polski, 1939--1989};



\item \textit{Polska na przestrzeni wieków}, red. J. Tazbir;



\item Przyboś A., \textit{Michał Korybut Wiśniowiecki 1640-1673};



\item Rok B., \textit{Człowiek wobec śmierci w kulturze staropolskiej};



\item \textit{Rzeczpospolita wielu narodów i jej tradycje}, red. M.
  Markiewicz, A. Link-Lenczewski;



\item \textit{Społeczeństwo polskie od X do XX wieku}, red. I.
  Ignatowicz, A. Mączak, B. Zientara, J. Żarnowski;



\item Staszewski J., \textit{August II Mocny};



\item Staszewski J., \textit{August III Sas};



\item Suleja W., \textit{Józef Piłsudski};



\item Szubarczyk P., \textit{Inka. Zachowałam się jak trzeba\ldots};



\item Topolski J., \textit{Polska w czasach nowożytnych. Od europejskiej
    potęgi do utraty niepodległości};



\item Terlecki R., \textit{Miecz i tarcza komunizmu. Historia aparatu
    bezpieczeństwa 1944--1990};



\item \textit{Tradycje polityczne dawnej Polski}, red. A.
  Sucheni-Grabowska, A. Dybowska;



\item Wandycz P., \textit{Pod zaborami. Ziemie Rzeczypospolitej w latach
    1795--1918};



\item Wisner H., \textit{Władysław IV Waza};



\item Wisner H., \textit{Zygmunt III Waza};



\item Zielińska Z., \textit{Ostatnie lata Pierwszej Rzeczypospolitej};



\item Zblewski Z, \textit{Abecadło Peerelu};



\item Zienkowska K., \textit{Stanisław August Poniatowski};



\item Zdrada J., \textit{Historia Polski 1795--1914};



\item Żarnowski, J., \textit{Polska 1918--1939. Praca, technika,
    społeczeństwo};



\item Ziejka F., \textit{Złota legenda chłopów polskich};



\item Żołądź D., \textit{Ideały edukacyjne doby staropolskiej. Stanowe
    modele i potrzeby edukacyjne szesnastego i siedemnastego wieku};



\item R. Wapiński, \textit{Historia polskiej myśli politycznej XIX
    i~XX~wieku};



\item R.R. Ludwikowski, \textit{Historia polskiej myśli politycznej};



\item W. Bernacki, \textit{Liberalizm polski};



\item B. Szlachta, \textit{Z dziejów polskiego konserwatyzmu};



\item M. Śliwa, \textit{Polska myśl socjalistyczna 1892--1948};



\item Z. Ogonowski, \textit{Filozofia polityczna w~Polsce XVII w.
    i~tradycje demokracji europejskiej};



\item S. Tarnowski, \textit{Pisarze polityczni XVI wieku};



\item S. Tarnowski, \textit{Historia literatury polskiej, t.2};



\item W. Konopczyński, \textit{Polscy pisarze polityczni XVIII w.};



\item H. Olszewski, \textit{Doktryny prawno-ustrojowe czasów saskich};



\item K. Waliszewski, \textit{Potoccy i Czartoryscy, walka stronnictw i
    programów politycznych przed upadkiem Rzeczypospolitej 1734--1763};



\item Red. Tomasz Dołęgowski, \textit{Przewodnik po moralnym
    kapitalizmie};



\item Besala J., \textit{Stefan Batory};



\item Cieślak E., \textit{Stanisław Leszczyński};



\item Bogucka M., \textit{Staropolskie obyczaje XVI-XVII w};



\item Cz. Michalski, \textit{Western};



\item A. Chwalba, \textit{III Rzeczpospolita~-- raport specjalny};



\item Jean-Paul Bled, \textit{Bismarck. Żelazny kanclerz};



\item Marcin Król, \textit{Byliśmy głupi};



\item J. Wójcik, \textit{Labirynt światła};



\item A. Chwalba, \textit{Historia Polski 1795-1918};



\item Brzoza C., Sowa A., \textit{Historia Polski 1918-1945};



\item Cz. Michalski, \textit{Western i~jego bohaterowie};



\item Rafał Marszałek, \textit{Pamflet na kino codzienne};



\item Rafał Marszałek, \textit{Polska wojna w obcym filmie};



\item Rafał Marszałek, \textit{Filmowa pop-historia};



\item Rafał Marszałek, \textit{Kino rzeczy znalezionych};



\item J. Skwara, \textit{Western odrzuca legendę};



\item Władysław Konopczyński, \textit{Konfederacja barska};



\item Michał Łuczewski, \textit{Odwieczny naród. Polak i~katolik
    w~Żmiącej};



\item Alan Bullock, \textit{Hitler. Studium tyrani};



\item Tadeusz Lubelski, \textit{Historia Kina Polskiego, Twórcy, Filmy,
    Konteksty};



\item Marek Haltof, \textit{Kino polskie};



\item \textit{Kino bez tajemnic};



\item David Bordwell, Kristin Thompson, \textit{Film Art. Sztuka
    filmowa. Wprowadzenie};



\item W. Stróżewski, \textit{Estetyka};



\item Vigarello Georges, \textit{Historia czystości i brudu};



\item Muchembled Robert, \textit{Orgazm i Zachód};



\item Sloterdijk Peter, \textit{Pogarda mas};



\item Gately Iain, \textit{Kulturowa historia alkoholu};



\item Eco Umberto, \textit{Poszukiwanie języka doskonałego w kulturze
    europejskiej};



\item Coogan Michael, \textit{Bóg i~seks. Co naprawdę mówi Biblia};



\item Secher Reynald, \textit{Ludobójstwo francusko-francuskie};



\item Vigarello Georges, \textit{Historia urody};



\item Higman B.W., \textit{Historia żywności};



\item Tannahill Reay, \textit{Historia seksu};



\item Ramamurti Rajaraman, \textit{Solitons and~instantons};



\item Wilson Edward O., \textit{Znaczenie ludzkiego istnienia};



\item Karl Loewith, \textit{Historia powszechna i dzieje zbawienia};



\item Tony Judt, \textit{Historia niedokończona. Francuscy
    intelektualiści 1944-1956};



\item Karl Loewith, \textit{Od Hegla do Nietzschego. Rewolucyjny przełom
    w myśli XIX wieku};



\item J. Fiske \textit{Zrozumieć kulturę popularną};



\item Anatol Taras, \textit{Anatomia nienawiści};



\item R. Rodes, \textit{Jak powstała bomba atomowa?};



\item A. Tarski;



\item Jacek Trznadel, \textit{Z~popiołów czy wstaniesz?};



\item Jacek Trznadel, \textit{Spór o~całość: Polska 1939-2004};



\item Karol Buczek, \textit{Studia z dziejów ustroju
    społeczno-gospodarczego Polski piastowskiej};



\item Red. S. Kowalczyk, E. Balawajder, \textit{Jacques Maritain,
    prekursor soborowego humanizmu};



\item T. M. Jaroszewski, \textit{Osobowość i wspólnota. Problemy
    osobowości we współczesnej antropologii filozoficznej --~marksizm,
    strukturalizm, egzystencjalizm, personalizm chrześcijański};



\item Roman Graczyk, \textit{Od uwikłania do autentyczności. Biografia
    polityczna Tadeusza Mazowieckiego};



\item S. Wiggins, \textit{Introduction to Applied Nonlinear Dynamical
    Systems and Chaos};



\item Władymir Arnold, \textit{Catastrophe Theory};



\item Red. Christian B\"{a}r, Klaus Fredenhagen, \textit{Quantum Field
    Theory on Curved Spacetimes};



\item Sholmo Sternberg, \textit{Semi-Riemann Geometry and General
    Relativity};



\item Peter B. Gilkey, \textit{Invariance theory, the heat equation, and
    the Atiyah-Singer Index Theorem};



\item Richard S. Palais, \textit{A Modern Course on Curves and
    Surfaces};



\item J\"{u}rgen Jost, \textit{Geometry and physics};



\item Charles Freeman, \textit{A New History of Early Christianity};



\item Bart D. Ehrman, \textit{Whose Word is it. The Story Behind who
    changed the New Testament and why};



\item Robert V. Huber, Stephen M. Miller \textit{Historia Biblii};



\item John Galindo, Owen F. Cummings, \textit{Duchowość, intymność i
    seksualność};



\item James L. Papandrea, \textit{Depozyt wiary};



\item Anadijiban Das, Andrew DeBenedictis, \textit{The general theory of
    relativity. A mathematical exposition};



\item Sergio A. Albeverio, Raphael J. H\o egh-Krohn, Sonia Mazzucchi,
  \textit{Mathematical theory of Feynman path integrals};



\item Leah Darrow, \textit{Inna strona piękna};



\item Michał Paradowski, \textit{Trzydziestolecie Drugiego Soboru
    Watykańskiego};



\item C. Radhakrishna Rao, \textit{Statystyka i~prawda};



\item Donald Ritchie, \textit{The Films of Akira Kurosawa};



\item Martin Konings, \textit{The Emotional Logic of Capitalism. What
    Progressives Have Missed};



\item David Sloan Wilson, \textit{Darwin's Cathedral: Evolution,
    Religion, and the~Nature~of Society};



\item Ch.~R.~Browning, \textit{Zwykli ludzie. 101~Policyjny Batalion
    Rezerwowy i~„ostateczne rozwiązanie” w~Polsce}



\item Viviana A. Zelizer, \textit{The Social Meaning of Money: Pin
    Money, Paychecks, Poor Relief, and Other Currencies};



\item Randy Shilts, \textit{And the Band Played On};



\item Rana Mitter, \textit{Gorzka rewolucja};



\item Rana Mitter, \textit{Chiny nowoczesne};



\item Peter Seewald, \textit{Benedykt XVI. Portret z~bliska};



\item C. Radhakrishna Rao, \textit{Modele liniowe statystyki matematycznej};



\item Marta Przybyła, \textit{I dam wam serce nowe};



\item Alfred Tarski, \textit{Wprowadzenie do logiki};



\item I. Wallerstein, \textit{Europejski uniwersalizm. Retoryka władzy};



\item Abbé Jacques Meinvielle, \textit{De Lamennais ŕ Maritain};



\item Zofia Szmydt, \textit{Transformacja Fouriera i~równania różniczkowe
    liniowe};



\item Ralph Martin, \textit{Kościół w kryzysie. Ścieżki wyjścia};



\item Stefan Wyszyński, \textit{Przestrogi dla Polaków};



\item Tomasz Terlikowski, \textit{Czego księża nie powiedzą Ci
    o~antykoncepcji?};



\item V. Messori, \textit{Kościół Katolicki i~jego wrogowie};



\item Stefan Ziemba, \textit{Analiza drgań}, dwa tomy;



\item Romano Guardini, \textit{Wolność-łaska-los};



\item Romano Guardini, \textit{O~istocie chrześcijaństwa};



\item Gianfranco Ravasi, \textit{Kohelet};



\item Carrie Gress, \textit{Odnowa};



\item Jan Krempa, Barbara Mażbic-Kulma, \textit{Elementy logiki, teorii
    mnogości i~algebry};



\item Charles Moore, \textit{Margaret Thatcher};



\item G. Polya, \textit{Jak to rozwiązać?};



\item Donald J. Trump, Meredith McIver, \textit{Nigdy się nie poddawaj!};



\item Patti Bellantoni, \textit{Jeśli to fiolet, ktoś umrze. Teoria koloru
    w~filmie};



\item Józef Pawłowski, \textit{Przeszłość w~ideologii Komunistycznej Partii
    Chin};



\item Marjorie Bowen?;



\item Heinrich Lausberg, \textit{Retoryka literacka};



























\end{enumerate}
% ##################










% ######################################
\newpage
\section{Zaczęte i~nieskończone}

\vspace{\spaceTwo}
% ######################################














% ######################################
\newpage
\section{Articles}

\vspace{\spaceTwo}
% ######################################



% ##################
\begin{enumerate}

\item Edward Witten, \textit{Notes on Some Entanglement Properties of
    Quantum Field Theory},
  \href{https://arxiv.org/abs/1803.04993}{arXiv:1803.04993};



\item Jeff Bezanson et al, "Julia: dynamism and performance reconciled
  by design"
  \href{https://doi.org/10.1145/3276490}{https://doi.org/10.1145/3276490};



\item Francesco Zappa Nardelli et al., "Julia subtyping: a rational
  reconstruction",
  \href{https://doi.org/10.1145/3276914}{https://doi.org/10.1145/3276914};



\item Artem Pelenitsyn et al., "Type Stability in Julia: Avoiding
  Performance Pathologies in JIT Compilation",
  https://doi.org/10.1145/3485527,
  \href{arXiv:2109.01950}{https://arxiv.org/abs/2109.01950};



\item \textit{How SQLite Is Tested},
  \href{https://www.sqlite.org/testing.html}{https://www.sqlite.org/testing.html};



\item \textit{SpotBugs},
  \href{https://spotbugs.github.io/}{https://spotbugs.github.io/};



\item \textit{A tool to detect bugs in Java and C/C++/Objective-C code
    before it ships},
  \href{https://fbinfer.com/}{https://fbinfer.com/la};



\item \textit{Go 1 and the Future of Go Programs},
  \href{https://go.dev/doc/go1compat}{https://go.dev/doc/go1compat};



\item \textit{SD-8: Standard Library Compatibility},
  \href{https://isocpp.org/std/standing-documents/sd-8-standard-library-compatibility}{https://isocpp.org/std/standing-documents/sd-8-standard-library-compatibility};



\item \textit{GNU General Public License},
  \href{https://www.gnu.org/licenses/gpl-3.0.html}{https://www.gnu.org/licenses/gpl-3.0.html};



\item \textit{Reflections on trusting trust},
  \href{https://dl.acm.org/doi/10.1145/358198.358210}{https://dl.acm.org/doi/10.1145/358198.358210};



\item \textit{Go \& Versioning},
  \href{https://research.swtch.com/vgo}{https://research.swtch.com/vgo};



\item \textit{Why Google stores billions of lines of code in a single
    repository},
  \href{https://dl.acm.org/doi/10.1145/2854146}{https://dl.acm.org/doi/10.1145/2854146};



\item \textit{Testing Chromium: ThreadSanitizer v2, a next-gen data
    race
    detector}, \\
  \href{https://blog.chromium.org/2014/04/testing-chromium-threadsanitizer-v2.html}{https://blog.chromium.org/2014/04/testing-chromium-threadsanitizer-v2.html};



\item \textit{Search Vulnerability Database},
  \href{https://nvd.nist.gov/vuln/search}{https://nvd.nist.gov/vuln/search};



\item \textit{Regular Expression Matching with a Trigram Index or How
    Google Code Search Worked}, \\
  \href{https://swtch.com/~rsc/regexp/regexp4.html}{https://swtch.com/~rsc/regexp/regexp4.html};



\item \textit{Licenses},
  \href{https://opensource.google/docs/thirdparty/licenses}{https://opensource.google/docs/thirdparty/licenses};



\item \textit{ImperialViolet},
  \href{https://www.imperialviolet.org/2009/08/26/seccomp.html}{https://www.imperialviolet.org/2009/08/26/seccomp.html};



\item \textit{Multi-process Architecture},
  \href{https://blog.chromium.org/2008/09/multi-process-architecture.html}{https://blog.chromium.org/2008/09/multi-process-architecture.html};



\item \textit{A single Node of failure},
  \href{https://lwn.net/Articles/681410/}{https://lwn.net/Articles/681410/};



\item \textit{Interpreting the Data: Parallel Analysis with Sawzall},
  \href{https://www.hindawi.com/journals/sp/2005/962135/}{https://www.hindawi.com/journals/sp/2005/962135/};



\item \textit{Go Proverbs},
  \href{https://go-proverbs.github.io/}{https://go-proverbs.github.io/};



\item \textit{RE2: a principled approach to regular expression
    matching}, \\
  \href{https://opensource.googleblog.com/2010/03/re2-principled-approach-to-regular.html}{https://opensource.googleblog.com/2010/03/re2-principled-approach-to-regular.html};



\item \textit{Details about the event-stream incident}, \\
  \href{https://blog.npmjs.org/post/180565383195/details-about-the-event-stream-incident}{https://blog.npmjs.org/post/180565383195/details-about-the-event-stream-incident};



\item \textit{Open-sourcing gVisor, a sandboxed container runtime}, \\
  \href{https://cloud.google.com/blog/products/identity-security/open-sourcing-gvisor-a-sandboxed-container-runtime}{https://cloud.google.com/blog/products/identity-security/open-sourcing-gvisor-a-sandboxed-container-runtime};



\item Samuel R. Buss, Alexander S. Kechris, Anand Pillay, Richard A.
  Shore, \textit{The prospects for mathematical logic in the
    twenty-first century},
  \href{https://arxiv.org/abs/cs/0205003v1}{arXiv:cs/0205003v1};



\item Christian Retore, \textit{On the system F as a glue language for
    natural-language compositional-semantics},
  \href{https://arxiv.org/abs/1108.5084}{arXiv:1108.5084};



\item Robert Harper, \textit{An Equational Logical Framework for Type
    Theories},
  \href{https://arxiv.org/abs/2106.01484}{arXiv:2106.01484};



\item Jan Leike, et al., \textit{AI Safety Gridworlds},
  \href{https://arxiv.org/abs/1711.09883}{arXiv:1711.09883v2};



\item Sergei Gukov, Edward Witten, \textit{Branes and Quantization},
  \href{https://arxiv.org/abs/0809.0305}{https://arxiv.org/abs/0809.0305};



\item R. Estrada, J. M. Gracia-Bondia, J. C. Varilly, \textit{On
    summability of distributions and spectral geometry},
  \href{https://arxiv.org/abs/funct-an/9702001v1}{arXiv:funct-an/9702001};



\item Ghanashyam Date, \textit{Lectures on Constrained Systems},
  \href{https://arxiv.org/abs/1010.2062v1}{arXiv:1010.2062};



\item Frank Wilczek, \textit{Quantum Field Theory},
  \href{https://arxiv.org/abs/hep-th/9803075v2}{arXiv:hep-th/9803075};



\item Karl Michael Schmidt, Karl Michael Schmidt, \textit{Schnol’s
    Theorem and Spectral Properties of Massless Dirac Operators with
    Scalar Potentials};



\item Peter J. Olver, \textit{Dirac’s theory of constraints in fields
    theory and the canonical form of Hamiltonian differential
    operators};



\item Lorenzo Iorio, \textit{Editorial for the Special Issue 100 Years
    of Chronogeometrodynamics: The Status of the Einstein's Theory of
    Gravitation in Its Centennial Year},
  \href{https://arxiv.org/abs/1504.05789v2}{arXiv:1504.05789};



\item B. Mutet, P. Grang\'{e}, E. Werner, \textit{Taylor–Lagrange
    renormalization and gauge theories in four dimensions};



\item \textit{Surviving Software Dependencies},
  \href{https://queue.acm.org/detail.cfm?id=3344149}{https://queue.acm.org/detail.cfm?id=3344149};



\item Clifford M. Will, \textit{The Confrontation between General
    Relativity and Experiment};



\item Paul Lopes, \textit{Culture and Stigma: Popular Culture and the
    Case of Comic Books};



\item Marina S.Butuzova 1, Alexander B. Pushkarev, \textit{Is OJ 287 a
    Single Supermassive Black Hole?};



\item Peter Selinger, \textit{Lecture notes on the lambda calculus},
  \href{https://arxiv.org/abs/0804.3434v2}{arXiv:0804.3434};



\item Alex Eskin, Maryam Mirzakhani, \textit{Counting closed geodesics
    in Moduli space},
  \href{https://arxiv.org/abs/0811.2362v3}{arXiv:0811.2362};



\item Jacques Carette, James H. Davenport, \textit{The Power of
    Vocabulary: The Case of Cyclotomic Polynomials},
  \href{https://arxiv.org/abs/1002.0012v1}{arXiv:1002.0012};



\item Chris Kapulkin, Peter LeFanu Lumsdaine, \textit{The Simplicial
    Model of Univalent Foundations (after Voevodsky)},
  \href{https://arxiv.org/abs/1211.2851v5}{arXiv:1211.2851};



\item Christopher J. Fewster, Rainer Verch, \textit{Quantum fields and
    local measurements},
  \href{https://arxiv.org/abs/1810.06512}{arXiv:1810.06512};



\item Paweł Duch, \textit{Infrared problem in perturbative quantum
    field theory},
  \href{https://arxiv.org/abs/1906.00940}{arXiv:1906.00940};



\item Christopher J. Fewster, \textit{A generally covariant
    measurement scheme for quantum field theory in curved spacetimes},
  \href{https://arxiv.org/abs/1904.06944v1}{arXiv:1904.06944};



\item Jacob Lurie, \textit{Higher Topos Theory},
  \href{https://arxiv.org/abs/math/0608040v4}{arXiv:math/0608040};



\item Konrad Osterwalder, and Robert Schrader, \textit{Axioms for
    Euclidean Green's Functions};



\item Vladimir Voevodsky, \textit{A very short note on homotopy
    $\lambda$-calculus};



\item Charles Rezk, \textit{Toposes and homotopy toposes};



\item Fredrik Nordvall Forsberg, Anton Setzer, \textit{A finite
    axiomatisation of inductive-inductive definitions};



\item Egbert Rilke, \textit{Introduction to homotopy type theory};



\item \'{A}lvaro Pelayo, Michael A. Warren, \textit{Homotopy type
    theory and Voevodsky’s Univalent Foundations};



\item H. Simmons, A. Schalk, \textit{An introduction to
    $\lambda$-calculi and arithmetics};



\item Roderich Tumulka, \textit{Lecture Notes on Mathematical
    Statistical Physics};



\item Martin Hofmann, \textit{Extensional concepts in intensional type
    theory};



\item Eugenio Moggi, \textit{Computational $\lambda$-calculus and monads};



\item \textit{Proof-theoretic semantics. Assessment and Future
    Perspectives};



\item Philip Walder, \textit{Propostions as Types};



\item Egbert Rijke, \textit{Homotopy type theory};



\item Eugenio Moggi, \textit{Notions of computation and monads};



\item Bruno Barras, Thierry Coquand and Simon Huber, \textit{A
    Generalization of Takeuti-Gandy Interpretation};



\item Michael Alton Warren, \textit{Homotopy Theoretic Aspects of
    Constructive Type Theory};



\item Vladimir Voevodsky, \textit{A universe polymorphic type system};



\item S. Marmi, \textit{An Introduction To Small Divisors},
  \href{https://arxiv.org/abs/math/0009232v1}{arXiv:math/0009232};



\item Paweł Duch, Michael Duetsch, Jose M. Gracia-Bondia,
  \textit{Diphoton decay of the higgs from the Epstein--Glaser
    viewpoint},
  \href{https://arxiv.org/abs/2011.12675v2}{arXiv:2011.12675};



\item Juliette Kennedy, Menachem Magidor, Jouko V\"{a}\"{a}n\"{a}nen,
  \textit{Inner Models from Extended Logics: Part 1},
  \href{https://arxiv.org/abs/2007.10764}{arXiv:2007.10764};



\item Paul W. Gross, P. Robert Kotiuga, \textit{Electromagnetic Theory
    and Computation: A Topological Approach};



\item Andreas R. Blass, Jeffry L. Hirst, and Stephen G. Simpson,
  Logical analysis of some theorems of combinatorics and topological
  dynamics, Logic and combinatorics (Arcata, Calif., 1985), Contemp.
  Math., vol. 65, Amer. Math. Soc., Providence, RI, 1987, pp. 125–156.



\item Jared Corduan, Marcia Groszek, and Joseph Mileti, Draft: A note
  on reverse mathematics and partitions of trees.



\item Jennifer Chubb, Jeffry Hirst, and Tim McNichol, Reverse
  mathematics and partitions of trees. To appear in J. Symbolic Logic.



\item Damir Dzhafarov and Jeffry Hirst, The polarized Ramsey theorem.
  Archive for Math. Logic, Online First: 2008.



\item Neil Hindman, The existence of certain ultra-filters on N and a
  conjecture of Graham and Rothschild, Proc. Amer. Math. Soc. 36
  (1972), 341–346.



\item Jeffry L. Hirst, Hindman’s theorem, ultrafilters, and reverse
  mathematics, J. Symbolic Logic 69 (2004), no. 1, 65–72.



\item Carl G. Jockusch Jr., Ramsey’s theorem and recursion theory, J.
  Symbolic Logic 37 (1972), 268–280.



\item J. Mileti, Partition theory and computability theory. Ph.D.
  Thesis.



\item Alexander Lange, \textit{The Epstein-Glaser approach to pQFT:
    graphs and Hopf algebras},
  \href{https://arxiv.org/abs/hep-th/0403246v4}{arXiv:hep-th/0403246v4};



\item Michael Murray, \textit{Line bundles};



\item Andrzej Dragan, \textit{Niezwykle szczególna teoria
    względności};



\item J\"{o}rg Frauendiener, \textit{Conformal infinity};



\item J\o rgen Bang-Jensen, Gregory Gutin, \textit{Digraphs Theory,
    Algorithms and Applications};



\item Liviu I. Nicolaescu, \textit{Notes on the Atiyah-Singer Index
    Theorem};



\item Bert Schroer, \textit{A note on Infraparticles and Unparticles},
  \href{https://arxiv.org/abs/0804.3563}{arXiv:0804.3563};



\item Bert Schroer, \textit{Wigner's infinite spin representations and
    inert matter},
  \href{https://arxiv.org/abs/1601.02477}{arXiv:1601.02477};



\item J. Dimock \textit{Ultraviolet Stability for QED in $d = 3$},
  \href{https://arxiv.org/abs/2009.01156}{arXiv: 2009.01156};



\item J. Dimock \textit{Stability for QED in $d = 3$: an overview},
  \href{https://arxiv.org/abs/2204.07201}{arXiv: 2204.07201};



\item














































































































































































































































\end{enumerate}
% ##################










% ############################

% Koniec dokumentu
\end{document}
