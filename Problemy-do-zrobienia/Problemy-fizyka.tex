% ---------------------------------------------------------------------
% Podstawowe ustawienia i pakiety
% ---------------------------------------------------------------------
\RequirePackage[l2tabu, orthodox]{nag} % Wykrywa przestarzałe i niewłaściwe
% sposoby używania LaTeXa. Więcej jest w l2tabu English version.
\documentclass[a4paper,11pt]{article}
% {rozmiar papieru, rozmiar fontu}[klasa dokumentu]
\usepackage[MeX]{polski} % Polonizacja LaTeXa, bez niej będzie pracował
% w języku angielskim.
\usepackage[utf8]{inputenc} % Włączenie kodowania UTF-8, co daje dostęp
% do polskich znaków.
\usepackage{lmodern} % Wprowadza fonty Latin Modern.
\usepackage[T1]{fontenc} % Potrzebne do używania fontów Latin Modern.



% ---------------------------------------
% Podstawowe pakiety (niezwiązane z ustawieniami języka)
% ---------------------------------------
\usepackage{microtype} % Twierdzi, że poprawi rozmiar odstępów w tekście.
% \usepackage{graphicx} % Wprowadza bardzo potrzebne komendy do wstawiania
% grafiki.
% \usepackage{verbatim} % Poprawia otoczenie VERBATIME.
% \usepackage{textcomp} % Dodaje takie symbole jak stopnie Celsiusa,
% wprowadzane bezpośrednio w tekście.
\usepackage{vmargin} % Pozwala na prostą kontrolę rozmiaru marginesów,
% za pomocą komend poniżej. Rozmiar odstępów jest mierzony w calach.
% ---------------------------------------
% MARGINS
% ---------------------------------------
\setmarginsrb
{ 0.7in}  % left margin
{ 0.6in}  % top margin
{ 0.7in}  % right margin
{ 0.8in}  % bottom margin
{  20pt}  % head height
{0.25in}  % head sep
{   9pt}  % foot height
{ 0.3in}  % foot sep



% ---------------------------------------
% Często używane pakiety
% ---------------------------------------
% \usepackage{csquotes} % Pozwala w prosty sposób wstawiać cytaty do tekstu.



% ---------------------------------------
% Pakiety do tekstów z nauk przyrodniczych
% ---------------------------------------
\let\lll\undefined % Amsmath gryzie się z językiem pakietami do języka
% % polskiego, bo oba definiują komendę \lll. Aby rozwiązać ten problem
% % oddefiniowuję tę komendę, ale może tym samym pozbywam się dużego Ł.
\usepackage[intlimits]{amsmath} % Podstawowe wsparcie od American
% Mathematical Society (w skrócie AMS)
\usepackage{amsfonts, amssymb, amscd, amsthm} % Dalsze wsparcie od AMS
% % \usepackage{siunitx} % Dla prostszego pisania jednostek fizycznych
\usepackage{upgreek} % Ładniejsze greckie litery
% % Przykładowa składnia: pi = \uppi
\usepackage{siunitx}
% \usepackage{slashed} % Pozwala w prosty sposób pisać slash Feynmana.
% \usepackage{calrsfs} % Zmienia czcionkę kaligraficzną w \mathcal
% % na ładniejszą. Może w innych miejscach robi to samo, ale o tym nic
% % nie wiem.





% ---------------------------------------
% Dodatkowe ustawienia dla języka polskiego
% ---------------------------------------
\renewcommand{\thesection}{\arabic{section}.}
% Kropki po numerach rozdziału (polski zwyczaj topograficzny)
\renewcommand{\thesubsection}{\thesection\arabic{subsection}}
% Brak kropki po numerach podrozdziału



% ---------------------------------------
% Pakiety napisane przez użytkownika.
% Mają być w tym samym katalogu to ten plik .tex
% ---------------------------------------
\usepackage{mathcommands}
\usepackage{functionalanalysiscommands}





% ---------------------------------------
% Pakiet "hyperref"
% Polecano by umieszczać go na końcu preambuły.
% ---------------------------------------
\usepackage{hyperref} % Pozwala tworzyć hiperlinki i zamienia odwołania
% do bibliografii na hiperlinki.










% ---------------------------------------------------------------------
% Tytuł tekstu
\title{Problemy do zrobienia z~fizyki}

% \author{}


% \date{}
% ---------------------------------------------------------------------










% ####################################################################
% Początek dokumentu
\begin{document}
% ####################################################################





% ######################################
\maketitle  % Tytuł całego tekstu
% ######################################





% ######################################
\section{Mechanika teoretyczna, problemy do zrobienia}

% \vspace{\spaceTwo}

% ######################################



% ##################
\begin{enumerate}

\item Narysować portrety fazowe dla ruchu jednowymiarowego po osi $x$ dla
  następujących potencjałów.

  a) $V( x ) = \frac{ 1 }{ 2 } k x^{ 2 }$, $k > 0$; \\
  b) $V( x ) = \frac{ 1 }{ 2 } k x^{ 2 } + \frac{ 1 }{ 4 } b x^{ 4 }$;
  rozważyć osobno dodatnie i~ujemne $b$;
  $k > 0$, \\
  c) $V( x ) = -A \cos( x )$, $A > 0$.

  Jak okres rozwiązań periodycznych zależy od energii?



\item Narysować portrety fazowe dla jednowymiarowego ruchu po osi~$x$
  w~następujących potencjałach \\
  a) $V( x ) = -1 / \cosh( x )^{ 2 }$; \\
  b) $V( x ) = \tan( x )^{ 2 }$.



\item Znaleźć krzywe geodezyjne na powierzchni cylindra.



\item Napisać lagranżja i~równania ruchu cząstki swobodnej we współrzędnych
  cylindrycznych i~sferycznych.



\item Napisać lagranżjan dla cząstki swobodnej, która znajduje~się:
  a) w~układzie obracającym~się wokół osi~$z$ ze stałą prędkością
  kątową~$\Omega$; \\
  b) w~układzie poruszającym~się ze stałym przyśpieszeniem~$a$ względem
  układu inercjalnego.



\item Klocek o~masie $m$ jest utrzymywany w~spoczynku na równi pochyłej
  (o~masie $M$), która leży na poziomej płaszczyźnie. W~pewnej chwili klocek
  jest puszczony. Znaleźć przyśpieszenie równi pochyłej.



\item Dwie jednakowe masy~$m$ są połączone nicią przewieszoną przez dwa
  bloczki. Lewa masa porusza~się pionowo, a~prawa może~się huśtać. Napisać
  równania ruchu.



\item Punk materialny porusza~się po~powierzchni stożka ustawionego pionowo
  na~wierzchołku. Napisać równania ruchu.



\item Dwie cząstki o~masach $m$ i~$M$ połączone są nitką przechodzącą
  przez otwór w~stole. Jednak cząstka porusza~się po~(poziomej) powierzchni
  stołu, a~druga porusza~się w~pionie. \\
  a) Napisać lagranżjan i~równania ruchu. \\
  b) Znaleźć całki ruchu. \\
  c) Znaleźć rozwiązanie stacjonarne i~zbadać jego stabilność.



\item Koralik o~masie $m$~się po obręczy o~promieniu~$R$, która wiruje wokół
  pionowej osi (zgodnej z~kierunkiem pola ciężkości $g$) ze stałą prędkością
  kątową $\omega$. \\
  a) Napisać lagranżjan. \\
  b) Znaleźć całki ruchu. \\
  c) Znaleźć położenie równowagi i~zbadać ich stabilność w~zależności od
  bezwymiarowego parametru $g / R \Omega^{ 2 }$.



\item Cząstka porusza~się w~potencjale centralnym
  \begin{equation*}
    U( \vecx ) = -\frac{ k }{ r } - \frac{ \varepsilon }{ r^{ 3 } }, \quad
    k > 0, \varepsilon > 0.
  \end{equation*}

  a) Znaleźć potencjał efektywny i~przedyskutować jakościowy charakter
  ruchu w~zależności od energii i~momentu pędu. \\
  b) Pokazać, że~orbita kołowa o~promieniu $R$ po małym zaburzeniu zacznie
  ulegać precesji. Znaleźć kąt precesji.



\item Niech infinitezymalne przekształcenie współrzędnych i~czasu ma
  postać:
  \begin{align*}
    q_{ i }' &= q_{ i } + \varepsilon \Psi_{ i }( q, t ), \\
    t' &= t + \varepsilon X( q, t ), \quad
         \varepsilon \searrow 0.
  \end{align*}

  i~niech przy tym przekształcenie zachowuje~się postać całki działania
  \begin{equation*}
    \int_{ t_{ 1 } }^{ t_{ 2 } } dt \, L\left( q( t ), \frac{ dq( t ) }{ dt },
      t \right) =
    \int_{ t_{ 1 }' }^{ t_{ 2 }' } dt' \, L\left( q'( t ),
      \frac{ dq'( t ) }{ dt }, t' \right).
  \end{equation*}

  Wykazać, że~wyrażenie
  \begin{equation*}
    \sum_{ i } \frac{ \partial L }{ \partial \dot{q}_{ i } } \cdot ( q_{ i } X - \Psi_{ i } )
    - L \cdot X
  \end{equation*}
  jest wtedy całką pierwszą ruchu.



\item Korzystając z~zadania 2 znaleźć trzecią, obok energii i~momentu pędu,
  całkę ruchu w~potencjale $U( \vecx ) = -k / r^{ 2 }$, a~następnie
  wyznaczyć równanie orbity w~sposób algebraiczny.



\item Cząstka porusza~się w~polu centralnym po spirali
  $r = A e^{ a \phi }$, gdzie $A$ i~$a$ to różne od zera stałe. Znaleźć
  potencjał.



\item Płaskie wahadło o~masie $m$ i~ramieniu $l$ jest przymocowana do~klocka
  o~masie $M$, który porusza~się po poziomej szynie. Znaleźć małe drgania
  (tzn. drgania i~częstości własne): \\
  a) w~przypadku, gdy klocek jest przymocowany do ściany sprężyną; \\
  b) w~przypadku, gdy klocek porusza~się swobodnie.



\item Wyznaczyć drgania normalne dwóch cząstek o~masie $m$, powiązanych
  między sobą i~ze stałym punktem $A$ jednakowymi sprężynkami, które
  mogą~się poruszać wzdłuż pierścienia. Wyznaczyć współrzędne normalne,
  dla których lagranżjan przejmuje postać sumy kwadratów.



\item Wyznaczyć drgania swobodne układu rozważanego w~poprzednim zadaniu,
  jeśli w~chwili początkowej jednak cząstka była wychylona z~położenia
  równowagi. Prędkości początkowe obu cząstek są równe zero.



\item Napisać lagranżaj dla podwójnego płaskiego wahadła o~masach
  $m_{ 1 }$, $m_{ 2 }$ i~ramionach $l_{ 1 }$, $l_{ 2 }$. Znaleźć małe
  drgania w~przypadku, gdy $l_{ 1 } = l_{ 2 }$ i~$m_{ 1 } = m_{ 2 }$.



\item Langranżjan nierelatywistycznej cząstki o~ładunku~$e$ poruszającej~się
  w~polu elektromagnetycznym ma postać
  \begin{equation*}
    L =
    \frac{ m }{ 2 } \vecvbold^{ 2 }
    + \frac{ e }{ c } \vecvbold \cdot \vecAbold - e \phi.
  \end{equation*}
  a) Napisać równanie ruchu.
  b) Rozważyć przypadek szczególny $\phi = 0$,
  $\vecAbold = \frac{ 1 }{ 2 } \vecHbold \times \vecrbold$, gdzie $\vecHbold$
  jest stałym wektorem. Pokazać, że~wielkość
  \begin{equation*}
    ( \vecrbold \times \vecvbold ) \cdot \vecHbold
    + \frac{ e }{ 2 mc } ( \vecrbold \times \vecHbold )^{ 2 }
  \end{equation*}
  jest zachowana.



\item Wyznaczyć prawo ruchu cząstki naładowanej w~jednorodnym polu
  magnetycznym $\vecHbold$ rozwiązując równania Hamiltona.



\item Napisać hamiltonian dla oscylatora anharmonicznego, dla którego
  lagranżjan ma postać
  \begin{equation*}
    L =
    \frac{ \dot{x}^{ 2 } }{ x } + \frac{ \omega^{ 2 } }{ 2 } x^{ 2 }
    - \alpha x^{ 3 } + \beta x \dot{x}^{ 2 }.
  \end{equation*}



\item Napisać hamiltonian dla cząstki swobodnej w~cylindrycznym układzie
  współrzędnych.



\item Sprawdzić, że~równanie ruchu cząstki:
  \begin{equation*}
    \dot{\vecv} = \vec{\omega} \times \vecx
  \end{equation*}
  opisuje ruch po okręgu ze stałą prędkością kątową. Ruch odbywa~się
  w~płaszczyźnie prostopadłej do wektora $\vec{\omega}$, kierunek ruchu jest
  zgodny ze zwrotem wektora $\vec{\omega}$ wedle reguły śruby lewoskrętnej,
  natomiast prędkość kątowa cząstki jest równa długości wektora $\vec{\omega}$.



\item Sprawdzić, że~transformacja
  \begin{equation*}
    Q =     \ln\left( \frac{ \sin( q ) }{ q } \right), \quad
    P = q \cot( p )
  \end{equation*}
  jest kanoniczna. Znaleźć funkcje tworzącą w~zmiennych $( p, Q )$
  i~$( q, P )$.



\item Układ mechaniczny z~jednym stopniem swobody ma hamiltonian
  \begin{equation*}
    H = p \cos( 2q ) \sin( 2q ).
  \end{equation*}
  a) Jakie warunki muszą spełniać stałe, $a = b = 1 / 2$, $c = 2$. \\
  b) Znaleźć $q( t )$ i~$p( t )$ z~warunkami początkowymi
  $q( 0 ) = \pi / 8$, $p( 0 ) = 2$. Wskazówka. Korzystając z~a) najpierw
  rozwiązać równania Hamiltona we~współrzędnych $Q, P$.



\item Pokazać, że~transformacja
  \begin{equation*}
    Q = ( 1 / q^{ 2 } ) + \ln( t p q^{ 3 } ), \quad
    P = p q^{ 3 } ( 1 + t \exp( 1 / q^{ 2 } ) )
  \end{equation*}
  jest kanoniczna i~zastosować ją do hamiltonianu $H = p q^{ 3 } / 2 t$.



\item Wyznaczyć częstość małych drgań jednorodnej półkuli, umieszczonej
  na~gładkiej płaszczyźnie poziomej w~polu sił ciężkości.



\item Moment pędu cząstki definiujemy jako
  $\vecMbold = \vecxbold \times \vecpbold$. Obliczyć podane nawiasy Poissona:
  \begin{equation*}
    \{ M_{ i }, x_{ j } \}, \quad \{ M_{ i }, p_{ j } \}, \quad
    \{ M_{ i }, M_{ j } \}.
  \end{equation*}



\item Lagranżjan swobodnej cząstki relatywistycznej ma postać:
  \begin{equation*}
    L = -m c^{ 2 } \sqrt{ 1 - v^{ 2 } / c^{ 2 } }.
  \end{equation*}
  Znaleźć hamiltonian.



\item Lagranżjan cząstki relatywistycznej poruszającej~się w~jednorodnym
  polu elektrycznym ma postać:
  \begin{equation*}
    L = -m c^{ 2 } \sqrt{ 1 - v^{ 2 } / c^{ 2 } } + e \, E \, x.
  \end{equation*}
  Wyprowadzić równanie ruchu i~rozwiązać je.



\item Hamiltonian ma postać $H = p^{ 2 } / 2m + k / q^{ 2 }$. Stosując
  metodę Hamiltona-Jacobiego rozwiązać równania ruchu dla warunków
  początkowych $q( 0 ) = 1$, $p( 0 ) = 0$.



\item Pokazać, że hamiltonian cząstki naładowanej, która porusza~się
  po~płaszczyźnie w~stałym polu magnetycznym $\vecBbold$ prostopadłym
  do~płaszczyzny ruchu ma postać (w~odpowiednim cechowaniu)
  \begin{equation*}
    H( q, p ) =
    \frac{ 1 }{ 2m } \left[ ( p_{ x } + k \, y )^{ 2 }
      + ( p_{ y } + k \, x )^{ 2 } \right], \quad
    k = \absOne{ \vecBbold } / 2.
  \end{equation*}
  Napisać równanie Hamiltona-Jacobiego i~znaleźć jego całkę zupełną
  zakładając, że~funkcja charakterystyczna Hamiltona ma postać
  $W( x, y ) = k x y + X( x ) + Y( y )$.



\item Wahadło matematyczne porusza~się w~płaszczyźnie pod kątem $\alpha$
  do~poziomu. Znaleźć zmianę amplitudy oscylacji przy powolnej zmianie
  kąta nachylenia $\alpha$ (założyć małe drgania).



\item Piłka odbija~się elastycznie od~podłogi na wysokość~$h$.
  O~ile zmieni~się $h$, jeśli natężenie pola grawitacyjnego zmieni~się
  bardzo powoli o~10\%?


\end{enumerate}
% ##################










% ######################################
\section{Elektromagnetyzm, problemy do zrobienia}

% \vspace{\spaceTwo}

% ######################################



% ##################
\begin{enumerate}

\item W wierzchołkach wieloboku foremnego o liczbie boków $n$ i o
  długości boku $a$ znajdują się (jednakowe) ujemne ładunki $e$. Jaką
  pracę należy wykonać, aby dwukrotnie zmniejszyć liniowe rozmiary
  wieloboku dla $n = 2$ (odcinek, oba ładunki $e$ na końcach),
  $n = 3$, $n = 4$, $n > 4$.

\item Proszę znaleźć potencjał oraz pole elektryczne wytwarzane przez
  dipol (ładunki $-q$ i $-q$ odległe o $l$): wzdłuż linii dipola, na
  symetralnej, w dowolnym punkcie.

\item Proszę policzyć z definicji pojemność kondensatora: płaskiego,
  zbudowanego z dwóch współśrodkowych sfer o promieniach $r_{ 1 }$ i
  $r_{ 2 }$, zbudowanego z dwóch współosiowych walców o promieniach
  $r_{ 1 }$ i $r_{ 2 }$ oraz długości $l$.

\item W odległości $2D$ od płaskiej, cienkiej, przewodzącej,
  uziemionej, nieskończonej płyty (nieskończonej płaszczyzny) znajduje
  się ładunek $e$. Wprowadzamy układ $xyz$ w taki sposób, że osie $xy$
  leżą w płaszczyźnie, a oś $z$ przechodzi przez $e$. Na osi $z$
  umieszczamy ładunek $q$, w odległości $D$ od płyty, czyli w połowie
  odległości między $e$ i płytą. Proszę policzyć jaka siła działa na
  ładunek $e$ i znaleźć wartość $q$, dla której ta siła się zeruje.

\item W powyższym zadaniu ładunek $e$ ponownie znajduje się w
  odległości $2D$ od płaszczyzny, ale ładunek $q$ proszę umieścić na
  osi $z$ w odległości $Z$ od płaszczyzny. Proszę założyć, że ładunek
  $q$ ma taki sam znak oraz wartość jak ładunek $e$ ($q = e$). Proszę
  znaleźć wszystkie możliwe położenia ładunku $q$ na osi $z$
  (wszystkie możliwe odległości $Z$) dla których siła działa na
  ładunek $e$ się zeruje.

\item Wewnątrz metalowej, przewodzącej, elektrycznie neutralnej,
  nieuziemionej sfery o promieniu $R$ (o grubości sferycznej powłoki
  $r$ znacznie mniejszej od promienia $R$ sfery, $r \ll R$, czyli
  zaniedbywalnie małej grubości skorupy) umieszczono ładunek punktowy,
  w pierwszym przypadku w środku sfery, w drugim w odległości od
  środka sfery $d < R$. Proszę policzyć potencjał wewnątrz i na
  zewnątrz sfery. Proszę policzyć i narysować pole na zewnątrz sfery.

\item Proszę policzyć z definicji pojemność kondensatora (A)
  płaskiego, (B) zbudowanego z dwóch współśrodkowych sfer o
  promieniach $r_{ 1 }$ i $r_{ 2 }$, (C) zbudowanego z dwóch
  współosiowych walców o promieniach $r_{ 1 }$ i $r_{ 2 }$ oraz
  długości $l$. Kondensatory są wypełnione dielektrykiem o
  przenikalności elektrycznej $\epsilon$.

\end{enumerate}
% ##################










% ######################################
\section{Mechanika kwantowa, problemy do zrobienia}

% \vspace{\spaceTwo}

% ######################################



% ##################
\begin{enumerate}

\item Dystrybucję $\delta( x )$ Diraca można określić jako granicę
  funkcji regularnych
  \begin{equation}
    \label{QM:01}
    \delta( x ) = \lim_{ a \to 0 } \rho_{ a }( x ), \qquad
    \rho_{ a }( x ) =
    \frac{ 1 }{ a \sqrt{ \pi } } e^{ -\frac{ x^{ 2 } }{ a^{ 2 } } }.
  \end{equation}
  Mówimy, że ciąg funkcji $\rho_{ a }( x )$ jest modelem dla funkcji
  $\delta( x )$. Udowodnić, że $\delta( x )$ ma następujące własności
  \begin{align}
    &\int_{ a }^{ b } \delta( x - x_{ 0 } ) \, dx = 1, \qquad
      x_{ 0 } \in ( a, b ) \\
    &\int_{ a }^{ b } \delta( x - x_{ 0 } ) \, dx = 0, \qquad
      x_{ 0 } \notin ( a, b ) \\
    &\int_{ -\infty }^{ +\infty } \delta( x - x_{ 0 } ) \phi( x ) \, dx = \phi( x_{ 0 } )
  \end{align}
  dla dowolnej funkcji próbnej (tj. regularnej) $\phi( x )$. \\
  Wykorzystując ostatnią własność, oraz właściwości odwrotnej
  transformaty Fouriera udowodnić, że
  \begin{equation}
    \label{QM:02}
    \delta( x ) = \frac{ 1 }{ 2\pi } \int_{ -\infty }^{ +\infty } e^{ i p x } \, dp
  \end{equation}
  Zapisując ostatnią całkę jako
  \begin{equation}
    \label{QM:03}
    \lim_{ \varepsilon \to 0 } \frac{ 1 }{ 2\pi } \int_{ -\infty }^{ +\infty } e^{ ipx - \varepsilon | p | } \, dp,
  \end{equation}
  udowodnić inną, równoważną, reprezentację dystrybucji $\delta( x )$
  \begin{equation}
    \label{QM:04}
    \delta( x ) = \lim_{ \varepsilon \to 0 } f_{ \varepsilon }( x ), \qquad
    f_{ \varepsilon }( x ) = \frac{ 1 }{ \pi } \frac{ \varepsilon }{ x^{ 2 } + \varepsilon^{ 2 } }
  \end{equation}

  Wykreślić funkcje modelujące dla kilku wartości $\varepsilon$.
  Uzasadnić, że rzeczywiście jest to dobry model funkcji $\delta$.
  W~szczególności, sprawdzić warunek normalizacji oraz że dla dowolnej
  funkcji próbnej zachodzi
  \begin{equation}
    \label{QM:05}
    \lim_{ \varepsilon \to 0 } \int_{ -\infty }^{ +\infty } f_{ \varepsilon }( x - x_{ 0 } ) \phi( x ) \, dx
    =
    \phi( x_{ 0 } )
  \end{equation}
  Ostatnią całkę wykonać metodą residuów zakładając wystarczająco
  szybkie znikanie funkcji próbnej na dużych okręgach.

  Udowodnić, że fale płaskie są znormalizowane do $\delta$ Diraca.

\item Rozwiązać zależne od czasu równanie Schr\"{o}dingera dla cząstki
  swobodnej w jednym i trzech wymiarach, za pomocą separacji
  zmiennych. Jako warunek początkowy przyjąć, że gęstość
  prawdopodobieństwa znalezienia cząstki w~przestrzeni ma rozkład
  gaussowski o~dyspersji $\sigma^{ 2 } = a^{ 2 }$ i że pakiet ma
  średnią prędkość $v_{ 0 }$. Wykonać \textit{explicite} wszystkie
  transformaty Fouriera i~uzyskać jawne wyrażenie na $\Psi( x, t )$.
  Obliczyć gęstość prawdopodobieństwa $\rho( x, t )$ i gęstość
  strumienia prawdopodobieństwa $\vec{ j }( x, t )$. Sprawdzić
  równanie ciągłości. Wyprowadzić analogiczne wzory (np. na
  $\vec{ j }( x, t )$) dla zbioru cząstek klasycznych.

\item Show that if the spinors $w^{ r }( \vec{ p } )$ satisfy the
  normalization
  $w^{ r }( \epsilon_{ r } \vec{ p } )^{ \dagger } w^{ r' }(
  \epsilon'_{ r } \vec{ p } ) = 2 E_{ p } \delta_{ r, \, r' }$ it
  follows that
  $\bar{w}^{ r }( \vec{ p } ) w^{ r' }( \vec{ p } ) = 2 \epsilon_{ r }
  m c^{ 2 } \delta_{ r,\, r' }$. Here
  \begin{equation}
    \label{QM:06}
    \bar{w}^{ r }( \vec{ p } )
    = w^{ r }( \vec{ p } )^{ \dagger } \gamma^{ 0 }.
  \end{equation}
  Calculate quantities
  \begin{equation}
    \label{QM:07}
    a_{ r }^{ \mu }
    = \bar{w}^{ r }( \vec{ p } ) \gamma^{ \mu } w^{ r }( \vec{ p } ), \quad
    b_{ r }^{ \mu \nu }
    = \bar{w}^{ r }( \vec{ p } ) \sigma^{ \mu \nu } w^{ r }( \vec{ p } ), \quad
    c_{ r }
    = \bar{w}^{ r }( \vec{ p } ) \gamma_{ 5 } w^{ r }( \vec{ p } ).
  \end{equation}

\item Show that if we boost the solution of the Dirac equation
  \begin{equation}
    \label{QM:08}
    \Psi_{ \vec{ p } }^{ r }( x )
    = e^{ -i \epsilon_{ r } m c^{ 2 } t } w^{ r }( \vec{ p } = 0 )
  \end{equation}
  to a reference frame $x'$ where the particle has momentum
  $\vec{ p }$ we obtain
  \begin{equation}
    \label{QM:09}
    \Psi_{ \vec{ p } }^{ r }( x )
    \to e^{ -i \epsilon_{ r } p_{ \mu } x'^{ \mu } } w^{ r }( \vec{ p } ).
  \end{equation}
  \textbf{Hint.} We need a boost in a direction
  $\vec{ n } = -\vec{ p } / \lvert \vec{ p } \lvert$ with a bost
  parameter which transforms $m c^{ 2 }$ to $E_{ p }$.

\item The parity transformation $P$ transforms coordinate
  $x = \{ x^{ 0 }, x^{ 1 }, x^{ 2 }, x^{ 3 } \}$ to
  $\widetilde{x} = \{ x^{ 0 }, -x^{ 1 }, -x^{ 2 }, -x^{ 3 } \}$. Find
  a form of the transformation $U_{ P }$ such that
  \begin{equation}
    \label{QM:10}
    \widetilde{\Psi}( \widetilde{x} ) = U_{ P } \Psi( x )
  \end{equation}
  which preserves a form of the Dirac equation
  \begin{equation}
    \label{QM:11}
    \left( i \hbar c \gamma^{ \mu } \partial_{ \mu } - m c^{ 2 } \right) \Psi( x ) = 0.
  \end{equation}
  Determine how this transformation changes special solutions
  \begin{equation}
    \label{QM:12}
    \Psi_{ \vec{ p } }^{ r }( x )
    = e^{ -i \epsilon_{ r } p_{ \mu } x^{ \mu } } w^{ r }( \vec{ p } ).
  \end{equation}

\item The charge conjugation transformation $C$ transform a spinor
  $\Psi( x )$ to $\Psi_{ C }( x ) = U_{ C } \Psi^{ * }( x )$. Find a
  form of the transformation $U_{ C }$ which preserves a form of the
  Dirac equation. Determine how this transformation changes special
  solutions
  \begin{equation}
    \label{QM:13}
    \Psi_{ \vec{ p } }^{ r }( x )
    = e^{ -i \epsilon_{ r } p_{ \mu } x^{ \mu } } w^{ r }( \vec{ p } )
  \end{equation}
  \textbf{Note.} Here $*$ means complex conjugation and not Herimitian
  conjugation!

\item The time inversion transformation $T$ transforms coordinates
  $x = \{ x^{ 0 }, x^{ 1 }, x^{ 2 }, x^{ 3 } \}$ to
  $x' = \{ -x^{ 0 }, x^{ 1 }, x^{ 2 }, x^{ 3 } \}$. Find a form of the
  transformation $U_{ T }$ such that
  \begin{equation}
    \label{QM:14}
    \Psi_{ T }( x' ) = U_{ T } \Psi^{ * }( x )
  \end{equation}
  which preserves a form of the Dirac equation
  \begin{equation}
    \label{QM:15}
    \left( i\hbar c \gamma^{ \mu } \partial_{ \mu } - m c^{ 2 } \right) \Psi( x ) = 0.
  \end{equation}
  Determine how this transformation changes special solutions
  \begin{equation}
    \label{QM:16}
    \Psi_{ \vec{ p } }^{ r }( x )
    = e^{ -i\epsilon_{ r } p_{ \mu } x^{ \mu } } w^{ r }( \vec{ p } \, ).
  \end{equation}

\item We define matrices
  \begin{equation}
    \label{QM:17}
    \Lambda_{ \pm }( p )
    = \frac{ \pm\gamma^{ \mu } p_{ \mu } + m c^{ 2 } }{ 2 m c^{ 2 } }.
  \end{equation}
  Show that these matrices are projections in the space of spinors,
  i.e.
  \begin{equation}
    \label{QM:18}
    \Lambda_{ + }^{ \, 2 } = \Lambda_{ + }, \quad
    \Lambda_{ - }^{ \, 2 } = \Lambda_{ - }, \quad
    \Lambda_{ + } \cdot \Lambda_{ - } = 0.
  \end{equation}
  Check the result of $\Lambda_{ \pm } w^{ r }( \vec{ p } \,a )$.

\item Repeat the properties of the Pauli matrices $\sigma_{ i }$,
  $i = 1, 2, 3$
  \begin{equation}
    \label{QM:19}
    \sigma_{ i } \sigma_{ j } = \delta_{ i j } + i \epsilon_{ i j k } \sigma_{ k }, \quad
    \left( \vec{ \sigma } \cdot \vec{ a } \right)
    \left( \vec{ \sigma } \cdot \vec{ b } \, \right)
    =
    \vec{ a } \cdot \vec{ b }
    + i \vec{ \sigma } \cdot \left( \vec{ a } \times \vec{ b } \, \right).
  \end{equation}

\item Start with the Dirac equation in the form:
  \begin{equation}
    \label{QM:20}
    i \hbar \partial \Psi( x )
    =
    \left( -ic\hbar \left( \vec{ \alpha } \cdot \vec{ \nabla } \right)
      + \beta m c^{ 2 } + V( x ) \right) \Psi( x ).
  \end{equation}
  Show that.

  \textbf{a)} This equations can be interpreted as describing
  interaction of the electron with the electromagnetic field
  ($\partial_{ \mu } \to D_{ \mu }$) in the case $\vec{ A }( x ) = 0$,
  $V( x ) = e A_{ 0 }( x )$.

  \textbf{b)} Show that the equation is invariant under
  $\vec{ x } \to -\vec{ x }$ and
  $\Psi( \vec{ x }, t ) \to \beta \Psi( -\vec{ x }, t )$.

  \textbf{c)} Rewrite the equation as an equation for a stationary
  state and show that the Hamiltionia commutes with the rotation
  operators $J_{ i } = L_{ i } + S_{ i }$, where
  \begin{equation}
    \label{QM:21}
    S_{ i } = \frac{ \hbar }{ 2 } \Sigma_{ i }, \quad
    \Sigma_{ i }
    =
    \begin{pmatrix}
      \sigma_{ i } & 0 \\
      0 & \sigma_{ i }
    \end{pmatrix}.
  \end{equation}

  \textbf{d)} Using the operators $2 \times 2$:
  $j_{ i } = L_{ i } + \hbar \sigma_{ i } / 2$ construct the
  eigenstates $\vec{ j }^{ 2 }$, $\vec{ L }^{ 2 }$, $j_{ 3 }$ in the
  form
  \begin{equation}
    \label{QM:22}
    \Omega_{ j l m } =
    \begin{pmatrix}
      \alpha_{ 1 } Y_{ l, \, m - 1/2 }( \theta, \phi ) \\
      \alpha_{ 2 } Y_{ l, \, m + 1/2 }( \theta, \phi )
    \end{pmatrix}
  \end{equation}
  find the eigenvalues $\alpha_{ 1 }$ and $\alpha_{ 2 }$. What is the
  eigenvalue of $j_{ 3 }$? Show that eigenstates with a given $j$ can
  be constructed only from $l$ or $l'$ with $l + l' = 2j$, i.e.
  $l, l' = j \pm \frac{ 1 }{ 2 }$.

  \textbf{e)} Find a form of the Dirac equation in the parametrization
  \begin{equation}
    \label{QM:23}
    \Psi( \vec{ x }, t ) =
    e^{ -\frac{ i }{ \hbar } E t }
    \begin{pmatrix}
      f( r ) \Omega_{ j, l, m } \\
      g( r ) \Omega_{ j, l', m }
    \end{pmatrix}
  \end{equation}
  where $l' = 2j - l$ and determine the set of equations for $f( r )$
  and $g( r )$.

  \textbf{Hint.} Assume a simplified relation (without referring to
  the standard phase factor convention):
  \begin{equation}
    \label{QM:24}
    \frac{ \vec{ \sigma } \cdot \vec{ r } }{ r } \Omega_{ j, l, m }
    = \Omega_{ j, l', m }, \quad
    \frac{ \vec{ \sigma } \cdot \vec{ r } }{ r } \Omega_{ j, l', m }
    = \Omega_{ j, l, m }
  \end{equation}
  where $l + l' = 2 j$.

\item Using the reduced from of the Dirac equation analyze the
  eigenvalue spectrum of the ``hydrogen-like'' atom
  $V( \vec{ x } ) = -Z e^{ 2 } / r$:

  \textbf{a)} Rewrite the Dirac equation as a set of two equations
  satisfied by functions $f( r )$ and $g( r )$ at fixed values of
  quantum numbers $j, m, l, l' = 2j - l$.

  \textbf{b)} Check the asymptotic behavior at $r \to \infty$. What
  restriction should be satisfied by the localized eigenstate? Show
  that in this limit $f( r )$ and $g( r )$ should behave as
  $e^{ -kr }$. Calculate $k$. Check the asymptotics $r \to 0$
  (behavior $\varpropto r^{ \alpha }$).

  \textbf{c)} Find a form of the reduced equation (eliminating the
  asymptotic behavior at $r \to \infty$) for $F( r )$ i~$G( r )$
  (where $f( r ) = F( r ) e^{ -kr }$, similarly $G( r )$).

  \textbf{d)} Determine the recurrence equation for coefficients of
  the power expansion
  \begin{equation}
    \label{QM:24}
    F( r ) = \sum_{ n } f_{ n } r^{ n + \alpha } \quad \textrm{and} \quad
    G( r ) = \sum_{ n } g_{ n } r^{ n + \alpha }
  \end{equation}
  as a two-component matrix equation
  \begin{equation}
    \label{QM:25}
    \begin{pmatrix}
      f_{ n + 1 } \\
      g_{ n + 1 }
    \end{pmatrix}
    =
    A_{ n }
    \begin{pmatrix}
      f_{ n } \\
      g_{ n }
    \end{pmatrix}.
  \end{equation}
  Find eigenvalues $A_{ n }$ and show that this equation permits a
  determination of the spectrum of localized states.

\item Strumień cząstek pada z lewej strony na próg potencjału $E < U$.
  Pokazać, że funkcję falową w obszarze dopuszczalnym klasycznie można
  zapisać jako
  \begin{equation}
    \label{QM:26}
    \psi( x ) = e^{ ikx } + e^{ -ikx + 2 i \delta( x ) }.
  \end{equation}
  Wyjaśnić czemu jest to możliwe? Obliczyć zależność przesunięcia
  fazowego $\delta( E )$ od energii.

\item Przedyskutować rozpraszanie na progu potencjału zależnego od
  czasu \textit{pakietu falowego} (Messiah, rozdział II.3 i III.3).
  Obliczyć opóźnienie kwantowe występujące przy odbiciu i powiązać je
  z przesunięciem fazowym $\delta( E )$.

\item Wykonując poniższą całkę metodą residuów
  \begin{equation}
    \label{QM:27}
    \lim_{ \varepsilon \to 0 } \int_{ -\infty }^{ \infty } \frac{ f( x ) }{ x - x_{ 0 } + i \varepsilon },
  \end{equation}
  udowodnić tożsamość dystrybucyjną
  \begin{equation}
    \label{QM:28}
    \lim_{ \varepsilon \to 0 } \frac{ 1 }{ x - x_{ 0 } + i \varepsilon } =
    P \frac{ 1 }{ x - x_{ 0 } } - i \pi \delta( x - x_{ 0 } ),
  \end{equation}
  gdzie $P( 1 / x )$ oznacza dystrybucję znaną jako wartość główna
  całki.
  \begin{equation}
    \label{QM:29}
    P \frac{ 1 }{ x } \equiv
    \lim_{ \varepsilon \to 0 } \left( \int_{ -\infty }^{ -i \varepsilon } dx \, \frac{ 1 }{ x }
      + \int_{ i \varepsilon }^{ \infty } dx \, \frac{ 1 }{ x } \right).
  \end{equation}
  Wykorzystujący ten związek (i model delty dany przez krzywą
  Lorentza) podać model dla wartości głównej. Przeanalizować jego
  wykres dla kilku wartości $\varepsilon$.

\item Operatory kreacji i anihilacji dla oscylatora harmonicznego
  definiuje się jako
  \begin{equation}
    \label{QM:30}
    a =
    \sqrt{ \frac{ m \omega }{ 2 \hbar } } x
    + \frac{ i }{ \sqrt{ 2 m \omega \hbar } } p, \quad
    a^{ \dagger } =
    \sqrt{ \frac{ m \omega }{ 2 \hbar } } x
    - \frac{ i }{ \sqrt{ 2 m \omega \hbar } } p.
  \end{equation}
  Obliczyć, metodą algebraiczną tzn. wykorzystując tylko reguły
  komutacji operatorów pędu i~położenia, następujące komutatory
  \begin{equation}
    \label{QM:31}
    [ a, a ], \quad [ a^{ \dagger }, a^{ \dagger } ], \quad [ a, a^{ \dagger } ], \quad
    [ a, N ], \quad [ a^{ \dagger }, N ]
  \end{equation}
  gdzie $N = a^{ \dagger } a$ jest operatorem „liczby kwantów”.

\item Wyprowadzić te same reguły komutacji używając reprezentacji
  położeń dla operatorów kreacji i anihilacji. Wsk. Użyć zmiennej
  bezwymiarowej $\zeta = \sqrt{ m \omega / 2 \hbar } x$.

\item Obliczyć wartość oczekiwaną $L_{ z }^{ 2 }$ ($L_{ i }$ to
  składowe operator momentu pędu) w stanie opisany funkcją falową
  $\psi( r, \theta, \psi ) = \sqrt{ 4 / ( 3 \pi ) } ( \sin \phi )^{ 2
  }$, $0 \leq \phi \leq 2\pi$.

\item Pokazać, że relacja $\vec{ L }^{ 2 } = l ( l + 1 )$ może zostać
  otrzymana na podstawie elementarnej wiedzy z rachunku
  prawdopodobieństwa. Założyć, że jedyne możliwe wartości rzutu na
  dowolnie wybraną oś to $m = -l, l + 1, \ldots, l$ i że
  prawdopodobieństwo otrzymania tych wartości są takie same, zaś wybór
  osi jest symetryczny.

\item Obliczyć ślady następujących macierzy: $L_{ i }$,
  $L_{ i } L_{ k }$, $L_{ i } L_{ j } L_{ k }$,
  $L_{ i } L_{ j } L_{ k } L_{ m }$. Gdzie $L_{ i }$ to składowa
  momentu pędu.

\item Operator mnożenia przez kąt biegunowy $\phi$ i operator
  składowej $z$ momentu pędu $L_{ z }$ spełniają kanoniczną relację
  komutacji $[ L_{ z }, \widehat{ \phi } ] = -i\hbar$. Czy zachodzi
  odpowiednia zasada nieoznaczoności Heisenberga dla tej pary?

\item Zbudować jawnie operator $\vec{ s }_{ \vec{ n } }$ odpowiadający
  rzutowi spinu $1 / 2$ na kierunku $\vec{ n }$. Dla stanu o ustalonej
  wartości spinu na kierunek $z$, znaleźć średnią dla
  $\vec{ s }_{ \vec{ n } }$ i prawdopodobieństwo, że wartość rzutu na
  kierunek $\vec{ n }$ będzie wynosić $\pm 1 / 2$.

\item Rozważmy trzy macierze $M_{ x }$, $M_{ y }$, $M_{ z }$ o
  wymiarach 256 na 256 każda. Macierze spełniają relacje komutacji
  $[ M_{ x }, M_{ y } ] = i M_{ z }$ (oraz ich cykliczne permutacje
  względem $x$, $y$, $z$). Macierz $M_{ x }$ ma wartości własne:
  $\pm 2$ (niezdegenerowane), $\pm 3 / 2$ (ośmiokrotnie
  zdegenerowane), $\pm 1$ (28-krotnie zdegenerowane), $\pm 1 / 2$
  (56-krotnie zdegenerowane) oraz 0 (70-krotnie zdegenerowane).
  Obliczyć wartości własne macierzy
  $M^{ 2 } = M_{ x }^{ 2 } + M_{ y }^{ 2 } + M_{ z }^{ 2 }$.

\item Consider special solutions of Klein-Gordon equation
  \begin{equation}
    \label{eq:4}
    ( \hbar^{ 2 } g^{ \mu \nu } \partial_{ \mu } \partial_{ \nu } + m^{ 2 } c^{ 4 } ) \Psi( \vecx, t )
    = 0
  \end{equation}
  in a form
  \begin{equation}
    \label{eq:5}
    \Psi^{ \pm }_{ \vecp }( \vecx, t ) =
    e^{ \mp \frac{ i }{ \hbar } E_{ p } t + \frac{ i }{ \hbar } \vecp \cdot \vecx }, \qquad
    E_{ p } = +\sqrt{ \vecp^{ 2 } c^{ 2 } + m^{ 2 } c^{ 4 } }.
  \end{equation}
  Check the norm of these states, i.e. calculate
  \begin{equation}
    \label{eq:11}
    \int d^{ 3 } x \, i \hbar
    \left( \Psi^{ \pm }_{ \vecq }( \vecx, t )
      ( \partial_{ t } \Psi^{ \pm }_{ \vecp }( \vecx, t ) )
      - ( \partial \Psi^{ \pm }_{ \vecq }( \vecx, t ) ) \Psi^{ \pm }_{ \vecp }( \vecx, t )
    \right)
  \end{equation}

  How this norm transforms under orthochronous Lorentz
  transformations?

  What are the transformation rules for: $d^{ 3 } p$,
  $d^{ 3 } p / ( 2 E_{ p } )$.

\item We introduce interaction to the K-G equation:
  \begin{equation}
    \label{eq:12}
    ( \hbar^{ 2 } g^{ \mu \nu } \partial_{ \mu } \partial_{ \nu } + m^{ 2 } c^{ 4 } + W( \vecx, t ) )
    \Psi( \vecx, t )
    = 0,
  \end{equation}
  where $W( \vecx, t )$ is a given function. Let us assume a
  particular form of this function
  \begin{equation}
    \label{eq:13}
    W( \vecx, t ) = -\frac{ C }{ r }
  \end{equation}
  (which means it is time independent and spherically symmetric, with
  some constant $C$). We are looking for special solutions of this
  equation in a form
  \begin{equation}
    \label{eq:14}
    \Psi_{ E }( \vecx, t ) = e^{ -\frac{ i }{ \hbar } E t } \Phi_{ E }( \vecx ).
  \end{equation}

  Check if there are localized states ($\Phi_{ E }( \vecx )$ vanish in
  spatial infinity). For which $C$? Is energy negative for these
  states?

  Find the energy spectrum of localized states. How are these energies
  related to $m c^{ 2 }$?

  Check if there are restrictions on the admissible values of $C$?

  Check the transformation properties of solutions with respect to
  $\Cbold$, $\Pbold$, $\Tbold$.

  Find the non-relativistic limit of solutions (expand in $c$ around
  $c = \infty$). Give the first 3 therms of the expansion.

  \textbf{Hint.} Use the analogy with the nonrelativistic
  Schr\"{o}dinger equation for the hydrogen atom. \textit{Do not}
  derive the form of the spherical functions
  $Y_{ l m }( \theta, \phi )$, only use their properties.

\item We introduce electromagnetic interactions with an external field
  $A_{ \mu }( \vecx, t )$ defining a \textit{covariant derivative}
  \begin{equation}
    \label{eq:15}
    D_{ \mu } = \partial_{ \mu } + \frac{ i e }{ \hbar c } A_{ \mu },
  \end{equation}
  where $e$ is a \textit{charge} of the field. The modified K-G
  equation has a form
  \begin{equation}
    \label{eq:16}
    ( \hbar^{ 2 } g^{ \mu \nu } D_{ \mu } D_{ \nu } + m^{ 2 } c^{ 2 } ) \Psi( \vecx, t )
    = 0.
  \end{equation}
  Check that this equation is invariant under a local gauge
  transformation
  \begin{align}
    \Psi( \vecx, t )
    &= e^{ \frac{ i e }{ \hbar c } \chi( \vecx, t ) } \Psi'( \vecx, t ), \\
    A_{ \mu }( \vecx, t )
    &= A_{ \mu }'( \vecx, t ) - \partial_{ \mu } \chi( \vecx, t ).
  \end{align}

\item Consider a special case of equation from the one of the previous
  exercises, for which $A_{ \mu }( x )$ in some gauge have a form:
  \begin{equation}
    \label{eq:17}
    A_{ 0 } = -\frac{ Z e }{ r }, \quad
    A_{ i } = 0.
  \end{equation}
  Similarly as in one of previous exercises, find the stationary
  solutions
  \begin{equation}
    \label{eq:18}
    \Psi_{ E }( \vecx, t ) = e^{ -\frac{ i }{ \hbar } E t } \Phi_{ E }( \vecx ).
  \end{equation}
  Find the energy spectrum of these localized solutions.

  Do we have localized states with positive and negative energy? How
  does it depend on $Z$?

  Are there constraints on $Z$?

  Check the transformation of solutions under $\Cbold$, $\Pbold$ and
  $\Tbold$.

  Find the non-relativistic limit of solutions (expand in $c$ round
  $c = \infty$). Give the first 3 terms of the expansion.

\item We assume that the one-particle non-relativistic hamiltonian $H$
  has discrete eigenstates $| n \rangle$
  \begin{equation}
    \label{eq:1}
    H | n \rangle = E_{ n } | n \rangle, \quad
    \langle n | m \rangle = \delta_{ n m }.
  \end{equation}
  A set of $N$ identical, indistinguishable particles of this type is
  described by a hamiltonian $\Hcal$
  \begin{equation}
    \label{eq:2}
    \Hcal = \sum_{ j = 1 }^{ N } H_{ j },
  \end{equation}
  where $H_{ j }$ is a hamiltonian for the particle $j$. The
  eigenstates of $\Hcal$ in the Fock space have a form
  $| N_{ 1 }, N_{ 2 }, \ldots, N_{ k }, \ldots \rangle$, where
  $N_{ k } = 0, 1, 2, \ldots$ is a number of particles in the state
  $k$. The \textit{vacuum} state
  $| 0 \rangle = | 0, 0, \ldots, 0, \ldots \rangle$ is defined as a state without particles
  with a norm $\langle 0 | 0 \rangle = 1$.

  A one-particle state in the eigenstate
  $| n \rangle \equiv | 0, 0, \ldots, 1_{ n }, 0, \ldots \rangle$ can be obtained acting
  with a \textit{creation} operator $a_{ n }^{ \dagger } | 0 \rangle$. Defining a
  conjugate operator of \textit{annihilation} $a_{ n }$ we assume that
  \begin{equation}
    \label{eq:3}
    a_{ n } | 0 \rangle = 0.
  \end{equation}
  Operator of creation and annihilation of one-particle states satisfy
  \begin{align}
    &[ a_{ i }, a_{ j } ] = [ a_{ i }^{ \dagger }, a_{ j }^{ \dagger } ] = 0, \quad
      [ a_{ i }, a_{ j }^{ \dagger } ] = \delta_{ i j }, \quad
      \textrm{for bosons}, \\
    &\{ a_{ i }, a_{ j } \} = \{ a_{ i }^{ \dagger }, a_{ j }^{ \dagger } \} = 0, \quad
      \{ a_{ i }, a_{ j }^{ \dagger } \} = \delta_{ i j }, \quad
      \textrm{for fermions}.
  \end{align}

  How can we construct an arbitrary state
  $| N_{ 1 }, N_{ 2 }, \ldots, N_{ k }, \ldots \rangle$ acting with creation operators
  on a vacuum state (both bosonic and fermionic case)? States should
  be normalized to unity.

  Show that both for bosons and fermions the hamiltonian is
  \begin{equation}
    \label{eq:19}
    \Hcal = \sum_{ k } E_{ k } a_{ k }^{ \dagger } a_{ k }.
  \end{equation}

  Above definitions are for the Schr\"{o}dinger picture, where the states of the system depend on time and satisfy the Schr\"{o}dinger equation. In the Heisenberg picture states are time independent, but operators depend on time. Find the form of creation and annihilation operators in the Heisenberg picture (bosons and fermions).

  Show that
  \begin{equation}
    \label{eq:22}
    \Psi_{ n }( \vecx \, ) =
    \langle 0 | \Phi( \vecx \, ) | 0, 0, \ldots, 0, 1_{ n }, 0, \ldots \rangle
    = \langle \Phi( \vecx ) a_{ n }^{ \dagger } | 0 \rangle.
  \end{equation}

  The operator conjugate to $\Phi( \vecx \, )$ is
  \begin{equation}
    \label{eq:23}
    \Phi^{ \dagger }( \vecx \, ) =
    \sum_{ n } \Psi_{ n }^{ * }( \vecx \, ) a_{ n }^{ \dagger }.
  \end{equation}
  This operator acting on a vacuum state generates a one-particle state $| \vecx \rangle$ localized in the $\vecx$. Show that
  \begin{align}
    \textrm{for bosons} \;
    &[ \Phi( \vecx \, ), \Phi( \vecx' \, ) ] = 0, \quad
      [ \Phi( \vecx \, ), \Phi^{ \dagger }( \vecx' \, ) ]
      = \delta^{ ( 3 ) }( \vecx - \vecx' \, ), \\
    \textrm{for fermions}
    &\{ \Phi( \vecx \, ), \Phi( \vecx' \, ) \} = 0, \quad
    \{ \Phi( \vecx \, ), \Phi^{ \dagger }( \vecx' \, ) \}
    = \delta^{ ( 3 ) }( \vecx - \vecx' \, ).
  \end{align}

  Show that in both cases $\langle \vecx | \vecy \rangle = \delta^{ ( 3 ) }( \vecx - \vecy \, )$.

  Assuming that in a position representation the one-particle hamiltonian has a form
  \begin{equation}
    \label{eq:24}
    H = -\frac{ \hbar^{ 2 } }{ 2m } \Delta_{ x } + V( \vecx \, ).
  \end{equation}
  Show that
  \begin{equation}
    \label{eq:25}
    \Hcal =
    \int d^{ 3 } x \, \Phi^{ \dagger }( \vecx \, ) \,
    \left( -\frac{ \hbar^{ 2 } }{ 2m } \Delta_{ x } + V( \vecx \, ) \right)
    \Phi( \vecx \, ).
  \end{equation}

  Find a form of $\Phi_{ H }( \vecx, t )$ in the Heisenberg picture.
  Show that the operator in the Heisenberg picture $\Phi_{ H }( \vecx, t )$ satisfies the Schr\"{o}dinger equation
  \begin{equation}
    \label{eq:26}
    i \hbar \frac{ \partial }{ \partial t } \Phi_{ H }( \vecx, t )
    =
    \left( -\frac{ \hbar^{ 2 } }{ 2m } \Delta_{ x } + V( \vecx \, ) \right)
    \Phi_{ H }( \vecx, t ).
  \end{equation}

\item Find a form of a two-particle wave function (bosons and fermions)
  \begin{equation}
    \label{eq:27}
    \Psi_{ n_{ 1 }, n_{ 2 } }( \vecx_{ 1 }, \vecx_{ 2 } ) =
    \langle 0 | \Phi( \vecx_{ 1 } \, ) \Phi( \vecx_{ 2 } \, )
    a_{ n_{ 1 } }^{ \dagger } a_{ n_{ 2 } }^{ \dagger } | 0 \rangle.
  \end{equation}
  Check the normalization.

\item Particular case: a system contains \textit{free} particles in a cubic box with the edge $L$ (assuming periodic boundary conditions). Write explicit form of the field operator $\Phi( \vecx \, )$ in this case. Check what happens in the limit $L \to \infty$, when the discrete sum over the eigenstates should be replaced by the integral. How should we define the one-particle creation and annihilation operators and their (anit)commutation rules?

  \textbf{Hint.} We aim at a new normalization of the one-particle states $| \vecp \, \rangle$ of a form
  \begin{equation}
    \label{eq:28}
    \langle \vecp | \vecq \rangle = ( 2\pi \hbar )^{ 3 } \delta^{ ( 3 ) }( \vecp - \vecq ).
  \end{equation}
  Why?

\item The wave function of the one-particle eigenstate $| n \rangle$ is defined by
  \begin{equation}
    \label{eq:20}
    \Psi_{ n }( vecx ) = \langle \vecx \, | n \rangle.
  \end{equation}
  In the Schr\"{o}dinger picture we define a field operator
  \begin{equation}
    \label{eq:21}
    \Phi( \vecx \, ) = \sum_{ n } \Psi_{ n }( \vecx \, ) a_{ n }.
  \end{equation}

\item Repeat the derivation of the Euler-Lagrange equations. The
  action $S$ is given by
  \begin{equation}
    \label{eq:QM32}
    S = \int\limits_{ t_{ 1 } }^{ t_{ 2 } } dt \, L
  \end{equation}
  where the Lagrangian $L$
  \begin{equation}
    \label{eq:QM33}
    L =
    \int d^{ 3 } x \, \Lcal\left( \Psi_{ \sigma }( \vecx, t ),
      \frac{ \partial \Psi_{ \sigma }( \vecx, t ) }{ \partial x^{ \mu } } \right), \quad
    \mu = 0, 1, 2, 3.
  \end{equation}
  $\Lcal$ is the Lagrangian \textit{density} and numbers the
  \textit{fields}.

  Show that from the variational principle $\delta S = 0$ under conditions
  $\delta \Psi_{ \sigma }( \vecx, t_{ 1 } ) = \delta \Psi_{ \sigma }(
  \vecx, t_{ 2 } ) = 0$ leads to
  \begin{equation}
    \label{eq:QM34}
    \partial_{ \mu } \frac{ \partial \Lcal }{ \partial \big( \partial_{ \mu } \Psi_{ \sigma }( \vecx, t ) \big) }
    - \frac{ \partial \Lcal }{ \partial \Psi_{ \sigma }( \vecx, t ) }
    = 0.
  \end{equation}
  What are these equations if $x^{ 0 }$ is expressed by time $t$ and
  spatial coordinates $\vecx$? We assume that fields
  $\Psi_{ \sigma }( \vecx, t )$ vanish sufficiently fast for all $t$
  when $\vecx \to \infty$.

\item Show that if a Lagrangian density has a form
  \begin{equation}
    \label{eq:QM35}
    \Lcal =
    \frac{ \hbar^{ 2 } }{ 2m } \big( \partial_{ i } \Psi^{ * }( \vecx, t ) \big)
    \big( \partial_{ i } \Psi( \vecx, t ) \big)
    - \Psi^{ * }( \vecx, t ) V( \vecx ) \Psi( \vecx, t )
    + \frac{ i \hbar }{ 2 } \left( \Psi^{ * } \frac{ \partial \Psi }{ \partial t }
      - \frac{ \partial \Psi^{ * } }{ \partial t } \Psi \right)
  \end{equation}
  and assuming fields $\Psi$ and $\Psi^{ * }$ are independent, the E-L
  equations are identical to the one-particle Schr\"{o}dinger equation
  (both for $\Psi$ and $\Psi^{ * }$).

  Check that the same equations are obtained if the term with time
  derivatives is replaced by
  \begin{equation}
    \label{eq:QM36}
    \textrm{a)} i \hbar \left( \Psi^{ * } \frac{ \partial \Psi }{ \partial t } \right), \qquad
    \textrm{b)} -i \hbar \left( \frac{ \partial \Psi^{ * } }{ \partial t } \Psi \right).
  \end{equation}
  Why?

\item Assume that the Lagrangian density has a form b) from previous
  exercises. Perform the canonical quantization in this case: find the
  canonical momentum conjugate to $\Psi^{ * }( \vecx, t )$, determine
  the canonical relations, find the Hamiltonian and show that result
  is the same as for the case presented on the lecture.

\item Assume that the Lagrangian density has a form
  \begin{equation}
    \label{eq:QM37}
    \Lcal =
    \hbar^{ 2 } c^{ 2 } g^{ \mu \nu } \big( \partial_{ \mu } \phi^{ * }( x ) \big)
    \big( \partial_{ \nu } \phi( x ) \big)
    - m^{ 2 } c^{ 4 } \phi^{ * }( x ) \phi( x )
  \end{equation}
  with $x = \{ x^{ 0 }, x^{ 1 }, x^{ 2 }, x^{ 3 } \}$,
  $x^{ 0 } = c t$.

  Determine the E-L equations assuming that fields $\phi^{ * }$ and
  $\phi$ are independent.

  Rewrite the Lagrangian explicitly in time and space derivatives and
  fields. Find canoncial momenta $\pi( x )$ and $\pi^{ * }( x )$
  conjugate respectively to $\phi( x )$ and $\phi^{ * }( x )$.

  Derive a form of the Hamiltonian.

  \textbf{Hint.} Use the concept of a variational derivative
  $\delta L / \delta \big( \partial_{ t } \phi( x ) \big)$.

\item Assume that the Lagrangian density has a form
  \begin{equation}
    \label{eq:QM38}
    \Lcal =
    \frac{ \hbar^{ 2 } c^{ 2 } }{ 2 } g^{ \mu \nu }
    \big( \partial_{ \mu } \phi( x ) \big) \big( \partial_{ \nu } \phi( x ) \big)
    - \frac{ m^{ 2 } c^{ 4 } }{ 2 } \phi( x )^{ 2 }
  \end{equation}
  with $x = \{ x^{ 0 }, x^{ 1 }, x^{ 2 }, x^{ 3 } \}$, $x^{ 0 } = c t$
  and $\phi( x )$ a real scalar field.

  Determine the E-L equation.

  Rewrite the Lagrangian explicitly in time and space derivatives of
  fields. Assume that field $\phi( x )$ is periodic in a spatial box
  $L \times L \times L$. Decompose field $\phi( x )$ as a combination
  of momentum eigenstates in a periodic box
  \begin{equation}
    \label{eq:QM39}
    \phi( x ) =
    \sum_{ \vecn } c_{ \vecn }( t ) \psi_{ \vecn }( \vecx ), \qquad
    \psi_{ \vecn }( \vecx ) =
    \frac{ 1 }{ L^{ 3 / 2 } } e^{ i 2\pi \vecn \cdot \vecx / L }.
  \end{equation}
  What relations between coefficients $c_{ \vecn }( t )$ follow from
  the fact that field $\phi( x )$ is real?

  Derive a form of the Hamiltonian.

  Find independent degrees of freedom and quantize the system.

\end{enumerate}
% ##################










% ######################################
\section{Fizyka fazy skondensowanej, problemy do zrobienia}

% \vspace{\spaceTwo}

% ######################################



% ##################
\begin{enumerate}

\item Wyznaczyć relacje dyspersji dla jednowymiarowej sieci złożonej z
  identycznych atomów przy założeniu, że stałe siłowe opisujące
  oddziaływanie par atomów są na przemian równe $C_{ 1 }$ i~$C_{ 2 }$
  ($C_{ 1 } \neq C_{ 2 }$).

\item Oblicz $c_{ V }( T )$ przyjmujące
  $D( \omega ) = 3 N \delta( \omega - \omega_{ E } )$, gdzie $N$ to
  liczba atomów w~krysztale. Dla $T \to \infty$ pokaż, że
  $c_{ V }( T )$ spełnia prawo Dulonga-Petita.

\item Pokaż, że dla $N$ atomów w sześcianie o~bloku $L$, gęstość stanów w~przestrzeni pędów jest dana przez $\rho( k ) = \left( \frac{ L }{ 2\pi } \right)^{ 3 }$. \\
  Używając zależności dyspersyjnej $\omega = \nu k$, pokaż że
  $D( \omega ) = \frac{ 3 \omega^{ 2 } L^{ 3 } }{ 2\pi^{ 2 } \nu^{ 3 }
  }$ dla
  $\omega < \omega_{ D } = ( 6 \pi^{ 2 } N )^{ \frac{ 1 }{ 3 } }
  \frac{ V }{ L }$. Oblicz $c_{ V }( T )$ i pokaż, że dla
  $T \searrow 0$, $c_{ V }( T ) \propto T^{ 3 }$.
\item Obliczyć odległości między sąsiadującymi jonami w sieci CsCl,
  stała sieci: $a = 4.11$ \AA{} (albo $a = 3.55$ \AA), NaCl,
  $a = 5.63$ \AA{} albo $a = 2.82$ \AA{} oraz KBr o strukturze NaCl,
  $a = 6.59$ \AA{} albo $a = 3.29$ \AA{} na podstawie danych struktury
  krystalicznych i~porównać je z~odległościami wyznaczonymi przy
  pomocy odpowiednich promieni standardowych jonów oraz poprawek
  zależnych od liczby koordynacyjnej (zobacz jakieś tabele).

\item Podobnie jak w poprzednim zadaniu, porównać odległość między
  jonami dla związków o~strukturze ZnS (blendy cynkowej): CuF, stała
  sieci $a = 4.26$ \AA, ZnS, stała sieci $a = 5.41$ \AA{} i~InSb,
  stała sieci $a = 6.46$ \AA, biorąc pod uwagę promienie atomów przy
  tetraedrycznym wiązaniu kowalencyjnym.

\item Używając potencjału Lennarda-Jonesa, oblicz stosunek energii
  wiązania kryształu gazu szlachetnego krystalizującego w~strukturach
  \textbf{bcc} i~\textbf{fcc}.

\item Oszacować wartość stałej Madelunga dla struktury NaCl na trzy
  sposoby.

  \begin{itemize}
  \item[a)] Stosując dodawanie energii oddziaływania kolejnych
    ładunków całkowitych, wyliczyć kilka pierwszych wyrazów szeregu
    wprost z~definicji.

  \item[b)] Wyliczyć stałą Madelunga numerycznie.

  \item[c)] Stosując metodę Evjena (ułamkowych ładunków) dla sześcianu
    zawierającego 26 ładunków.

  \end{itemize}

\item Znaleźć graficznie pierwsze 4 strefy Brillouina dla sieci
  płaskiej utworzonej z~kwadratów o~boku~$a$. Sprawdzić, że wszystkie
  strefy Brillouina mogą być sprowadzone do pierwszej strefy tzn.
  że~powierzchnia ich jest identyczna.

\item \textbf{Metoda Laueego.} W~metodzie Laueego bada się ugięcie
  polichromatycznej wiązki promieniowania X na płaszczyznę
  nieruchomego monokryształu. W~trakcie pomiaru kąt podania promieni X
  na kryształ jest wielkością stałą. Rozważ wiązkę rentgenowską
  o~długości $\lambda$ z~przedziału, określonego frakcjami stałej $a$ sieci
  regularnej, $\frac{ a }{ 3 } \leq \lambda \leq \frac{ 2a }{ 3 }$ i padającą na
  tę sieć wzdłuż kierunku $[-100]$. Korzystając z~konstrukcji Ewalda
  znajdź płaszczyzny typu $(hk0)$ od których obserwuje~się refleksy.

\item W kamerze obrotu pojedynczego kryształu o~średnicy
  $2R = 57.3 \, \si{mm}$ jest umieszczony monokryształ obracający~się
  wokół osi $a$ (patrz rysunek poniżej). Na monokryształ pada wiązka
  promieniowania o~długości fali $\lambda = 1.5418 \, \si{\r{A}}$. Otrzymano
  warstwice zerową i~cztery warstwice wyższego rzędu. Odstępy między
  symetrycznymi warstwicami wynoszą odpowiednio:
  $2 h_{ 1 } = 10.2 \, \si{mm}$, $2 h_{ 2 } = 21.6 \, \si{mm}$,
  $2 h_{ 3 } = 35.3 \, \si{mm}$, $2 h_{ 4 } = 56.6 \, \si{mm}$. Oblicz
  stałą sieciową $a_{ 0 }$.

  % \item

  % \item

  % \item

  % \item

  % \item

  % \item

  % \item

  % \item

  % \item

  % \item

\end{enumerate}
% ##################










% ######################################
\section{QFT, problemy do zrobienia}

% \vspace{\spaceTwo}

% ######################################


% ##################
\begin{enumerate}
\item Dla zespolonego pola skalarnego (postać operatora tego pola
  zawiera poprzedni zestaw zadań) proszę wyznaczyć (w~postaci
  \textit{czterowymiarowej} całki Fourierowskiej) postać propagatora
  \begin{equation}
    \label{eq:6}
    i \Delta_{ F }( x - y )
    =
    \left\langle 0 \left| T\left( \widehat{\phi}( x ) \widehat{\phi}^{ \dagger }( x )
        \right) \right| 0 \right\rangle,
  \end{equation}
  gdzie $T$ jest operatorem uporządkowania chronologicznego.
  \begin{equation}
    \label{eq:7}
    T\left( \widehat{\phi}( x^{ 0 }, \vecxbold )
      \widehat{\phi}^{ \dagger }( y^{ 0 }, \vecybold ) \right)
    =
    \widehat{\phi}( x^{ 0 }, \vecxbold ) \widehat{\phi}^{ \dagger }( y^{ 0 }, \vecybold)
    \theta( x^{ 0 } - y^{ 0 } )
    +
    \widehat{\phi}^{ \dagger }( y^{ 0 }, \vecybold ) \widehat{\phi}( x^{ 0 }, \vecxbold )
    \theta( y^{ 0 } - x^{ 0 } )
  \end{equation}

\item Prosty model (bezspinowego) pola nukleonów oddziałującego
  z~polem skalarnym pionów opisany jest hamiltonianem
  \begin{align}
    H &= H_{ 0 } + H_{ I }, \\
    H_{ 0 }
      &=
        m_{ 0 } \int d^{ 3 }\vecpbold \, \psi^{ \dagger }( \vecpbold )
        \psi( \vecpbold )
        + \int d^{ 3 } \veckbold \, \omega( \veckbold ) a^{ \dagger }( \veckbold )
        a( \veckbold ),
        \quad
        \omega( \veckbold ) = \sqrt{ \veckbold^{ 2 } + \mu^{ 2 } }, \\
    H_{ I }
      &=
        \frac{ \lambda }{ \sqrt{ 2\pi }^{ 3 } }
        \int d^{ 3 } \vecpbold
        \int \frac{ d^{ 3 } \veckbold }{ \sqrt{ 2\omega( \veckbold ) } }
        f( \veckbold^{ 2 } ) \psi^{ \dagger }( \vecpbold + \veckbold )
        \psi( \vecpbold )
        \left( a( \veckbold ) + a^{ \dagger }( \veckbold ) \right),
  \end{align}
  gdzie $f( \veckbold^{ 2 } )$ jest funkcją opisującą strukturę
  elektronu. Zakładamy, że~$f( \veckbold^{ 2 } )$ dostatecznie szybko
  znika dla duży wartości $\veckbold^{ 2 }$. Operatory
  $a( \veckbold )$, $a^{ \dagger }( \veckbold )$, $\psi( \vecpbold )$,
  $\psi^{ \dagger }( \vecpbold )$ spełniają reguły komutacji i~antykomutacji
  \begin{align}
    \left[ \psi( \vecpbold ), \psi^{ \dagger }( \vecpbold ) \right]_{ + }
    &= \delta^{ ( 3 ) }( \vecpbold - \vecpbold' ), \\
    \left[ \psi( \vecpbold ), \psi( \vecpbold' ) \right]_{ + }
    &=
      \left[ \psi^{ \dagger }( \vecpbold ), \psi^{ \dagger }( \vecpbold' ) \right]_{ + }
      = 0, \\
    \left[ a( \veckbold ), a^{ \dagger }( \veckbold' ) \right]
    &= \delta^{ ( 3 ) }( \veckbold - \veckbold' ), \\
    \left[ a( \veckbold), a( \veckbold' ) \right]
    &=
      \left[ a^{ \dagger }( \veckbold ), a^{ \dagger }( \veckbold' ) \right] = 0, \\
    \left[ \psi( \vecpbold ), a( \veckbold ) \right]
    = \left[ \psi( \vecpbold ), a^{ \dagger }( \veckbold ) \right]
    &=
      \left[ \psi^{ \dagger }( \vecpbold ), a( \veckbold ) \right]
      = \left[ \psi^{ \dagger }( \vecpbold ), a^{ \dagger }( \veckbold ) \right].
  \end{align}

  \textbf{a)} Proszę pokazać, że~pęd oraz liczba nukleonów są w~tym
  modelu wielkościami zachowanymi, t.j. że~operator
  \begin{equation}
    \vecPbold
    =
    \int d^{ 3 } \vecpbold \, \vecpbold \, \psi^{ \dagger }( \vecpbold )
    \psi( \vecpbold )
    + \int d^{ 3 } \veckbold \, \veckbold \, a^{ \dagger }( \veckbold )
    a( \veckbold ), \quad
    N_{ \psi } = \int d^{ 3 } \vecpbold \, \psi^{ \dagger }( \vecpbold ) \psi( \vecpbold ),
  \end{equation}
  komutują z~\textit{pełnym} hamiltonianem $H$.

  \textbf{b)} Proszę pokazać, że~stan perturbacyjny próżni
  $| 0 \rangle$ zdefiniowany poprze
  \begin{equation}
    \label{eq:8}
    \psi( \vecpbold ) | 0 \rangle = a( \veckbold ) | 0 \rangle = 0, \quad
    \forall \, \vecpbold, \veckbold
  \end{equation}
  oraz stan jednomezonowy
  \begin{equation}
    \label{eq:9}
    | \veckbold \rangle = a^{ \dagger }( \veckbold ) | 0 \rangle
  \end{equation}
  są stanami własnymi $H$.

  \textbf{c)} Proszę pokazać, że~stanem własnym hamiltonianu $H$,
  będącym równocześnie stanem własnym operatorem $N_{ \psi }$
  z~wartością własną 1, czyli stanem własnym pełnego hamiltonianu
  zawierającego jeden nukleon) jest
  \begin{equation}
    \label{eq:10}
    | \Psi^{ ( 1 ) }( \vecpbold ) \rangle
    =
    \frac{ \sqrt{ Z } }{ ( 2\pi )^{ 3 } }
    \int d^{ 3 } \vecpbold \int d^{ 3 } \vecxbold
    e^{ i ( \vecqbold - \vecpbold ) \cdot \vecxbold }
    \exp \left\{ \int \frac{ d^{ 3 } \veckbold }{ 2 ( 2\pi )^{ 3 }
        \omega^{ 3 }( \veckbold ) } f( \veckbold^{ 2 } )
      e^{ i \veckbold \cdot \vecxbold } a^{ \dagger }( \veckbold ) \right\}
    \psi^{ \dagger }( \vecqbold ) | 0 \rangle.
  \end{equation}

\item Gaussian integrals.

  \textbf{a)} Let $D_{ nm }$ be a symmetric positive-definite square
  matrix. Show that
  \begin{equation}
    \label{eq:32}
    \int \prod_{ n } d\phi_{ n } \, \exp\left( -\sum_{ n,\, m } \phi_{ n } D_{ nm }
      \phi_{ m } \right)
    = \pi^{ n/2 } ( \det D )^{ -1/2 }.
  \end{equation}

  \textbf{b)} Show that
  \begin{equation}
    \label{eq:33}
    \det{ }^{ -1/2 } ( 2 \pi D ) \int \prod_{ n } d\phi_{ n } \,
    \exp\left( -\frac{ 1 }{ 2 } \sum_{ n,\, m } \phi_{ n } D_{ nm }^{ -1 }
      \phi_{ m } + i \sum_{ n } J^{ n } \phi_{ n } \right)
    =
    \exp\left( -\frac{ 1 }{ 2 } \sum_{ n,\, m } J^{ n } D_{ nm } J^{ m } \right)
  \end{equation}
  and verify that, up to the determinant, the result is exactly what
  you would expect from the method of the steepest descent (evaluate
  the integrated at the extremum of the argument inside the
  exponential).

\item From the Hamiltonian to the Lagrangian path integral. Assume
  that the Hamiltonian is quadratic in $\vec{ \pi }$, and that the
  coefficients of such quadratic terms do not depend on
  $\vec{ \phi }$. Show that
  \begin{equation}
    \label{eq:34}
    \int \Dcal \phi \Dcal \pi \, \exp\left( i \int dt \left( ( \vec{ \pi }
        \cdot \dot{ \vec{ \phi } } ) - H( \vec{ \phi }, \vec{ \pi }, t )
      \right) \right)
    =
    \Ncal \int \Dcal \phi \, \exp\left( L( \vec{ \phi }, \dot{ \vec{ \phi } }, t )
    \right)
  \end{equation}
  where $\Ncal$ is a constant.

\item Let us consider the one-dimensional harmonic oscillator
  interacting with an external source (forced oscillator):
  \begin{equation}
    \label{eq:35}
    L =
    \frac{ m \dot{ x }^{ 2 } }{ 2 } - \frac{ k^{ 2 } x^{ 2 } }{ 2 } + J x
  \end{equation}
  where $J( t )$ is a time dependent external source (e.g., an
  electric field if the oscillator is supposed to carry an electric
  charge). Evaluate the path integral for such a system using the
  saddle point approximation under assumption that
  $J( 0 ) = J( T ) = 0$.

\item A formal substitution $\tau = it$ defines the Euclidean real time
  $\tau$. One could imagine this transformation as the analytic
  continuation from the Minkowski space with coordinates $( x, t )$ to
  the Euclidean space.

  \textbf{a)} Write the corresponding Euclidean action of the forced
  harmonic oscillator which arises as a results of such an analytic
  continuation. What becomes of the corresponding Schr\"{o}dinger
  equation? Discuss the properties of the corresponding Green
  functions in the Euclidean space.

  \textbf{b)} The group of the space-time symmetry of
  $( 3 + 1 )$-dimensional Minkowski space is the group of hyperbolic
  rotations $\SO( 3, 1 )$. What is the group of symmetry of Euclidean
  space?

\item Let us consider the non-relativistic Hamiltonian operator in
  one-dimensional space
  $H = -\frac{ \hbar^{ 2 } }{ 2m } \frac{ d^{ 2 } }{ dx^{ 2 } } + V( x )$
  where the double-well potential is
  \begin{equation}
    \label{eq:36}
    V( x ) = \lambda ( x^{ 2 } - a^{ 2 } )^{ 2 }
  \end{equation}
  and $\lambda$ and $a$ are some constants. The operator of spacial parity
  $P$ which commutes with the Hamiltonian has its eigenfunctions
  $\psi( x )$. It action is defined as
  $P : \psi( x ) = \pm\psi( -x )$. What are the corresponding eigenstates of
  the Hamiltonian.

  % \item

  % \item

  % \item

  % \item

  % \item

  % \item

  % \item

  % \item

  % \item

  % \item

  % \item

  % \item

  % \item

  % \item

  % \item

  % \item

  % \item

  % \item

  % \item

  % \item

  % \item

  % \item

  % \item

  % \item

  % \item

  % \item

  % \item

  % \item

  % \item

  % \item

  % \item

  % \item

  % \item

  % \item

  % \item

  % \item

  % \item

  % \item

  % \item

  % \item

  % \item

  % \item

  % \item

  % \item

  % \item

  % \item

  % \item

  % \item

  % \item

  % \item

  % \item

  % \item

  % \item

  % \item

  % \item

  % \item

  % \item

  % \item

  % \item

  % \item

  % \item

  % \item

  % \item

  % \item

  % \item

  % \item

  % \item

  % \item

  % \item

  % \item

  % \item

  % \item

  % \item

  % \item

  % \item

  % \item

  % \item

  % \item

  % \item

  % \item

  % \item

  % \item

  % \item

  % \item

  % \item

  % \item

  % \item

  % \item

  % \item

  % \item

  % \item

  % \item

  % \item

  % \item

  % \item

  % \item

  % \item

  % \item

  % \item

  % \item

  % \item

  % \item

  % \item

  % \item

  % \item

  % \item

  % \item

  % \item

  % \item

  % \item

  % \item

  % \item

  % \item

  % \item

  % \item

  % \item

  % \item

  % \item

  % \item

  % \item

  % \item

  % \item

  % \item

  % \item

  % \item

  % \item

  % \item

  % \item

  % \item

  % \item

  % \item

  % \item

  % \item

  % \item

  % \item

  % \item

  % \item

  % \item

  % \item

  % \item

  % \item

  % \item

  % \item

  % \item

  % \item

  % \item

  % \item

  % \item

  % \item

  % \item

  % \item

  % \item

  % \item

  % \item

  % \item

  % \item

  % \item

  % \item

  % \item

  % \item

  % \item

  % \item

  % \item

  % \item

  % \item

  % \item

  % \item

  % \item

  % \item

  % \item

  % \item

  % \item

  % \item

  % \item

  % \item

  % \item

  % \item

  % \item

  % \item

  % \item

  % \item

  % \item

  % \item

  % \item

  % \item

  % \item

  % \item

  % \item

  % \item

  % \item

  % \item

  % \item

  % \item

  % \item

  % \item

  % \item

  % \item

  % \item

  % \item


  % \item

  % \item

  % \item

  % \item

  % \item

  % \item

  % \item

  % \item

  % \item

  % \item

  % \item

  % \item

  % \item

  % \item

  % \item

  % \item

  % \item

  % \item

  % \item

  % \item

  % \item

  % \item

  % \item

  % \item

  % \item

  % \item

  % \item

  % \item

  % \item

  % \item

  % \item

  % \item

  % \item

  % \item

  % \item

  % \item

  % \item

  % \item

  % \item

  % \item

  % \item

  % \item

  % \item

  % \item

  % \item

  % \item

  % \item

  % \item

  % \item

  % \item

  % \item

  % \item

  % \item

  % \item

  % \item

  % \item

  % \item

  % \item

  % \item

  % \item

  % \item

  % \item

  % \item

  % \item

  % \item

  % \item

  % \item

  % \item

  % \item

  % \item

  % \item

  % \item

  % \item

  % \item

  % \item

  % \item

  % \item

  % \item

  % \item

  % \item

  % \item

  % \item

  % \item

  % \item

  % \item

  % \item

  % \item

  % \item

  % \item

  % \item

  % \item

  % \item

  % \item

  % \item

  % \item

  % \item

  % \item

  % \item

  % \item

  % \item

  % \item

  % \item

  % \item

  % \item

  % \item

  % \item

  % \item

  % \item

  % \item

  % \item

  % \item

  % \item

  % \item

  % \item

  % \item

  % \item

  % \item

  % \item

  % \item

  % \item

  % \item

  % \item

  % \item

  % \item

  % \item

  % \item

  % \item

  % \item

  % \item

  % \item

  % \item

  % \item

  % \item

  % \item

  % \item

  % \item

  % \item

  % \item

  % \item

  % \item

  % \item

  % \item

  % \item

  % \item

  % \item

  % \item

  % \item

  % \item

  % \item

  % \item

  % \item

  % \item

  % \item

  % \item

  % \item

  % \item

  % \item

  % \item

  % \item

  % \item

  % \item

  % \item

  % \item

  % \item

  % \item

  % \item

  % \item

  % \item

  % \item

  % \item

  % \item

  % \item

  % \item

  % \item

  % \item

  % \item

  % \item

  % \item

  % \item

  % \item

  % \item

  % \item

  % \item

  % \item

  % \item

  % \item

  % \item

  % \item

  % \item

  % \item

  % \item

  % \item

  % \item

  % \item

  % \item

  % \item

  % \item

  % \item

  % \item

  % \item

  % \item

  % \item

  % \item

  % \item

  % \item

  % \item

  % \item

  % \item

  % \item

  % \item

  % \item

  % \item

  % \item

  % \item

  % \item

  % \item

  % \item

  % \item

  % \item

  % \item

  % \item

  % \item

  % \item

  % \item

  % \item

  % \item

  % \item

  % \item

  % \item

  % \item

  % \item

  % \item

  % \item

  % \item

  % \item

  % \item

  % \item

  % \item

  % \item

  % \item

  % \item

  % \item

  % \item

  % \item

  % \item

  % \item

  % \item

  % \item

  % \item

  % \item

  % \item

  % \item

  % \item

  % \item

  % \item

  % \item

  % \item

  % \item

  % \item

  % \item

  % \item

  % \item

  % \item

  % \item

  % \item

  % \item

  % \item

  % \item

  % \item

  % \item

  % \item

  % \item

  % \item

  % \item

  % \item

  % \item

  % \item

  % \item

  % \item

  % \item

  % \item

  % \item

  % \item

  % \item

  % \item

  % \item

  % \item

  % \item

  % \item

  % \item

  % \item

  % \item

  % \item

  % \item

  % \item

  % \item

  % \item

  % \item

  % \item

  % \item

  % \item

  % \item

  % \item

  % \item

  % \item

  % \item

  % \item

  % \item

  % \item

  % \item

  % \item

  % \item

  % \item

  % \item












































\end{enumerate}
% ##################










% ######################################
\section{Teoria grup i fizyka, problem do zrobienia}

% \vspace{\spaceTwo}

% ######################################


% ##################
\begin{enumerate}

\item Rozważmy reprezentację grupy $\SO( 3 )$ przez macierze
  $\SU( 2 )$ w przestrzeni spinorów. W tej reprezentacji definiujemy
  algebrę operatorów liniowych $\operatorA = A^{ i } \sigma_{ i }$, gdzie
  $A^{ i }$ to zespolone współczynniki, $\sigma_{ i }$, $i = 1, 2, 3$ to
  macierze Pauliego (stosujemy konwencję sumacyjną Einsteina).
  Transformacje grupy $\SU( 2 )$ parametryzujemy wzorem
  $U( \vecn ) = \exp( i \vec{ \sigma } \vecn )$. Proszę obliczyć jak
  transformuje się operator $\operatorA$ przy obrotach. Czy operator
  $\operatorA$ jest operatorem tensorowym?

\item Analogicznie do poprzedniego zadania, definiujemy operator
  $\operatorC = A^{ i } B^{ j } \sigma_{ i } \sigma_{ j }$. Proszę sprawdzić,
  jak transformują się składowe tego operatora przy operatorach.
  Proszę pokazać, że algebra operatorów $\operatorC$ tworzy
  reprezentację redukowalną, przywiedlną grupy $\SO( 3 )$. Ze
  składowych $A^{ i } B^{ j }$ proszę zbudować operatory tensorowe
  transformują się w nieredukowalnych reprezentacjach $\SO( 3 )$.

  \textbf{Wskazówka.} Przykładem reprezentacji nieredukowalnych
  reprezentacji grupy obrotów o całkowitych wartościach krętu $j$ są
  funkcje kuliste.

\item Obroty $R( \alpha, \beta, \gamma )$ w przestrzeni trójwymiarowej można
  sparametryzować za pomocą kątów Eulera $0 < \alpha < 2\pi$,
  $0 < \beta < \pi$, $0 < \gamma < 2\pi$ w następujący sposób:
  $R( \alpha, \beta, \gamma ) = R_{ z_{ 2 } }( \gamma ) R_{ y_{ 1 } }( \beta ) R_{ z }( \alpha )$,
  gdzie $y_{ 1 }$ jest osią otrzymaną z osi $y$ po obrocie
  $R_{ z }( \alpha )$ względem osi $z$, a $z_{ 2 }$ osią, w którą
  przechodzi oś $z$ po dwóch obrotach, względem osi $z$ i $y_{ 1 }$.
  Proszę wyliczyć jak transformują się składowe wektora $\vecr$ przy
  obrotach $R( \alpha, \beta, \gamma )$. Czy taka transformacja ma wektory
  niezmiennicze.

\item Wiadomo z kursu mechaniki kwantowej, że operatory momentu pędu $\vecJ$ są proporcjonalne do generatorów grupy obrotów, czyli operatory obrotów skończonych można zapisać jako $\exp( -i \vec{ \theta } \vecJ \, )$ (przyjmujemy $\hbar = 1$), gdzie składowe $\vec{ \theta }$ parametryzują dowolny obrót. Pamiętając o tym, że w reprezentacji spinowej, dla $J = \frac{ 1 }{ 2 }$, $\vecJ = \vec{ \sigma } / 2$, proszę wyliczyć macierze operatorów obrotów $R( \alpha, \beta, \gamma )$ w tej reprezentacji.

\item Proszę, korzystając ze znanych z mechaniki kwantowej wyników dla wartości elementów macierzowych operatorów $\vecJ$ w bazie $| j, m \rangle$ stanów własnych $\operatorJ^{ \; 2 }$ i $\operatorJ_{ z }^{ \; 2 }$, wyliczy macierze reprezentujące obroty skończone $R( \alpha, \beta, \gamma )$ w bazie $| j, m \rangle$, dla $j = 1$.

  \textbf{Wskazówka.} Proszę pokazać, że dla $j = 1$, potęgę operatora $( \vec{ \theta } \vecJ \, )^{ n }$ dla $n > 2$ można wyrazić przez kombinację liniową wyrazów z $n = 0, 1$ i $2$.

\item

\end{enumerate}
% ##################


% % ######################################
% \newpage
% \section{Zaczęte i~nieskończone}

% \vspace{\spaceTwo}
% % ######################################














% % ######################################
% \newpage
% \section{Articles}

% \vspace{\spaceTwo}
% % ######################################



% \begin{enumerate}

% \item



% \item


% \item



% \item

% \item

% \item

% \item

% \item

% \item

% \item

% \item

% \item

% \item

% \item

% \item

% \item

% \item

% \item

% \item

% \item

% \item

% \item

% \item

% \item

% \item

% \item

% \item

% \item

% \item

% \item

% \item

% \item

% \item

% \item

% \item


% \item

% \item

% \item

% \item

% \item

% \item

% \item

% \item

% \item

% \item

% \item

% \item

% \item

% \item

% \item

% \item

% \item

% \item

% \item

% \item

% \item

% \item

% \item

% \item

% \item

% \item

% \item

% \item

% \item

% \item

% \item

% \item

% \item

% \item

% \item

% \item

% \item

% \item

% \item

% \item

% \item

% \item

% \item


































































































































































































































































% \end{enumerate}










% ############################

% Koniec dokumentu
\end{document}
