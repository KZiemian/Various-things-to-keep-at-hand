% \somebeamercommand[<optiona arguments>]{<first argument>}{<second argument>}

% \begin{somebeamerenviroment}[<optiona argument>]{<first argument>}
%   <environment contents>
% \end{somebeamerenvironment}

% \setbeamertemplate{some beamer element}{your definition for this template}

% \setbeamertemplate{some beamer element}[rose]
% \setbeamertemplate{some beamer element}[shamrock]{3}
% \setbeamercolor{some beamer element}{fg=red}
% \setbeamercolor{some beamer element}{bg=black}

% \setbeamerfont{some beamer element}{size=\large}

% \setbeamercolor{fg=red, bg=black}

% \setbeamerfont{some beamer element}{series=\bfseries, shape=\itshape, family=\sffamily}

\documentclass{beamer}

\mode<presentation>
% \usetheme{Warsaw}
\usetheme{Frankfurt}

\usecolortheme{seahorse}
\usecolortheme{rose}

\usefonttheme[onlylarge]{structuresmallcapsserif}

\usefonttheme[onlysmall]{structurebold}


% \setbeamerfont{title}{shape=\itshape}
% \setbeamerfont{title}{family=\rmfamily}
\setbeamerfont{title}{shape=\itshape,family=\rmfamily}

% \setbeamercolor{title}{fg=red!80!black}
\setbeamercolor{title}{fg=red!80!black,bg=red!20!white}





\title{Theres Is No Largest Prime Number}

\author[Euclid]{Euclid of Alexandria \\
  \texttt{euclid@alexandria.edu}}

\date[ISPN '80]{27th International Symposium of Prime Numbers}


\begin{document}

\begin{frame}

  \titlepage

\end{frame}





\begin{frame}
  \frametitle{Outline}


  \tableofcontents

\end{frame}





\section{Motivation}
\subsection{The Basic Problem That We Studied}



\begin{frame}
  \frametitle{What Are Prime Numbers?}


  \begin{definition}

    A~\alert{prime number} is a number that has exactly two divisors.

  \end{definition}



  \begin{example}

    \begin{itemize}

    \item 2 is prime (two divisors: 1 and 2).

    \item 3 is prime (two divisors: 1 and 3).

    \item 4 is not prime (\alert{three} divisors: 1, 2 and 4).

    \end{itemize}

  \end{example}

\end{frame}





\section{Results}


\begin{frame}
  \frametitle{There Is No Largest Prime Number}
  \framesubtitle{The proof uses \textit{reuction ad absurdum}.}

  \begin{theorem}

    There is no largest prime number.

  \end{theorem}



  \begin{proof}

    \begin{enumerate}

    \item<1-> Suppose $p$ were the largest prime number.

    \item<2-> Let $q$ be product of the first $p$ numbers.

    \item<3-> Then $q + 1$ is not divisible by any of them.

    \item<1-> But $q + 1$ is greater than $1$, thus divisible by some prime
      number not in the first $p$ numbers. \qedhere

    \end{enumerate}

  \end{proof}

  \uncover<4->{The proof used \textit{reduction ad absurdum}.}

\end{frame}





\begin{frame}
  \frametitle{What's Still To Do?}


  \begin{block}{Answered Questions}

    How many primes are there?

  \end{block}



  \begin{block}{Open Questions}

    Is every even number the sum of two primes?

  \end{block}

\end{frame}






\begin{frame}
  \frametitle{What's Still To Do?}


  \begin{itemize}

  \item Answered Questions

    \begin{itemize}

    \item How many primes are there?

    \end{itemize}

  \item Open Questions

    \begin{itemize}

    \item Is every even number the sum of two primes?

    \end{itemize}

  \end{itemize}

\end{frame}





\begin{frame}
  \frametitle{What's Still To Do?}


  \begin{columns}
    \column{0.5\textwidth}

    \begin{block}{Answered Questions}

      How many primes are there?

    \end{block}



    \column{0.5\textwidth}

    \begin{block}{Open Questions}

      Is every even number the sum of two primes?

    \end{block}

  \end{columns}

\end{frame}





\begin{frame}
  \frametitle{What's Still To Do?}


  \begin{columns}[t]

    \column{0.5\textwidth}

    \begin{block}{Answered Questions}

      How many primes are there?

    \end{block}



    \column{0.5\textwidth}

    \begin{block}{Open Questions}

      Is every even number the sum of two primes?

    \end{block}

  \end{columns}

\end{frame}





\begin{frame}
  \frametitle{What's Still To Do?}


  \begin{columns}[t]

    \column{0.5\textwidth}

    \begin{block}{Answered Questions?}

      How many primes are there?

    \end{block}

    \pause





    \column{0.5\textwidth}

    \begin{block}{Open Questions}

      Is every even number the sum of two primes?

    \end{block}

  \end{columns}

\end{frame}





\begin{frame}
  \frametitle{What's Still To Do?}


  \begin{columns}[t]

    \column{0.5\textwidth}

    \begin{block}{Answered Questions}

      How many primes are there?

    \end{block}



    \column{0.5\textwidth}

    \begin{block}{Open Questions}

      Is every even numer the sum of two primes?
      \cite{Goldbach1742}

    \end{block}

  \end{columns}

\end{frame}





\begin{frame}[fragile]
  \frametitle{An Algorithm For Finding Prime Numbers}


\begin{verbatim}

int main(void) {
  std::vector<bool> is_prime(100, true);

  for (int i = 2; i < 100; i++) {
    std::cout << i << " ";

    for (int j = i; j < 100; is_prime[j] = false, j += i);
  }

  return 0;
}

\end{verbatim}



  \begin{uncoverenv}<2>

    Note the use of \verb|std::|.

  \end{uncoverenv}

\end{frame}





\begin{frame}[fragile]
  \frametitle{An Algorithm For Finding Primes Numbers.}


\begin{semiverbatim}
\uncover<1->{\alert<0>{int main(void)}}
\uncover<1->{\alert<0>{\{}}
\uncover<1->{\alert<1>{  \alert<4>{std::}vector<bool> is_prime(100, true);}}
\uncover<1->{\alert<1>{  for (int i = 2; i < 100; i++) \{ }}
\uncover<2->{\alert<2>{    if (is_prime[i]) \{ }}
\uncover<3->{\alert<3>{      \alert<4>{std::}cout << i << " ";}}
\uncover<3->{\alert<3>{      for (int j = i; j < 100;}}
\uncover<3->{\alert<3>{           is_prime[j] = false; j += i);}}
\uncover<2->{\alert<0>{      \} }}
\uncover<1->{\alert<0>{  return 0;}}
\uncover<1->{\alert<0>{\} }}
\end{semiverbatim}

  \visible<4->{Note the use of \alert{\texttt{std::}}.}

\end{frame}




\begin{thebibliography}{10}

\bibitem{Goldbach1742}[Goldbach, 1742]
  Christian Goldbach.
  \newblock A~problem we should try to solve before the IPSN '43 deadline.
  \newblock \emph{Letter to Leonhard Euler}, 1742.

\end{thebibliography}





\end{document}
