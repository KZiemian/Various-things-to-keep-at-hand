% ------------------------------------------------------------------------------------------------------------------
% Basic configuration of Beamera class and Jagiellonian theme
% ------------------------------------------------------------------------------------------------------------------
\RequirePackage[l2tabu, orthodox]{nag}



\ifx\PresentationStyle\notset
  \def\PresentationStyle{dark}
\fi



\documentclass[10pt,t]{beamer}
\mode<presentation>
\usetheme[style=\PresentationStyle,JUlogotitle=no]{jagiellonian}




% ---------------------------------------------------------------------
% Procesing configuration files of Jagiellonian theme loceted in directory
% "preambule".
% ---------------------------------------------------------------------
\input{./preambule/LanguageSettings/JagiellonianPolishLanguageSettings.tex}
\input{./preambule/TextposConfiguration/TextposConfiguration.tex}

\input{./preambule/ImportingLocalPackages.tex}

\input{./preambule/JagiellonianCustomizationGeneral.tex}
\input{./preambule/JagiellonianCustomizationCommands.tex}










% ---------------------------------------
% BibLaTeX
% ---------------------------------------
% Package biblatex, with biber as its backend, allow us to handle
% bibliography entries that use Unicode symbols outside ASCII.
\usepackage[
language=polish,
backend=biber,
style=alphabetic,
url=false,
eprint=true,
]{biblatex}

\addbibresource{SystemyOperacyjneBibliography.bib}





% ------------------------------
% Importing packages, libraries and setting their configuration.
% ------------------------------





% ------------------------------
% Special configuration for this particular presentation.
% ------------------------------










% ------------------------------------------------------------------------------------------------------------------
\title{Systemy operacyjne}
\subtitle{Wprowadzenie do systemu GNU/Linux}

\author{Kamil Ziemian}


\date{}
% ------------------------------------------------------------------------------------------------------------------










% ####################################################################
% Beginning of the document
\begin{document}
% ####################################################################





% Text is adjusted to the left and words are broken at the end of the line.
% Number of chars: 62k+, 12k+,
\RaggedRight





% ######################################
\maketitle
% ######################################





% ##################
\begin{frame}
  \frametitle{Spis treści}


  \tableofcontents

\end{frame}
% ##################





% ######################################
\section{Informacje ogólne}
% ######################################


% ##################
\begin{frame}
  \frametitle{???}


  Obawiam~się, że na tych konkretnych zajęciach będzie sporo przynudzania,
  ale nie widzę sposobu, jak tego uniknąć.

  Pytanie. Kto z~Państwa \textit{nie} używa na codzień systemu operacyjnego
  GNU/Linux?

  Według mnie to zajęcia są dla studentów, nie studenci dla zajęć. Tak samo
  ja tu jestem dla Państwa, a~nie Państwo dla mnie. W~związku z~tym, ja nie
  będę Państwa rozliczał z~obecności na zajęciach, tylko z~wiedzy
  i~umiejętności. Jeśli uważają Państwo, że~są lepsze rzeczy do robienia,
  niż bycie na tych zajęciach, w~to mi graj.

\end{frame}
% ##################










% ####################################################################
% ####################################################################
% Bibliography

\printbibliography





% ####################################################################
% End of the document

\end{document}
