% Autor: Kamil Ziemian

% ---------------------------------------------------------------------
% Podstawowe ustawienia i pakiety
% ---------------------------------------------------------------------
\RequirePackage[l2tabu, orthodox]{nag} % Wykrywa przestarzałe i niewłaściwe
% sposoby używania LaTeXa. Więcej jest w l2tabu English version.
\documentclass[a4paper,11pt]{article}
% {rozmiar papieru, rozmiar fontu}[klasa dokumentu]
\usepackage[MeX]{polski} % Polonizacja LaTeXa, bez niej będzie pracował
% w języku angielskim.
\usepackage[utf8]{inputenc} % Włączenie kodowania UTF-8, co daje dostęp
% do polskich znaków.
\usepackage{lmodern} % Wprowadza fonty Latin Modern.
\usepackage[T1]{fontenc} % Potrzebne do używania fontów Latin Modern.



% ---------------------------------------
% Podstawowe pakiety (niezwiązane z ustawieniami języka)
% ---------------------------------------
\usepackage{microtype} % Twierdzi, że poprawi rozmiar odstępów w tekście.
% \usepackage{graphicx} % Wprowadza bardzo potrzebne komendy do wstawiania
% grafiki.
% \usepackage{verbatim} % Poprawia otoczenie VERBATIME.
% \usepackage{textcomp} % Dodaje takie symbole jak stopnie Celsiusa,
% wprowadzane bezpośrednio w tekście.
\usepackage{vmargin} % Pozwala na prostą kontrolę rozmiaru marginesów,
% za pomocą komend poniżej. Rozmiar odstępów jest mierzony w calach.
% ---------------------------------------
% MARGINS
% ---------------------------------------
\setmarginsrb
{ 0.7in}  % left margin
{ 0.6in}  % top margin
{ 0.7in}  % right margin
{ 0.8in}  % bottom margin
{  20pt}  % head height
{0.25in}  % head sep
{   9pt}  % foot height
{ 0.3in}  % foot sep



% ---------------------------------------
% Często używane pakiety
% ---------------------------------------
% \usepackage{csquotes} % Pozwala w prosty sposób wstawiać cytaty do tekstu.
\usepackage{xcolor} % Pozwala używać kolorowych czcionek (zapewne dużo
% więcej, ale ja nie potrafię nic o tym powiedzieć).



% ---------------------------------------
% Pakiety do tekstów z nauk przyrodniczych
% ---------------------------------------
% \let\lll\undefined % Amsmath gryzie się z językiem pakietami do języka
% % polskiego, bo oba definiują komendę \lll. Aby rozwiązać ten problem
% % oddefiniowuję tę komendę, ale może tym samym pozbywam się dużego Ł.
% \usepackage[intlimits]{amsmath} % Podstawowe wsparcie od American
% % Mathematical Society (w skrócie AMS)
% \usepackage{amsfonts, amssymb, amscd, amsthm} % Dalsze wsparcie od AMS
% % \usepackage{siunitx} % Dla prostszego pisania jednostek fizycznych
% \usepackage{upgreek} % Ładniejsze greckie litery
% % Przykładowa składnia: pi = \uppi
% \usepackage{slashed} % Pozwala w prosty sposób pisać slash Feynmana.
% \usepackage{calrsfs} % Zmienia czcionkę kaligraficzną w \mathcal
% % na ładniejszą. Może w innych miejscach robi to samo, ale o tym nic
% % nie wiem.





% ---------------------------------------
% Dodatkowe ustawienia dla języka polskiego
% ---------------------------------------
\renewcommand{\thesection}{\arabic{section}.}
% Kropki po numerach rozdziału (polski zwyczaj topograficzny)
\renewcommand{\thesubsection}{\thesection\arabic{subsection}}
% Brak kropki po numerach podrozdziału



% ---------------------------------------
% Pakiety napisane przez użytkownika.
% Mają być w tym samym katalogu to ten plik .tex
% ---------------------------------------
\usepackage{latexgeneralcommands}
% \usepackage{mathshortcuts}



% ---------------------------------------
% Ustawienia różnych parametrów tekstu
% ---------------------------------------
\renewcommand{\arraystretch}{1.2} % Ustawienie szerokości odstępów między
% wierszami w tabelach.





% ---------------------------------------
% Pakiet „hyperref”
% Polecano by umieszczać go na końcu preambuły.
% ---------------------------------------
\usepackage{hyperref} % Pozwala tworzyć hiperlinki i zamienia odwołania
% do bibliografii na hiperlinki.










% ---------------------------------------------------------------------
% Tytuł, autor, data
\title{Rzeczy do kupienia}

% \author{}


% \date{}
% ---------------------------------------------------------------------










% ####################################################################
% Początek dokumentu
\begin{document}
% ####################################################################





% ######################################
\maketitle  % Tytuł całego tekstu
% ######################################





% ######################################
\section{Rzeczy które trzeba kupić}

\vspace{\spaceTwo}
% ######################################





\begin{enumerate}

\item

% \item

% \item

% \item

% \item

% \item \textit{Teoria pomiarów};

% \item Wojciech Roszkowski, \textit{Świat Chrystusa. Tom I};

% \item Michał Heller, Józef Życiński, \textit{Wszechświat --~maszyna
%     czy~myśl?};

% \item Red. L. A. Steen, \textit{Matematyka współczesna. Dwanaście
%     esejów};

% \item S. J. Gould, \textit{Niewczesny pogrzeb Darwina};

% \item Olga Tokarczuk, \textit{Bieguni};

% \item Hanya Yanagihara, \textit{Małe życie};

% \item Homer, \textit{Iliada}, \textit{Odyseja};

% \item Ks. Jelonek, \textit{Wprowadzenie do Biblii};

% \item Gustaw Meyrink, \textit{Golem};

% \item Tomasz Mann, \textit{Dokotor Faustus};

% \item Roger Scruton, \textit{Przewodnik po~kulturze współczesnej dla
%     inteligentnych};

% \item Radek Rak, \textit{Baśń o wężowym sercu albo wtóre słowo o Jakubie
%     Szeli};

% \item Alain Besancon, \textit{Anatomia widma};

% \item Red. E. Tarkowska, \textit{Zrozumieć biednego. O~dawnej i~obecnej
%     biedzie w~Polsce};

% \item Mario Vatgas Lloysa, \textit{Rozmowa w katedrze};

% \item Juan Gabriel Vasquez, \textit{Kształt ruin};

% \item Marquez, \textit{Miłość w czasach zarazy};

% \item Wiesław Myśliwski, \textit{Traktat o łuskaniu fasoli},
%   \textit{Widnokrąg}, \textit{Kamień na kamieniu};

% \item Bourbaki, \textit{Elementy historii matematyki};

% \item Platon, \textit{Państwo};

% \item Arystoteles, \textit{Etyka Nikomochejska};

% \item \textit{Recepcja w~Polsce nowych kierunków i~teorii naukowych};

% \item Herodot;

% \item H. Steinhaus;

% \item \textit{Liryka polska. Interpretacje};

% \item \textit{Zarys dziejów filozofii w~Polsce};

% \item \textit{Państwo Boże};

% \item \textit{Drabina Raju};

% \item Tomas Halik, \textit{Co nie jest chwiejne jest nietrwałe};

% \item Bernard Korzeniewski, \textit{Trzy ewolucje};

% \item R. Krasowski;

% \item Benedict, \textit{Wzory kultury};

% \item I. Bokwa, \textit{Wprowadzenie do teologii Karla Rahnera};

% \item R. Brandstaetter, \textit{Patriarchowie};

% \item G. K. Chesterton, \textit{Ortodoksja};

% \item M. Davies, \textit{Liturgiczne bomby zegarowe Vaticanum II.
%     Zniszczenie katolickiej wiary przez zmiany w katolickim kulcie};

% \item Michał Heller, \textit{Filozofia przyrody};

% \item A. MacIntyre, \textit{Dziedzictwo cnoty};

% \item M. Takesaki, \textit{Theory of operator algebras};

% \item R. Penrose, \textit{Droga do rzeczywistości};

% \item P. Johnson, \textit{Historia świata XX wieku};

% \item M. Dzielski;

% \item A. Zamoyski, \textit{Własną drogą};

% \item P. Johnson, \textit{Narodziny nowoczesności};

% \item R. Terlecki, \textit{Dzieje sowieckiej kolonii};

% \item B. Cywiński, \textit{Rodowody niepokornych};

% \item A. Nowak, \textit{Od Polski do postpolityki};

% \item P. Gay;

% \item A. Leder, \textit{Prześniona rewolucja};

% \item \textit{Oświecenie dzisiaj};

% \item E. J. Hobsbawn, \textit{Tradycje wynalezione};

% \item \textit{Słowacki. Szat-Anioł};

% \item T. Enderson, \textit{Kultura popularna, tożsamość narodowa i~życie
%     codzienne};

% \item A. Friszke, \textit{Rok 1989~r.};

% \item R. Legutko, \textit{Esej o~duszy polskiej};

% \item K. Brodacki, \textit{Trzy twarze Juliana Haraschina};

% \item J. Kurtyka, \textit{Z dziejów agonii i~podboju. Prace zebrane
%     z~zakresu najnowszej historii Polski};

% \item G. Kucharczyk, \textit{Polska myśl polityczna po 1939~r.};

% \item A. L. Sowa, \textit{Historia polityczna Polski 1944-1991};

% \item K. Janicki, \textit{Epoka hipokryzji. Seks i~erotyka
%     w~przedwojennej Polsce};

% \item A. Wielomski, \textit{Prawica w~XX wieku};

% \item N. Ferguson, \textit{Niebezpieczne związki};

% \item W. Roszkowski, \textit{Najnowsza historia Polski};

% \item Frank E. Manuel, (książka o~religioznawstwie);

% \item Urs Alterman;

% \item \textit{Chrześcijaństwo, demokracja, kapitalizm};

% \item J. Aumont, M. Marie, \textit{Analiza filmu};

% \item J. Osterhammel, \textit{Historia XIX wieku. Przebudowa świata};

% \item P. Holmes, \textit{Wiek cudów};

% \item E. Black, \textit{Wojna przeciw słabym};

% \item \textit{Polska poezja baroku};

% \item J. Gray, \textit{Liberalizm};

% \item E. Gellner;

% \item A. Golicyn, \textit{Nowe kłamstwa w~miejsce starych};

% \item \textit{Monografie historii nauki \textsc{pau}};

% \item J. Browne, \textit{Darwin, o~powstawaniu gatunków. Biografia};

% \item R. Butterwick, \textit{Polska rewolucja a~Kościół Katolicki};

% \item B. Gogol, \textit{Czerwony Sztandar. Rzecz o~sowietyzacji ziem
%     Małopolski Wschodniej};

% \item Timothy Gray Ash, \textit{Polska rewolucja. Solidarność};

% \item A. Brzezicki, \textit{Tadeusz Mazowiecki. Biografia naszego
%     premiera};

% \item I. Berlin, \textit{Korzenie romantyzmu};

% \item E. Lucas, \textit{Nowa zimna wojna};

% \item A. R. Hall, \textit{Rewolucja naukowa 1500-1800};

% \item P. Jenkis, \textit{Historia Stanów Zjednoczonych};

% \item Lew Gumilow;

% \item M. Janowski, \textit{Polska myśl liberalna do~1918 roku};

% \item M. Blondel;

% \item K. Wyka, \textit{Pan Tadeusz. Studia o~poemacie};

% \item M. Zaremba, \textit{Wielka trwoga};

% \item \textit{Eklektyzm, synkretyzm, uniwersa};

% \item P. Fiegut, \textit{Poezja w fazie krytycznej};

% \item T. Snyder, \textit{Nacjonalizm, marksizm i współczesna Europa
%     Środkowa};

% \item Ziemkiewicz, \textit{Polactwo};

% \item M. Voelle, et al., \textit{Człowiek Oświecenia};

% \item Ks. J. Tischner, \textit{Polski kształt dialogu};

% \item F. Collins, \textit{Język Boga};

% \item F. Braudel;

% \item Ks. J. Tischner, \textit{Nieszczęsny dar wolności};

% \item Ks. J. Tischner, \textit{Ksiądz na manowcach};

% \item \textit{Darwin, żywot uczonego};

% \item Roger Kimball, \textit{The rape of the masters: how political
%     correctness sabotages art};

% \item Arnold Janssen;

% \item Ludwik von Mises, \textit{Socjalizm};

% \item Robert Conquest, \textit{Wielki terror};

% \item S. Runciman, \textit{Dzieje wypraw krzyżowych};

% \item R. Graves, \textit{Mity greckie};

% \item R. Graves, \textit{Mity hebrajskie};

% \item L. Strauss, \textit{Prawo naturalne w świetle historii};

% \item Berlinski, \textit{Szatańskie urojnie};

% \item R. Wiltgen, \textit{Ren wpada do Tybru};

% \item Sokal, Briemont;

% \item Paul Davies;

% \item Wolfgang Schivelbusch, \textit{Culture of Defeat: On National
%     Trauma, Mourning, and Recovery};

% \item Bockenheim K., \textit{Dworek, kontusz, karabela};

% \item Paweł Śpiewak, \textit{Gramsci};

% \item Polska 1989-2009: ilustrowany komentarz historyczny;

% \item Spór o Polskę 1989-99: wybór tekstów prasowych;

% \item J. Tazbir, \textit{Świat panów Pasków};

% \item A. Sosnowska, \textit{Zrozumieć zacofanie: spory historyków
%     o~Europę Wschodnią, 1947-1994};

% \item \textit{Patron i dwór. Magnateria Rzeczypospolitej w XVI-XVIII
%     wieku.};

% \item Jon Dover, Helen W.~Kennedy, \textit{Kultura gier komputerowych};

% \item Diarmaid MacCulloch, \textit{The Reformation: A History},
%   alternatywny tytuł to \textit{Reformation: Europe's House Divided};

% \item \textit{Skonsumowani: jak rynek psuje dzieci, infantylizuje
%     dorosłych i~połyka obywateli};

% \item Bartłomiej Dobroczyński, \textit{New Age};

% \item Leszek Kołakowski, \textit{Główne nurty marksizmu};

% \item \textit{Państwo Boże Osiemnastowiecznych Filozofów};

% \item \textit{Liberty. The god that Failed};

% \item Ron Jeffery, \textit{Wisła jak krew czerwona};

% \item R. Browning, \textit{Cesarstwo Bizantyńskie}, \textit{Justynian
%     i~Teodora};

% \item H. Chadwick, \textit{Historia rozłamu Kościoła Wschodniego
%     i~Zachodniego. Od~czasów apostolskich do~soboru florenckiego};

% \item P. K. Hitti, \textit{Dzieje Arabów};

% \item H. Kennedy, \textit{Wielkie arabskie podboje};

% \item \textit{Historia Persji. Tom~I. Od~czasów najważniejszych
%     do~najazdu arabów};

% \item Michał Lubina, \textit{Niedźwiedź w~cieniu smoka. Rosja-Chiny
%     1991--2014};

% \item Bronisław Wildstein, \textit{Śmieszna dwuznaczność świata, który
%     oszalał};

% \item Bronisław Wildstein, \textit{Długi cień PRL-u, czyli dekomunizacja
%     której nie było};

% \item D. Góra-Szopiński, \textit{Zakorzenienie wolności. Myśl polityczna
%     Michaela Novaka};

% \item J. Grzybowski, \textit{Jacques Maritain i nowa cywilizacja
%     chrześcijańska};

% \item Roger Kimball, \textit{Długi marsz: jak rewolucja kulturalna z lat
%     60. zmieniła Amerykę};

% \item Bellantoni Patti, \textit{Jeśli to fiolet, ktoś umrze. Teoria
%     koloru w~filmie};

% \item A.~Wolff-Powęzka, \textit{Pamięć~-- brzemię i~uwolnienie. Niemcy
%     wobec nazistowskiej przeszłości (1945--2010)};

% \item Valentin L. Popov, \textit{Contact Mechanics and Friction: Physics
%     Principles and Applications};

% \item L. Ambrosio, N. Dancer, \textit{Calculus of Variations and Partial
%     Differential Equations: Topics on Geometrical Evolution Problems
%     and Degree Theory};

% \item Stephen Wiggins, \textit{Global Bifurcations and Chaos: Analytical
%     Methods};

% \item Serbio Albeverio, \textit{Operator Methods in Ordinary and Partial
%     Differential Equations};

% \item D. Boccaletti, G. Pucacco, \textit{Theory of Orbits. 1: Integrable
%     Systems and Non-perturbative Methods};

% \item Vasil E. Tarasov, \textit{Fractional Dynamics: Applications of
%     Fractional Calculus to Dynamics of Particles, Fields and Media};

% \item E. C. Curtius;

% \item Ch. West, \textit{Teologia ciała dla początkujących};

% \item R. Hilbert, \textit{Zagłada Żydów Europejskich};

% \item J. Delumeau, \textit{Cywilizacja odrodzenia};

% \item P. Manent, \textit{Intelektualna historia liberalizmu};

% \item B. Baczko, \textit{Filozofia francuskiego oświecenia};

% \item J. Juszczak, \textit{Ordoliberalizm};

% \item Albaro Vargas Llosa, \textit{Mit Che a przyszłość wolności};

% \item T. Snyder, \textit{Rekonstrukcja narodów};

% \item Red. M. Rechowicz, \textit{Dzieje teologii katolickiej w Polsce};

% \item Ricceure, \textit{Symbolika zła};

% \item J. Śniadecki;

% \item S. McMeekin, \textit{Największa grabież w historii. Jak bolszewicy
%     złupili Rosję};

% \item R. Syme, \textit{Rewolucja rzymska};

% \item E. von Kuchnelt-Leddhin, \textit{Ślepy tor};

% \item A. McGrath, \textit{Jan Kalwin. Studium kształtowania się kultury
%     Zachodu};

% \item P. Kuncewicz, \textit{Samotni wobec historii};

% \item C. Ginzburg, \textit{Ser i~robak};

% \item James Conrayd Martin, \textit{Nie ponaglaj rzeki};

% \item W. Zajewski, \textit{Czy historycy piszą prawdę};

% \item S. Węgrzynowicz, \textit{Patrioci i zdrajcy};

% \item F. Musiał, \textit{Raj grabarzy narodu};

% \item Red. Marek Kornat, \textit{Pius XII --~papież w~epoce
%     totalitaryzmów};

% \item H. Głębocki, \textit{„Diabeł Asmodeusz” w~niebieskich binoklach
%     i~kraj przyszłości. Henryk Gurowski i~Rosja};

% \item R. Fiegut, \textit{Zaproszenie do „Quidama”};

% \item M. Goliczak, \textit{Związek Radziecki w~myśli politycznej
%     polskiej opozycji 1976-1989};

% \item M. Urbankowski, \textit{Romans z~Polską};

% \item E. J\"{u}nger, \textit{Węzeł gordyjski. Eseistyka lat
%     pięćdziesiątych};

% \item J. Besal, \textit{Stanisław Żółkiewski};

% \item J. Skowronek, \textit{Adam Jerzy Czartoryski, 1770-1861};

% \item C. Shindler, \textit{Historia współczesnego Izraela};

% \item W. Bernacki, \textit{Myśl polityczna I Rzeczpospolitej};

% \item \textit{Polsko, uwierz w~swoją siłę};

% \item Encyclopedia of Mathematical Sciences;

% \item \textit{Open GL. Księga eksperta};

% \item A. Nowak, \textit{Putin. Źródła imperialnej agresji};

% \item Red. P. Musiewicz, \textit{Ronald Reagan. Nowa odsłona w 100-lecie
%     urodzin};

% \item H. Pilus, \textit{Własność i zasady w katolickiej myśli
%     społecznej};

% \item N. von Below, \textit{Byłem adiunktem Hitlera};

% \item G. Kucharczyk, \textit{Czerwone karty Kościoła};

% \item G. Kucharczyk, \textit{Kielnią i cyrklem. Laicyzacja Francji w
%     latach 1870-1914};

% \item F. Koneczny, \textit{Dzieje Polski opowiedziane dla młodzieży};

% \item M. Ekstein, \textit{Święto Wiosny. Wielka wojna i narodziny nowego
%     wieku};

% \item B. Kiereś, \textit{Tylko rodzina!};

% \item M. Soska, \textit{Za Świętą Ruś. Współczesny nacjonalizm Rosyjski
%     -- zarys ideologi};

% \item H. Pająk, \textit{Rytualna zemsta na~„kolebce” Solidarności
%     1981-2011};

% \item M. Skousen, \textit{Narodziny współczesnej ekonomii};

% \item P. Gontarczyk, \textit{Najnowsze kłopoty z~historią};

% \item L. de~Whol;

% \item G. Bardy, \textit{Charles de~Gaulle. Biografia katolika i~męża
%     stanu};

% \item J. Garrison, \textit{Ameryka jako imperium. Przywódcy świata czy
%     bandycka potęga};

% \item E. Lucas, \textit{Operacja Snowden};

% \item C. S. Lewis, \textit{Ostatnia noc świata};

% \item N. Janner SJ, \textit{Krótka historia Kościoła Katolickiego. Nowe
%     spojrzenie};

% \item \textit{Literahistorica};

% \item \textit{Bóg Zła};

% \item J. Wieliczka-Szarkowa, \textit{III Rzesza. Zbrodnia bez kary};

% \item P. Gontarczyk, \textit{Polska Partia Robotnicza. Droga do władzy
%     1941-1944};

% \item P. Moa, \textit{Mity wojny domowej w Hiszpania 1936--1939};

% \item Władymir Arnold, \textit{Lectures on Partial Differential
%     Equations};

% \item Herman H. Goldstein, \textit{A History of Numerical Analysis. From
%     the 16th through the 19th century};

% \item G. Edward Griffin, \textit{Finansowy potwór z Jekyll Island};

% \item L. Ulicka, \textit{Daniel Stein, tłumacz};

% \item R. Scruton, \textit{Kultura jest ważna};

% \item P. Gottfried, \textit{Wojna i demokracja};

% \item S. Didler, \textit{Rola neofitów w dziejach Polskich};

% \item A. Wielomski, \textit{Konserwatyzm. Główne idee i postaci};

% \item C. S. Lewis, \textit{Bóg na ławie oskarżonych};

% \item R. Spałek, \textit{Komuniści przeciw komunistom};

% \item Doug Stanton, \textit{Dwunastu odważnych. Odtajniona historia
%     konnych żołnierzy};

% \item Wicek Warszawiak, \textit{Humor w~czasie okupacji};

% \item Lawrence Wright, \textit{Wyniosłe wieże. Al-Kaida i~atak
%     na~Amerykę};

% \item Robert Mason, \textit{Powiedz, że się boisz};

% \item Dla taty: Chufo Llorens;

% \item Jearl Walker, \textit{Latający cyrk fizyki};

% \item David J.~Griffits, \textit{Introduction to~Quantum Mechanics};

% \item B. Dembowski, \textit{O filozofii chrześcijańskiej w Ameryce
%     Północnej};

% \item Wiesław Caban, \textit{Powstanie styczniowe. Polacy i Rosjanie w
%     XIX wieku};

% \item Jacek Wegner, \textit{Biesy sarmackie};

% \item Andrzej Józef Kamiński, \textit{Koszmar niewolnictwa. Obozy
%     koncentracyjne od 1896 do dziś. Analiza};

% \item Jochen B\"{o}hler, \textit{Wojna domowa. Nowe spojrzenie
%     na~odrodzenie Polski};

% \item F. Wesołowski, \textit{Zasady muzyki};

% \item Red. A. Czarniecka-Stefańska, \textit{Szukając prawdy. Edyta Stein
%     w~kulturze polskiej};

% \item E. Stein, \textit{Kobieta. Jej zadanie według natury i~łaski};

% \item Red. Umberto Eco, \textit{Historia piękna};

% \item Zdzisław Krasnodębski, \textit{Rozumienie ludzkiego zachowania.
%     Rozważania o~filozoficznych podstawach nauk humanistycznych
%     i~społecznych};

% \item Adam Przechrzta, \textit{Chorągiew Michała Archanioła};

% \item Anna Sobolewska, \textit{Mapy duchowe współczesności: co~nam
%     zostało z~Nowej Ery?};

% \item Blake J. Harris, \textit{Wojny konsolowe};

% \item Nikołaj Zieńkowicz, \textit{Tajemnice mijającego wieku. Władza
%     zakulisowe działania zatargi};

% \item Nikołaj Zieńkowicz, \textit{Od~Lenina do~Jelcyna. Kremlowska
%     księga zamachów};

% \item Marek Jan Chodakiewicz, \textit{Transformacja czy~niepodległość?};

% \item Helmuth Plessner, \textit{Śmiech i~płacz. Badania nad~granicami
%     ludzkiego zachowania};

% \item Samuel M.~Katz, \textit{Aman. Wywiad wojskowy Izraela};

% \item Praca zbiorowa, \textit{Rosja -- Chiny. Dwa modele transformacji};

% \item Mariola Marczak, \textit{Poetyka filmu religijnego};

% \item Lech Bukowski, \textit{Sade, Kafka, Kierkegaard. Między rozkoszą a
%     opresją};

% \item Philip Earl Steele, \textit{Nawrócenie i chrzest Mieszka I};

% \item Henryk Samsonowicz, \textit{My o sobie. Portret własny mieszkańców
%     ziem polskich u schyłku średniowiecza};

% \item \textit{Piłsudski (nie)znany. Historia i popkultura};

% \item Jakub Z.~Lichański, \textit{Niepopularnie o popularnej. O
%     narzędziach badań literatury};

% \item Tadeusz Manteuffel, \textit{Historia Powszechna. Średniowiecze};

% \item Robert Jung, \textit{Jaśniej niż tysiąc słońc. Losy badaczy
%     atomu};

% \item Wiesław Bator, \textit{Religia starożytnego Egiptu. Perspektywa
%     religioznawcza};

% \item Gabriela Matuszek, \textit{Maski i demony wczesnego modernizmu};

% \item Artur Szarecki, \textit{Kapitalizm somatyczny. Ciało i władza w
%     kulturze korporacyjnej};

% \item \textit{Francuskie pisma o dramacie (1537-1631)};

% \item John A. McClure, \textit{Półwiary};

% \item Jerzy Axer, Tadeusz Bujnicki, \textit{Wokół „W pustyni i w
%     puszczy”. W stulecie pierwodruku powieści};

% \item Gerd-K1laus Kaltenbrunner;

% \item Steve Brusatte, \textit{Era dinozaurów - od narodzin do upadku.
%     Nowe odkrycia i fakty o zaginionym świecie};

% \item Robert Fabbri, \textit{Wespazjan, trybun Rzymu};

% \item Michael Billing, \textit{Banalny nacjonalizm};

% \item Ernest Gellner, \textit{Narody i~nacjonalizm};

% \item \textit{Oświecenie, nieoświecone. Człowiek, natura, magia};

% \item Eric Hobsbawm, Terence Ranger, \textit{Tradycja wynaleziona};

% \item Aleksander Śpiewakowski, \textit{Samuraje};

% \item N. Davies, \textit{Serce Europy};

% \item Thomas Hylland Eriksen, \textit{Etniczność i~nacjonalizm};

% \item Benedict Andreson, \textit{Wspólnoty wyobrażone};

% \item Karol Tarnowski, \textit{W~mroku uczonej niewiedzy};

% \item Barbary Tuchman, \textit{Odległe zwierciadło, czyli rozlicznymi
%     plagami nękane XIV stulecie};

% \item Homi K. Bhabha, \textit{Miejsca kultury};

% \item Tim Edensor, \textit{Tożsamość narodowa, kultura popularna i~życie
%     codzienne};

% \item Justyna Balisz-Schmelz, \textit{Przeszłość niepokonana. Sztuka
%     niemiecka po 1945 roku jako przestrzeń i medium pamięci};

% \item Anthony D.~Smith, \textit{Etniczne źródła narodów};

% \item Anthony D.~Smith, \textit{Kulturowe podstawy narodów};

% \item Piotr Eberhardt, \textit{Rozwój światowej myśli geopolitycznej};

% \item P. Bąk, \textit{Gramatyka języka polskiego. Zarys popularny};

% \item Jacek Wegner, \textit{Rzeczpospolita. Duma i~wstyd};

% \item Wassily Kandinsky, \textit{Punkt i~linia a~płaszczyzna. Przyczynek
%     do~analizy elementów malarskich};

% \item Red. Aneta Pawłowska, Julia Sowińska-Heim, \textit{Afryka
%     i~(post)kolonializm};

% \item Robert J. C. Young, \textit{Postkolonializm. Wprowadzenie};

% \item G.~Michaelson, \textit{An Introduction to~Functional Programming
%     through Lambda Calculus};

% \item Gilberto Freyre, \textit{Panowie i niewolnicy};

% \item H. P. Barendregt, \textit{The Lambda Calculus: Its Syntax and
%     Semantics};

% \item Leon Degrelle, \textit{Wiek Hitlera};

% \item N. D. Jones, \textit{Computability and Complexity: From
%     a~Programming Perspective};

% \item Ks. Marcin Worbs, \textit{Człowiek w~misterium liturgii};

% \item Anna Grześkowiak-Krwawicz, \textit{Dyskurs polityczny
%     Rzeczypospolitej Obojga Narodów};

% \item Nowak \textit{Dzieje Polski};

% \item Margaret Atwood, \textit{Dług. Rozrachunek z~ciemną stroną
%     bogactwa};

% \item Jared Diamond, \textit{Strzelby, zarazki, maszyny. Losy ludzkich
%     społeczeństw};

% \item Edward E. Evans-Pritchard, \textit{Czary, wyrocznie i~magia
%     u~Azande};

% \item W. Szumowski, \textit{Historia medycyny filozoficznie ujęta};

% \item Michał Dondzik, Krzysztof Jajko, Emil Swoiński, \textit{Elementarz
%     Wytwórni Filmów Oświatowych};

% \item James M. Murray, \textit{Brugia: Kolebka kapitalizmu};

% \item Maciej Janowski, \textit{Narodziny inteligencji: 1750--1831};

% \item Jerzy Jedlicki, \textit{Jakiej cywilizacji Polacy potrzebują:
%     studia z dziejów idei i~wyobraźni XIX wieku};

% \item Jerzy Jedlicki, \textit{Droga do narodowej klęski};

% \item Jerzy Jedlicki, \textit{Błędne koło: 1832-1864};

% \item Jerzy Jedlicki, \textit{Nieudana próba kapitalistycznej
%     industrializacji: analiza państwowego gospodarstwa przemysłowego
%     w~Królestwie Polskim XIX w.};

% \item Jerzy Jedlicki, \textit{Klejnot i~bariery społeczne: przeobrażenia
%     szlachectwa polskiego w~schyłkowym okresie feudalizmu};

% \item Jerzy Jedlicki, \textit{Świat zwyrodniały: lęki i~wyroki krytyków
%     nowoczesności};

% \item Rob Riemen, \textit{Wieczny powrót faszyzmu};

% \item Łukasz A. Plesnar, \textit{Twarze Westernu};

% \item S. Prat, \textit{Język C++. Szkoła programowania};

% \item L. Strauss, \textit{Prawo naturalne w~świetle historii};

% \item Oskar Halecki, \textit{Tysiąc lat polski katolickiej};

% \item Krzysztof Mazur, \textit{Przekroczyć nowoczesność. Projekt
%     polityczny ruchu społecznego Solidarność};

% \item Marian Henryk Serejski, \textit{Europa a rozbiory Polski: studium
%     historiograficzne};

% \item Jerzy Łanowski, \textit{Antologia anegdoty antycznej: teraz trzeci
%     raz wydane historyjki budujące i niebudujące z autorów greckich i
%     rzymskich};

% \item Artur Domosłowski, \textit{Kapuściński non-fiction};

% \item Michael Moran, \textit{Kraj z Księżyca: podróże do serca Polski};

% \item Pierre Hadot, \textit{Filozofia jako ćwiczenie duchowe};

% \item Robert D. Richtmayer, \textit{Principles of advanced mathematical
%     physics};

% \item Walter Burkert, \textit{Stwarzanie świętości. Ślady biologii
%     we~wczesnych wierzeniach religijnych};

% \item A. I. Anselm, \textit{Podstawy fizyki statystycznej
%     i~termodynamiki};

% \item Z. Krasnodębski, \textit{Demokracja peryferii};

% \item Maria Dzielska, \textit{Hypatia z~Aleksandrii};

% \item Paweł Śpiewak, \textit{Spór o~Polskę, 1989--99};

% \item Tadeusz Zieliński, \textit{Religia starożytnej Grecji};

% \item Ernest Gellner, \textit{Postmodernizm, rozum i~religia};

% \item Daniel Beauvois, \textit{Polacy na Ukrainie 1831-1863. Szlachta
%     polska na Wołyniu, Podolu i Kijowszczyźnie};

% \item \textit{Polskie mity polityczne XIX i XX wieku};

% \item \textit{O nas bez nas. Historia Polski w historiografiach
%     obcojęzycznych};

% \item Leibniz, \textit{Wyznanie wiary filozofa, Rozprawa metafizyczna;
%     Monadologia; Zasady natury i łaski oraz inne pisma filozoficzne};

% \item Hanna Świda-Ziemba, \textit{Człowiek wewnętrznie zniewolony.
%     Mechanizmy i konsekwencje minionej formacji --~analiza
%     psychologiczna};

% \item Braudel Fernand, \textit{Dynamika kapitalizmu};

% \item Bod Rens, \textit{Historia humanistyki};

% \item \textit{Teologia i filozofia żydowska wobec Holocaustu};

% \item Acton, \textit{Historia wolności: wybór esejów};

% \item Andrzej Żbikowski, \textit{Żydzi};

% \item Stefan Bartkowski, \textit{Pod wspólnym niebem: krótka historia
%     Żydów w~Polsce i~stosunków polsko-żydowskich};

% \item Arystoteles, \textit{Retoryka};

% \item Agnieszka Urbańczyk, Diamentowy Grant, \textit{Polityczność
%     science fiction w recepcji fanowskie};

% \item \textit{Lech Wzbudzony};

% \item \textit{Unintended Reformation};

% \item J. Polit, \textit{Chiny};

% \item Paweł Śpiewak, \textit{Teologia i~filozofia żydowska wobec
%     Holocaustu};

% \item J.K. Fairbank, \textit{Historia Chin. Nowe spojrzenie};

% \item \textit{Nowożytna historia Chin}, red. R. Sławiński;

% \item R. Sławiński, \textit{Geneza Chińskiej Republiki Ludowej};

% \item K. Seitz, \textit{Chiny. Powrót Olbrzyma};

% \item A. Bolesta, \textit{Chiny w okresie transformacji};

% \item \textit{Chiny. Przemiany państwa i społeczeństwa w okresie reform
%     1978--2000}, red. K. Tomala;

% \item Andrzej Napiórkowski OSPPE, \textit{Teologie XX i~XXI wieku};

% \item Alfred V.~Aho, Jeffrey D.~Ullman, \textit{Wykłady z~informatyki
%     z~przykładami w~języku~C};

% \item Jerzy Eisler, \textit{Co nam zostało z tamtych lat. Dziedzictwo
%     PRL};

% \item Peter Burke;

% \item B. Kozera, \textit{Literatura a~religia. Polska współczesna
%     powieść katolicka};

% \item \textit{Physics of living systems};

% \item Jadwiga Staniszkis, \textit{Samoograniczająca~się rewolucja};

% \item Jadwiga Staniszkis, \textit{Postkomunizm. Próba opisu};

% \item Z. Wójcik, \textit{Dzikie Pola w~ogniu. O~Kozaczyźnie w~dawnej
%     Rzeczypospolitej};

% \item Jan Kofman, Wojciech Roszkowski, \textit{Transformacja
%     i~postkomunizm};

% \item M. Gołaszewska, \textit{Estetyka współczesna};

% \item E. Badinter, \textit{XY~-- tożsamość mężczyzny};

% \item R. Bly, \textit{Żelazny Jan. Rzecz o~mężczyznach};

% \item Z. Wójcik, \textit{Wojny kozackie w dawnej Polsce};

% \item Z. Wójcik, \textit{Dzieje Rosji: 1533-1801};

% \item Vigarello Georges, \textit{Historia gwałtu};

% \item Tannahill Reay, \textit{Historia kuchni};

% \item Éliphas Lévi, \textit{Historia magii};

% \item Meyer Michel (red.), \textit{Historia retoryki od Greków do dziś};

% \item Simmel Georg, \textit{Filozofia pieniądza};

% \item Dahl Robert, \textit{Demokracja i~jej krytycy};

% \item Z. Wojcik, \textit{Jan Sobieski: 1629-1696};

% \item Z. Wójcik, \textit{Jan III Sobieski};

% \item Ariès Philippe, \textit{Historia dzieciństwa. Dziecko i rodzina w
%     czasach ancien régime’u};

% \item Bataille Georges, \textit{Historia erotyzmu};

% \item Vigarello Georges, \textit{Historia otyłości};

% \item Flandrin Jean-Louis, \textit{Historia rodziny};

% \item Z. Wójcik, \textit{Jan Kazimierz Waza};

% \item Z. Wójcik, \textit{Józef Piłsudski 1867-1935};

% \item \textit{Legendy uświęcone. Twórczość J. R. R. Tolkiena a
%     chrześcijaństwo};

% \item Stefan Bratkowski, \textit{Nieco inna historia cywilizacji: dzieje
%     banków, bankierów i obrotu pieniężnego};

% \item Izrael Szahad, \textit{Żydowskie dzieje i religia; Żydzi i goje –
%     XXX wieków historii};

% \item Izrael Szahad, \textit{Tel Awiw za zamkniętymi drzwiami};

% \item Andrzej Żbikowski, \textit{Ideologia antysemicka w~Polsce
%     1848-1918};

% \item Władysław Bruliński, \textit{Antykościół};

% \item Władysław Bruliński, \textit{Co to jest marksizm?};

% \item Władysław Bruliński, \textit{Czerwone palmy historii};

% \item Władysław Bruliński, \textit{Dokąd idziesz Polsko?};

% \item \textit{Żydzi i judaizm we współczesnych badaniach polskich};

% \item Artur Eisenbach, \textit{Emancypacja Żydów na ziemiach polskich
%     1785-1870 na tle europejskim};

% \item Artur Eisenbach, \textit{Z dziejów ludności żydowskiej w Polsce w
%     XVIII i XIX w.};

% \item Artur Eisenbach, \textit{Kwestia równouprawnienia Żydów w
%     Królestwie Polskim};

% \item Marian Fuks, \textit{Humor Żydów polskich (do 1939 r.)};

% \item Marian Fuks, \textit{Żydzi w Polsce – Dawniej i dziś};

% \item Marian Fuks, \textit{Prasa żydowska w Warszawie 1823-1939};

% \item Marian Fuks, \textit{Z dziejów wielkiej katastrofy narodu
%     żydowskiego};

% \item August Grabski, \textit{Studia z dziejów i kultury Żydów w Polsce
%     po 1945 r.};

% \item Michael Hesemann, \textit{Kłamstwa Hitlera};

% \item David Hockney, \textit{Wiedza tajemna. Sekrety technik malarskich
%     Dawnych Mistrzów};

% \item Anna Magdalena Mandrela, \textit{Tomizm Garrigou-Lagrange’a wobec
%     wizji filozoficznej Teilharda de Chardin};

% \item Leonie Swann, \textit{Powiększ Sprawiedliwość owiec. Filozoficzna
%     powieść kryminalna};

% \item Antoine de Saint-Exupéry, \textit{Twierdza};

% \item Umberto Eco, \textit{Historia brzydoty};

% \item Brian Reynolds Myers, \textit{Najczystsza rasa: Propaganda Korei
%     Północnej};

% \item Tomasz Strzyżewski, \textit{Wielka księga cenzury PRL w
%     dokumentach};

% \item Orlando Figes, \textit{Tragedia narodu. Rewolucja rosyjska
%     1891-1924};

% \item Jon Savage, \textit{Teenage: The Creation of Youth Culture};

% \item Lawrence Weschler, \textit{Mr. Wilson's Cabinet of Wonder: Pronged
%     Ants, Horned Humans, Mice on Toast, and Other Marvels of Jurassic
%     Technology};

% \item Christophe Galfard, \textit{Wszechświat w twojej dłoni};

% \item Swietłana Aleksijewicz, \textit{Cynkowi chłopcy};

% \item Aleksander Hertz, \textit{Amerykańskie stronnictwa polityczne};

% \item Aleksander Hertz, \textit{Żydzi w kulturze polskiej};

% \item Aleksander Hertz, \textit{Wyznania starego człowieka};

% \item Maurycy Horn, \textit{Żydowskie bractwa rzemieślnicze na ziemiach
%     polskich, litewskich, białoruskich i ukraińskich w latach
%     1613-1850};

% \item Maurycy Horn, \textit{Walka chłopów czerwonoruskich z wyzyskiem
%     feudalnym w latach 1600-1643};


% \item Adam Kaźmierczyk, \textit{Sejmy i sejmiki szlacheckie wobec Żydów
%     w II połowie XVII wieku};

% \item Adam Kaźmierczyk, \textit{Żydzi w dobrach prywatnych. W świetle
%     sądowniczej i administracyjnej praktyki dóbr magnackich w wiekach
%     XVI-XVIII};

% \item Krystyna Kersten, \textit{Polacy-Żydzi-Komunizm. Anatomia półprawd
%     1939-1968};

% \item Krystyna Kersten, \textit{Pogrom Żydów w Kielcach 4 lipca 1946
%     r.};

% \item Andrzej Sulima Kamiński, \textit{Historia Rzeczypospolitej Wielu
%     Narodów 1505-1795. Obywatele, ich państwa, społeczeństwo,
%     kultura};

% \item Cynarski S., \textit{Zygmunt August};

% \item Cyra A., \textit{Rotmistrz Pilecki. Ochotnik do Auschwitz};

% \item \textit{Czy ktoś przebije ten mur? Sprawa Pyjasa};

% \item Dudek A., Zblewski Z., \textit{Utopia nad Wisłą. Historia
%     Peerelu};

% \item Czapliński W., \textit{Władysław IV i jego czasy};

% \item Dybiec J., \textit{Nie tylko szablą. Nauka i kultura polska w
%     walce o~utrzymanie tożsamości narodowej 1795--1918};

% \item Eisler J., \textit{Zarys dziejów politycznych Polski 1944--1989};

% \item Grzybowski S., \textit{Henryk Walezy};

% \item Ignatowicz I., \textit{Społeczeństwo polskie 1864--1914};

% \item \textit{Inteligencja polska XIX i XX wieku. Studia}, red. R.
%   Czapulis-Rastenis, t.1-6;

% \item Jedynak B., \textit{Obyczaje domu polskiego w~czasach niewoli
%     1795--1918};

% \item Kaczmarczyk J., \textit{Bohdan Chmielnicki};

% \item Kawalec K., \textit{Roman Dmowski};

% \item \textit{Kobieta i kultura życia codziennego. Wiek XIX i XX. Zbiór
%     studiów}, red. A. Żarnowska, A. Szwarc;

% \item \textit{Kobieta i społeczeństwo na ziemiach polskich w XIX wieku,
%     zbiór studiów}, red. A. Żarnowska, A. Szwarc;

% \item Konopczyński W., \textit{Dzieje Polski nowożytnej};

% \item Kowecka E., \textit{W salonie i w kuchni. Opowieść o kulturze
%     materialnej pałaców i dworów polskich w XIX wieku};

% \item Krawczak T., \textit{W szlacheckim zaścianku};

% \item Kuchowicz Z., \textit{Miłość staropolska};

% \item Kuchowicz Z., \textit{Obyczaje staropolskie XVII-XVIII w.};

% \item Litak S., \textit{Od reformacji do Oświecenia. Kościół katolicki
%     w~Polsce nowożytnej};

% \item Mączak A., \textit{Klientela. Nieformalne systemy władzy w Polsce
%     Europie XVI-XVIII w.};

% \item Łuczak C., \textit{Polska i Polacy w drugiej wojnie światowej};

% \item Molenda J., \textit{Chłopi. Naród. Niepodległość. Kształtowanie
%     się postaw narodowych i~obywatelskich chłopów w~Galicji
%     i~Królestwie polskim w~przededniu odrodzenia Polski};

% \item Możdżyńska-Nawotka M., \textit{O~modach i~strojach};

% \item \textit{Obyczaje w Polsce. Od średniowiecza do czasów
%     współczesnych};

% \item Olszewski D., \textit{Polska kultura religijna na przełomie XIX
%     i~XX wieku};

% \item Olszewski H., \textit{O skutecznym rad sposobie};

% \item \textit{Polska XVII wieku. Państwo, społeczeństwo, kultura}, red.
%   J. Tazbir;

% \item Paczkowski A., \textit{Pół wieku dziejów Polski, 1939--1989};

% \item \textit{Polska na przestrzeni wieków}, red. J. Tazbir;

% \item Przyboś A., \textit{Michał Korybut Wiśniowiecki 1640-1673};

% \item Rok B., \textit{Człowiek wobec śmierci w kulturze staropolskiej};

% \item \textit{Rzeczpospolita wielu narodów i jej tradycje}, red. M.
%   Markiewicz, A. Link-Lenczewski;

% \item \textit{Społeczeństwo polskie od X do XX wieku}, red. I.
%   Ignatowicz, A. Mączak, B. Zientara, J. Żarnowski;

% \item Staszewski J., \textit{August II Mocny};

% \item Staszewski J., \textit{August III Sas};

% \item Suleja W., \textit{Józef Piłsudski};

% \item Szubarczyk P., \textit{Inka. Zachowałam się jak trzeba\ldots};

% \item Topolski J., \textit{Polska w czasach nowożytnych. Od europejskiej
%     potęgi do utraty niepodległości};

% \item Terlecki R., \textit{Miecz i tarcza komunizmu. Historia aparatu
%     bezpieczeństwa 1944--1990};

% \item \textit{Tradycje polityczne dawnej Polski}, red. A.
%   Sucheni-Grabowska, A. Dybowska;

% \item Wandycz P., \textit{Pod zaborami. Ziemie Rzeczypospolitej w latach
%     1795--1918};

% \item Wisner H., \textit{Władysław IV Waza};

% \item Wisner H., \textit{Zygmunt III Waza};

% \item Zielińska Z., \textit{Ostatnie lata Pierwszej Rzeczypospolitej};

% \item Zblewski Z, \textit{Abecadło Peerelu};

% \item Zienkowska K., \textit{Stanisław August Poniatowski};

% \item Zdrada J., \textit{Historia Polski 1795--1914};

% \item Żarnowski, J., \textit{Polska 1918--1939. Praca, technika,
%     społeczeństwo};

% \item Ziejka F., \textit{Złota legenda chłopów polskich};

% \item Żołądź D., \textit{Ideały edukacyjne doby staropolskiej. Stanowe
%     modele i potrzeby edukacyjne szesnastego i siedemnastego wieku};

% \item R. Wapiński, \textit{Historia polskiej myśli politycznej XIX
%     i~XX~wieku};

% \item R.R. Ludwikowski, \textit{Historia polskiej myśli politycznej};

% \item W. Bernacki, \textit{Liberalizm polski};

% \item B. Szlachta, \textit{Z dziejów polskiego konserwatyzmu};

% \item M. Śliwa, \textit{Polska myśl socjalistyczna 1892--1948};

% \item Z. Ogonowski, \textit{Filozofia polityczna w~Polsce XVII w.
%     i~tradycje demokracji europejskiej};

% \item S. Tarnowski, \textit{Pisarze polityczni XVI wieku};

% \item S. Tarnowski, \textit{Historia literatury polskiej, t.2};

% \item W. Konopczyński, \textit{Polscy pisarze polityczni XVIII w.};

% \item H. Olszewski, \textit{Doktryny prawno-ustrojowe czasów saskich};

% \item K. Waliszewski, \textit{Potoccy i Czartoryscy, walka stronnictw i
%     programów politycznych przed upadkiem Rzeczypospolitej 1734--1763};

% \item Red. Tomasz Dołęgowski, \textit{Przewodnik po moralnym
%     kapitalizmie};

% \item Besala J., \textit{Stefan Batory};

% \item Cieślak E., \textit{Stanisław Leszczyński};

% \item Bogucka M., \textit{Staropolskie obyczaje XVI-XVII w};

% \item Cz. Michalski, \textit{Western};

% \item A. Chwalba, \textit{III Rzeczpospolita~-- raport specjalny};

% \item Jean-Paul Bled, \textit{Bismarck. Żelazny kanclerz};

% \item Marcin Król, \textit{Byliśmy głupi};

% \item J. Wójcik, \textit{Labirynt światła};

% \item A. Chwalba, \textit{Historia Polski 1795-1918};

% \item Brzoza C., Sowa A., \textit{Historia Polski 1918-1945};

% \item Cz. Michalski, \textit{Western i~jego bohaterowie};

% \item Rafał Marszałek, \textit{Pamflet na kino codzienne};

% \item Rafał Marszałek, \textit{Polska wojna w obcym filmie};

% \item Rafał Marszałek, \textit{Filmowa pop-historia};

% \item Rafał Marszałek, \textit{Kino rzeczy znalezionych};

% \item J. Skwara, \textit{Western odrzuca legendę};

% \item Władysław Konopczyński, \textit{Konfederacja barska};

% \item Michał Łuczewski, \textit{Odwieczny naród. Polak i~katolik
%     w~Żmiącej};

% \item Alan Bullock, \textit{Hitler. Studium tyrani};

% \item Tadeusz Lubelski, \textit{Historia Kina Polskiego, Twórcy, Filmy,
%     Konteksty};

% \item Marek Haltof, \textit{Kino polskie};

% \item \textit{Kino bez tajemnic};

% \item David Bordwell, Kristin Thompson, \textit{Film Art. Sztuka
%     filmowa. Wprowadzenie};

% \item W. Stróżewski, \textit{Estetyka};

% \item Vigarello Georges, \textit{Historia czystości i brudu};

% \item Muchembled Robert, \textit{Orgazm i Zachód};

% \item Sloterdijk Peter, \textit{Pogarda mas};

% \item Gately Iain, \textit{Kulturowa historia alkoholu};

% \item Eco Umberto, \textit{Poszukiwanie języka doskonałego w kulturze
%     europejskiej};

% \item Coogan Michael, \textit{Bóg i~seks. Co naprawdę mówi Biblia};

% \item Secher Reynald, \textit{Ludobójstwo francusko-francuskie};

% \item Vigarello Georges, \textit{Historia urody};

% \item Higman B.W., \textit{Historia żywności};

% \item Tannahill Reay, \textit{Historia seksu};

% \item Ramamurti Rajaraman, \textit{Solitons and~instantons};

% \item Wilson Edward O., \textit{Znaczenie ludzkiego istnienia};

% \item Karl Loewith, \textit{Historia powszechna i dzieje zbawienia};

% \item Tony Judt, \textit{Historia niedokończona. Francuscy
%     intelektualiści 1944-1956};

% \item Karl Loewith, \textit{Od Hegla do Nietzschego. Rewolucyjny przełom
%     w myśli XIX wieku};

% \item J. Fiske \textit{Zrozumieć kulturę popularną};

% \item Anatol Taras, \textit{Anatomia nienawiści};

% \item R. Rodes, \textit{Jak powstała bomba atomowa?};

% \item A. Tarski;

% \item Jacek Trznadel, \textit{Z~popiołów czy wstaniesz?};

% \item Jacek Trznadel, \textit{Spór o~całość: Polska 1939-2004};

% \item Karol Buczek, \textit{Studia z dziejów ustroju
%     społeczno-gospodarczego Polski piastowskiej};

% \item Red. S. Kowalczyk, E. Balawajder, \textit{Jacques Maritain,
%     prekursor soborowego humanizmu};

% \item T. M. Jaroszewski, \textit{Osobowość i wspólnota. Problemy
%     osobowości we współczesnej antropologii filozoficznej --~marksizm,
%     strukturalizm, egzystencjalizm, personalizm chrześcijański};

% \item Roman Graczyk, \textit{Od uwikłania do autentyczności. Biografia
%     polityczna Tadeusza Mazowieckiego};

% \item S. Wiggins, \textit{Introduction to Applied Nonlinear Dynamical
%     Systems and Chaos};

% \item Władymir Arnold, \textit{Catastrophe Theory};

% \item Red. Christian B\"{a}r, Klaus Fredenhagen, \textit{Quantum Field
%     Theory on Curved Spacetimes};

% \item Sholmo Sternberg, \textit{Semi-Riemann Geometry and General
%     Relativity};

% \item Peter B. Gilkey, \textit{Invariance theory, the heat equation, and
%     the Atiyah-Singer Index Theorem};

% \item Richard S. Palais, \textit{A Modern Course on Curves and
%     Surfaces};

% \item J\"{u}rgen Jost, \textit{Geometry and physics};

% \item Charles Freeman, \textit{A New History of Early Christianity};

% \item Bart D. Ehrman, \textit{Whose Word is it. The Story Behind who
%     changed the New Testament and why};

% \item Robert V. Huber, Stephen M. Miller \textit{Historia Biblii};

% \item John Galindo, Owen F. Cummings, \textit{Duchowość, intymność i
%     seksualność};

% \item James L. Papandrea, \textit{Depozyt wiary};

% \item Anadijiban Das, Andrew DeBenedictis, \textit{The general theory of
%     relativity. A mathematical exposition};

% \item Sergio A. Albeverio, Raphael J. H\o egh-Krohn, Sonia Mazzucchi,
%   \textit{Mathematical theory of Feynman path integrals};

% \item Leah Darrow, \textit{Inna strona piękna};

% \item Michał Paradowski, \textit{Trzydziestolecie Drugiego Soboru
%     Watykańskiego};

% \item C. Radhakrishna Rao, \textit{Statystyka i~prawda};

% \item Donald Ritchie, \textit{The Films of Akira Kurosawa};

% \item Martin Konings, \textit{The Emotional Logic of Capitalism. What
%     Progressives Have Missed};

% \item David Sloan Wilson, \textit{Darwin's Cathedral: Evolution,
%     Religion, and the~Nature~of Society};

% \item Ch.~R.~Browning, \textit{Zwykli ludzie. 101~Policyjny Batalion
%     Rezerwowy i~„ostateczne rozwiązanie” w~Polsce}

% \item Viviana A. Zelizer, \textit{The Social Meaning of Money: Pin
%     Money, Paychecks, Poor Relief, and Other Currencies};

% \item Randy Shilts, \textit{And the Band Played On};

% \item Rana Mitter, \textit{Gorzka rewolucja};

% \item Rana Mitter, \textit{Chiny nowoczesne};

% \item Peter Seewald, \textit{Benedykt XVI. Portret z~bliska};

% \item C. Radhakrishna Rao, \textit{Modele liniowe statystyki matematycznej};

% \item Marta Przybyła, \textit{I dam wam serce nowe};

% \item Alfred Tarski, \textit{Wprowadzenie do logiki};

% \item I. Wallerstein, \textit{Europejski uniwersalizm. Retoryka władzy};

% \item Abbé Jacques Meinvielle, \textit{De Lamennais ŕ Maritain};

% \item Zofia Szmydt, \textit{Transformacja Fouriera i~równania różniczkowe
%     liniowe};

% \item Ralph Martin, \textit{Kościół w kryzysie. Ścieżki wyjścia};n

% \item Stefan Wyszyński, \textit{Przestrogi dla Polaków};

% \item Tomasz Terlikowski, \textit{Czego księża nie powiedzą Ci
%     o~antykoncepcji?};

% \item V. Messori, \textit{Kościół Katolicki i~jego wrogowie};

% \item Stefan Ziemba, \textit{Analiza drgań}, dwa tomy;

% \item Romano Guardini, \textit{Wolność-łaska-los};

% \item Romano Guardini, \textit{O~istocie chrześcijaństwa};

% \item Gianfranco Ravasi, \textit{Kohelet};

% \item Carrie Gress, \textit{Odnowa};

% \item Jan Krempa, Barbara Mażbic-Kulma, \textit{Elementy logiki, teorii
%     mnogości i~algebry};

% \item Charles Moore, \textit{Margaret Thatcher};

% \item G. Polya, \textit{Jak to rozwiązać?};

% \item Donald J. Trump, Meredith McIver, \textit{Nigdy się nie poddawaj!};

% \item Patti Bellantoni, \textit{Jeśli to fiolet, ktoś umrze. Teoria koloru
%     w~filmie};

% \item Józef Pawłowski, \textit{Przeszłość w~ideologii Komunistycznej Partii
%     Chin};

































\end{enumerate}










% ######################################
\newpage
\section{Rzeczy które można warto kupić}

\vspace{\spaceTwo}
% ######################################



\begin{enumerate}

\item A. Schneider, \textit{Wiosna Kościoła, która nie nadeszła};

\item Wojciech Kusarski, \textit{Kardynał Bolesław Kominek. Biskup,
  dyplomata, wizjoner};

\item Anna Dąmbska, \textit{Zadanie Polski};

\item Papież Franciszek, \textit{Powróćmy do marzeń};

\item Robert Skrzypczak, \textit{Wiara i~seks. JP II o małżeństwie
    i~rodzinie};

\item Robert Skrzypczak, \textit{Karol Wojtyła na Soborze Watykańskim~II};

\item Kard. Gerhard M\"{u}ller, \textit{Prawda. Raport o~stanie Kościoła};

\item Kard. Stefan Wyszyński, \textit{O~godności kobiety};

\item Abp Fulton Sheen, \textit{Komunizm i~sumienie Zachodu};

\item Peter Seewald, \textit{Jezus Chrystus. Biografia};

\item Dawn Eden, \textit{Dreszcz czystości};

\item Dawn Eden, \textit{Zaskoczenie radością};

\item Jakub Wiech, \textit{Globalne ocieplenie};

\item Jennifer Morse, \textit{Szczęścliwe małżeństwo};

\item A.W. Bicadze, D.F. Kaliniczenko, \textit{Zbiór zadań z~równań fizyki
    matematycznej};

\item Ks. Franciszek Blachnicki, \textit{Trzy nawrócenia};

\item Leszek Kołakowski, \textit{Świadomość religijna i~więź kościelna};

\item Janusz Tazbir, \textit{Arianie i~katolicy};

\item Henri de Lubac, \textit{Na drogach bożych};

\item Piotr Matywiecki, \textit{Dwa oddechy};

\item Hans Urs von Blthasar, \textit{Światło słowa};

\item Kard. Stefan Wyszyński, \textit{Kobieta w~Polsce współczesnej};

\item \textit{Nowa ewangelizacja i~jej realizacja w~Polsce po 1989 roku};

\item Abp Józef Życiński, \textit{Świat musi mieć sens};

\item Thomas Merton, \textit{Ślub konwersacji};

\item Św. Josemaria Escriva, \textit{Przyjaciele Boga};

\item Ks. Grzegorz Bachanek, \textit{Josepha Ratzingera nauka o~Kościele};

\item \textit{Prześladowani i~zapomniani. Raport o~prześladowaniu
    chrześcijan w~latach 2009-2010};

\item Paul Johnson, \textit{Historia chrześcijaństwa};

\item Mieczysław Albert Krąpiec OP, \textit{O~chrześcijańską kulturę};

\item \textit{Księga 1000-lecia katolicyzmu w~Polsce};

\item \textit{Ilustrowana Biblia dla dzieci};

\item Joseph Ratzinger, \textit{Chrystus i~Jego Kościół};

\item Ks. Piotr Paweł Łapa, \textit{Karios. Czas w~którym Bóg sam
    wystarcza};

\item Katarzyna ze Sieny, \textit{Modlitwy};

\item Ludwik Wiśniewski OP, \textit{Czarne z~białym};

\item Earle E. Cairns, \textit{Z~chrześcijaństwem przez wieki};

\item Carlo Maria Martini, \textit{Powołanie świeckich};

\item William J. Hoye, \textit{Teologiczne błędy myślowe};

\item Ks. Piotr Mazurkiewicz, \textit{Kościół i~demokracja};

\item \textit{Kościół a~środki społecznego przekazu};

\item Tomasz Zaklukiewicz, \textit{Nieśmiertelność sprawiedliwych};

\item \textit{Interpretacja (w) dialogu. Tożsamość egzegezy biblijnej};

\item \textit{Rok 2. Ewangelizacja świata};

\item Th. Deman OP, \textit{Chrystus Pan i~Sokrates};

\item Richard Niebuhr, \textit{Chrystus a~kultura};

\item \textit{Słowo nad słowami. Antologia poezji Starego Przymierza
    w~przekładzie R. Brandstaettera};

\item Kardynał Karol Wojtyła, \textit{Znak któremu sprzeciwiać~się będą};

\item \textit{Spory chrystologiczne w~Polsce w~drugiej połowie XVI wieku};

\item Kard. Józef Ratzinger, \textit{Służyć prawdzie};

\item Stanisław Obirek, \textit{Sezon dialogu};

\item \textit{Medytacja nad Ewangelią dni powszednich};

\item Magda Grabowska, \textit{Kobieta warta Królestwa};

\item \textit{Mów Panie, Twój sługa słucha};

\item Kard. Joseph Ratzinger, \textit{Nowa pieśń dla Pana};

\item Krzysztof Mech, \textit{Chrześcijaństwo i~dialektyka w~koncepcji
    Paula Tillicha};

\item Hopcroft, Ullman, \textit{Wprowadzenie do teorii automatów, języków
    i~obliczeń};

\item Piotr Bąk, \textit{Gramatyka języka polskiego};

\item Ewa Ross, Jacek Ross, \textit{Unity i~C\#. Podstawy programowania
    gier};

\item \textit{Propedeutyka informatyki};

\item Witold Sikorski, \textit{Wykłady z~podstaw informatyki};

\item \textit{List do Hebrajczyków. Tom~X: Wstęp, przekład, komentarz};

\item Giuseppe Ricciotii, \textit{Życie Chrystusa};

\item Aho, Hopcroft, Ullmann, \textit{Algorytmy i~struktury danych};

\item Hans K\"{u}ng, \textit{Życie wieczne};

\item \textit{Podstawy elektroniki};

\item Bill Browder, \textit{Czerwony alert};

\item \textit{Teoretyczne założenia katechezy młodzieżowej};

\item Roman Murawski, \textit{Filozofia matematyki. Antologia tekstów
    klasycznych};

\item \textit{Współczesna filozofia matematyki. Wybór tekstów};

\item Heidi Blake, \textit{Krwawe pozdrowienia z~Rosji};

\item \textit{Historia gospodarcza Polski (1939-1989)};

\item Tomasz A. Żak, \textit{Komu służy kultura?};

\item Michael Axworthy, książki o~Iranie;

\item Donald Trump, \textit{The Art of the Deal, sztuka robienia interesów};

\item Longin Pastusiak, \textit{Donald Trump. Pierwszy taki prezydent};

\item \textit{Wybrane metody numeryczne algebry liniowej w~ekonometrii};

\item Edward L. Bernays, \textit{Propaganda};

\item Janusz Kaliński, \textit{Zarys historii gospodarczej XIX i~XX~w.}

\item Janusz Kaliński, \textit{Gospodarka Polski w~latach 1944-1989};

\item Janusz Kaliński, \textit{Gospodarka Polski w~XX wieku};

\item Joseph Stiglitz, \textit{Cena nierówności};

\item Joseph Stiglitz, \textit{Globalizacja};

\item Joseph Stiglitz, \textit{Fair trade};

\item Joseph Stiglitz, \textit{Szalone lata dziewięćdziesiąte};

\item \textit{Podstawy biologii komórki};

\item Ks. Rajmund Pietkiewicz, \textit{Biblia Polonorum};

\item Mieczysław A. Krąpiec OP, \textit{Człowiek i~prawo naturalne};

\item D. J. Panow, \textit{Metod numeryczne rozwiązywania równań
    różniczkowych cząstkowych};

\item Francesco, \textit{Katechezy o~Eucharystii};

\item \textit{Myśl Amerykańskiego oświecenia};

\item Paweł Lisicki, \textit{Doskonałość i~nędza};

\item Paweł Lisicki, \textit{Nowa, wspaniała przyszłość};

\item Monika Wójcik, \textit{Człowiek, osoba, płeć};

\item Tomasz Wiślicz, \textit{Zelman Wolfowicz i~jego rządy w~starostwie
  drohobyckim w~połowie XVIII w.};

\item \textit{O~Bogu i~o~człowieku. Tom~2~SPK};

\item Barack Obama, \textit{Ziemia obiecana};

\item Rafał Łatka, \textit{Polityka władz PRL wobec Kościoła};

\item Rafał Łatka, \textit{Urząd do spraw wyzwań};

\item Marek Kornat, Mariusz Wołos, \textit{Józef Beck. Biografia};

\item \textit{Prymas Stefan Wyszyński. Biografia IPN};

\item Leah Darrow, \textit{Inna strona piękna};

\item Willard Van Orman Quine, \textit{Filozofia logiki};

\item Jan Łukasiewicz, \textit{Elementy logiki matematycznej};

\item Marguerite A. Peeters, \textit{Globalizacja zachodniej rewolucji
    kulturowej};

\item Michael S. Rose, \textit{Żegnajcie, dobrzy ludzie};

\item Anna Mandreal, \textit{Kościół Katolicki wobec buddyzmu};

\item Alex Berenson, \textit{Wierny szpieg};

\item Keith Houston, \textit{Ciemne typki};

\item Andrzej Walicki, \textit{O~Rosji inaczej};

\item Vivek Ramaswam, \textit{Woke, Inc.};

\item Garry Kasparov, \textit{Ostatni bastion};

\item Gabriele Kuby, \textit{Rewolucja genderowa};

\item Jarosław Hrycak, \textit{Prorok we własnym kraju};

\item A.W. Bicadze, \textit{Równania fizyki matematycznej};

\item Idith Zertal, \textit{Naród i śmierć. Zagłada w~dyskursie
    i~polityce Izraela};

\item L.S. Pontriagin, \textit{Równania różniczkowe zwyczajne};

\item Kaliński, \textit{Globalizacja, integracja, przedsiębiorczość};

\item Joseph Stiglitz, \textit{Paradoksy prawdopodobieństwa};

\item Jan Michał Małek, \textit{Chrześcijańska myśl ekonomiczna};

\item Richard Pipes, \textit{Czerwone imperium};

\item \textit{Odwieczna Msza. Świadectwa};

\item Nadia Murad, \textit{Ostatnia dziewczyna};

\item Nick Bradley, \textit{Kot w~Tokio};

\item Paweł Lisicki, \textit{Poza polityczną poprawnością};

\item Paweł Lisicki, \textit{Punkt oparcia};

\item Paweł Lisicki, \textit{Kto zabił Jezusa? Prawda i~interpretacja};

\item Paweł Lisicki, \textit{Dżihad i~samozagłada Zachodu};

\item Paweł Dembinski, \textit{Kryzys ekonomiczny i~kryzys wartości};

\item Miachael Cordone Jr., \textit{Firma z~duszą};

\item Paulina Siegień, \textit{Miasto bajka};

\item \textit{Życie Buddy według starych źródeł hinduskich};

\item Ajahn Brahm, \textit{Opowieści buddyjskiego mnicha};

\item Carl Mange, \textit{Zasady ekonomii};

\item Vence Packard, \textit{Hidden Persuaders};

\item Stefan Wyszyński, \textit{Kazania świętokrzyskie};

\item Peter Pomerantsev, \textit{Jądro dziwności. Nowa Rosja};

\item Wolfgang Wickler, \textit{Czy jesteśmy grzesznikami?};

\item Enzo Bianchi, \textit{Jezus i~kobieta};

\item David Cay Johnston, \textit{Donald Trump, jak on to zrobił?};

\item Rafał Kazimierz Wilk, \textit{Człowiek, istota wezwania};

\item \textit{Człowiek żyjący drogą Kościoła};

\item Maciej Leśniewski, \textit{Wojna Burów z~Zulusami 1837-1840};

\item Michel Foucault, \textit{Historia seksualności};

\item Czesław Stanisław Bartnik, \textit{Żyć w~słowie};

\item A. Kauffman Stuart, \textit{Świat poza fizyką};

\item \textit{Przyszło nam tu żyć. Reportaże z~Rosji};

\item Krystian Kratiuk, \textit{Pachamama i~umiłowana Amazonia};

\item Hans Urs Von Balthasar, \textit{Wieńczysz rok darami swojej dobroci};

\item Ludwig von Mises, \textit{Ekonomia i~polityka. Wykład elementarny};

\item \textit{Mama z pasją};

\item Paweł Lisicki, \textit{Kto fałszuje Jezusa?};

\item

\item












































































































\end{enumerate}

% \bibliographystyle{alpha}

% \bibliography{Bibliography}{}










% ############################

% Koniec dokumentu
\end{document}
