% ---------------------------------------------------------------------
% Podstawowe ustawienia i pakiety
% ---------------------------------------------------------------------
\RequirePackage[l2tabu, orthodox]{nag} % Wykrywa przestarzałe i niewłaściwe
% sposoby używania LaTeXa. Więcej jest w l2tabu English version.
\documentclass[a4paper,11pt]{article}
% {rozmiar papieru, rozmiar fontu}[klasa dokumentu]
\usepackage[MeX]{polski} % Polonizacja LaTeXa, bez niej będzie pracował
% w języku angielskim.
\usepackage[utf8]{inputenc} % Włączenie kodowania UTF-8, co daje dostęp
% do polskich znaków.
\usepackage{lmodern} % Wprowadza fonty Latin Modern.
\usepackage[T1]{fontenc} % Potrzebne do używania fontów Latin Modern.



% ---------------------------------------
% Podstawowe pakiety (niezwiązane z ustawieniami języka)
% ---------------------------------------
\usepackage{microtype} % Twierdzi, że poprawi rozmiar odstępów w tekście.
% \usepackage{graphicx} % Wprowadza bardzo potrzebne komendy do wstawiania
% grafiki.
% \usepackage{verbatim} % Poprawia otoczenie VERBATIME.
% \usepackage{textcomp} % Dodaje takie symbole jak stopnie Celsiusa,
% wprowadzane bezpośrednio w tekście.
\usepackage{vmargin} % Pozwala na prostą kontrolę rozmiaru marginesów,
% za pomocą komend poniżej. Rozmiar odstępów jest mierzony w calach.
% ---------------------------------------
% MARGINS
% ---------------------------------------
\setmarginsrb
{ 0.7in}  % left margin
{ 0.6in}  % top margin
{ 0.7in}  % right margin
{ 0.8in}  % bottom margin
{  20pt}  % head height
{0.25in}  % head sep
{   9pt}  % foot height
{ 0.3in}  % foot sep



% ---------------------------------------
% Często używane pakiety
% ---------------------------------------
% \usepackage{csquotes} % Pozwala w prosty sposób wstawiać cytaty do tekstu.
\usepackage{xcolor} % Pozwala używać kolorowych czcionek (zapewne dużo
% więcej, ale ja nie potrafię nic o tym powiedzieć).



% ---------------------------------------
% Pakiety do tekstów z nauk przyrodniczych
% ---------------------------------------
% \let\lll\undefined % Amsmath gryzie się z językiem pakietami do języka
% % polskiego, bo oba definiują komendę \lll. Aby rozwiązać ten problem
% % oddefiniowuję tę komendę, ale może tym samym pozbywam się dużego Ł.
% \usepackage[intlimits]{amsmath} % Podstawowe wsparcie od American
% % Mathematical Society (w skrócie AMS)
% \usepackage{amsfonts, amssymb, amscd, amsthm} % Dalsze wsparcie od AMS
% % \usepackage{siunitx} % Dla prostszego pisania jednostek fizycznych
% \usepackage{upgreek} % Ładniejsze greckie litery
% % Przykładowa składnia: pi = \uppi
% \usepackage{slashed} % Pozwala w prosty sposób pisać slash Feynmana.
% \usepackage{calrsfs} % Zmienia czcionkę kaligraficzną w \mathcal
% % na ładniejszą. Może w innych miejscach robi to samo, ale o tym nic
% % nie wiem.





% ---------------------------------------
% Dodatkowe ustawienia dla języka polskiego
% ---------------------------------------
\renewcommand{\thesection}{\arabic{section}.}
% Kropki po numerach rozdziału (polski zwyczaj topograficzny)
\renewcommand{\thesubsection}{\thesection\arabic{subsection}}
% Brak kropki po numerach podrozdziału



% ---------------------------------------
% Pakiety napisane przez użytkownika.
% Mają być w tym samym katalogu to ten plik .tex
% ---------------------------------------
\usepackage{latexgeneralcommands}
% \usepackage{mathshortcuts}



% ---------------------------------------
% Ustawienia różnych parametrów tekstu
% ---------------------------------------
\renewcommand{\arraystretch}{1.2} % Ustawienie szerokości odstępów między
% wierszami w tabelach.





% ---------------------------------------
% Pakiet „hyperref”
% Polecano by umieszczać go na końcu preambuły.
% ---------------------------------------
\usepackage{hyperref} % Pozwala tworzyć hiperlinki i zamienia odwołania
% do bibliografii na hiperlinki.










% ---------------------------------------------------------------------
% Tytuł, autor, data
\title{Pozycje do przeczytania}

% \author{}


% \date{}
% ---------------------------------------------------------------------










% ####################################################################
% Początek dokumentu
\begin{document}
% ####################################################################





% ######################################
\maketitle  % Tytuł całego tekstu
% ######################################





% ######################################
\section{Przeczytaj}

\vspace{\spaceTwo}
% ######################################





\begin{enumerate}

\item Abraxas;



% \item Zaawansowane książki o~historii fizyki;



% \item Red. L. A. Steen, \textit{Matematyka współczesna. Dwanaście
%     esejów};



% \item N. Bourbaki, \textit{Elementy historii matematyki};



% \item Hugo Steinhaus;



% \item Clifford Truesdell, \textit{Sześć wykładów nowoczesnej filozofii
%   przyrody};



% \item Frank E. Manuel, (książka o~religioznawstwie);



% \item \textit{Polska poezja baroku};



% \item Mauric Blondel;



% \item Leo Strauss, \textit{Prawo naturalne w świetle historii};



% \item Paweł Śpiewak, \textit{Gramsci};



% \item Zdzisław Krasnodębski, \textit{Rozumienie ludzkiego zachowania.
%     Rozważania o~filozoficznych podstawach nauk humanistycznych
%     i~społecznych};



% \item P. Bąk, \textit{Gramatyka języka polskiego. Zarys popularny};



% \item Wassily Kandinsky, \textit{Punkt i~linia a~płaszczyzna. Przyczynek
%     do~analizy elementów malarskich};



% \item Leibniz, \textit{Wyznanie wiary filozofa, Rozprawa metafizyczna;
%     Monadologia; Zasady natury i łaski oraz inne pisma filozoficzne};



% \item Lord Acton, \textit{Historia wolności: wybór esejów};



% \item J. Polit, \textit{Chiny};



% \item Robert Dahl, \textit{Demokracja i~jej krytycy};



% \item Zofia Szmydt, \textit{Transformacja Fouriera i~równania różniczkowe
%     liniowe};



% \item Stefan Ziemba, \textit{Analiza drgań}, dwa tomy;



% \item A.W. Bicadze, \textit{Równania fizyki matematycznej};




























\end{enumerate}

















% ############################

% Koniec dokumentu
\end{document}
